\documentclass{syx7}

% ============================================================================
% PAQUETES NECESARIOS
% ============================================================================
\usepackage{graphicx}
\usepackage{caption}
\usepackage{subcaption}
\usepackage{color}
\usepackage{float}
\usepackage{tikz}
\usepackage{epsfig}
\usepackage{epstopdf}
\usepackage{booktabs}
\usepackage{multirow}
\usepackage{listings}

% Configuración de idioma
\usepackage[spanish]{babel}

% ============================================================================
% CONFIGURACIÓN DE GEOMETRÍA
% ============================================================================
\geometry{
	a4paper,
	left=25mm,
	right=20mm,
	top=25mm,
	bottom=25mm
}

\setlength{\abstractwidth}{\textwidth}
\setlength{\headheight}{59pt}

% ============================================================================
% CONFIGURACIÓN DE BIBLIOGRAFÍA
% ============================================================================
\usepackage[backend=bibtex,style=ieee,natbib=true]{biblatex}
\defbibheading{bibliography}{\section*{Referencias}}
\addbibresource{biblio.bib}

\patchcmd{\bibsetup}{\begingroup}{\begingroup\let\clearpage\relax}{}{}

% ============================================================================
% CONFIGURACIÓN DE HYPERREF
% ============================================================================
\hypersetup{
	colorlinks=true,
	linkcolor=black,
	citecolor=blue,
	urlcolor=blue,
	filecolor=blue,
	pdfborder={0 0 0},
	breaklinks=true
}

% ============================================================================
% CONFIGURACIÓN DE CÓDIGO
% ============================================================================
\lstset{
	basicstyle=\ttfamily\small,
	breaklines=true,
	frame=single,
	language=Python,
	showstringspaces=false,
	commentstyle=\color{gray},
	keywordstyle=\color{blue},
	stringstyle=\color{red}
}

% ============================================================================
% METADATOS DEL DOCUMENTO
% ============================================================================
\receiveddate{01-nov-2024}
\accepteddate{20-nov-2024}

\title{Sistema Modular en Python para Modelado Cinético de Biodiesel: Alternativa de Código Abierto a Software Comercial}
\shorttitle{Sistema Python para Modelado de Biodiesel}

\keywords{biodiesel, transesterificación, modelado cinético, Python, código abierto, educación química, optimización de procesos}

% ============================================================================
% DOCUMENTO
% ============================================================================
\begin{document}

\author{Autor Principal*}
\email{autor@universidad.edu}

\vspace*{-1 \baselineskip}

\maketitle

% ============================================================================
% RESUMEN
% ============================================================================
\begin{abstract}
El software comercial para simulación de procesos químicos como ASPEN Plus representa costos que oscilan entre quince mil y cincuenta mil dólares estadounidenses anuales por licencia, lo cual limita significativamente el acceso a estas herramientas para instituciones educativas y laboratorios de investigación con recursos limitados. Este trabajo presenta un sistema modular completo desarrollado en Python de código abierto para el modelado cinético de producción de biodiesel mediante transesterificación de aceite de cocina usado. El sistema integra el procesamiento de datos experimentales obtenidos mediante cromatografía de gases con detector de ionización de llama, el ajuste de parámetros cinéticos mediante algoritmos de regresión no lineal, la optimización de condiciones operacionales y la implementación de criterios de escalado desde reactores de laboratorio de trescientos cincuenta mililitros hasta escala piloto de veinte litros. Se desarrolló además una metodología educativa estructurada en doce prácticas progresivas que abarcan desde conceptos fundamentales de Python hasta técnicas avanzadas de dinámica de fluidos computacional, diseñada para personas sin conocimientos previos de programación. La validación del modelo mediante comparación con datos experimentales reportados en la literatura científica muestra desviaciones inferiores al cinco por ciento en los parámetros cinéticos determinados. Un análisis exhaustivo de sensibilidad realizado mediante diseño de experimentos identifica la temperatura como el parámetro que ejerce mayor influencia sobre la conversión, seguido por la relación molar metanol a triglicérido. El sistema permite la configuración completa de todos los parámetros mediante archivos en formato JSON, facilitando la adaptación del modelo a diferentes materias primas, catalizadores y geometrías de reactor sin necesidad de modificar el código fuente. Se incluye documentación detallada del proceso de instalación y configuración en sistemas operativos Windows. Este trabajo demuestra que las herramientas de código abierto pueden constituir una alternativa técnicamente viable, completamente transparente y económicamente accesible frente al software comercial propietario, democratizando el acceso a tecnologías de modelado matemático para la comunidad científica internacional.
\end{abstract}

% ============================================================================
% 1. INTRODUCCIÓN
% ============================================================================
\section{Introducción}

\subsection{Contexto del Biodiesel de Segunda Generación}

El biodiesel producido mediante transesterificación de aceites de cocina usados constituye una alternativa energética sostenible que permite valorizar residuos lipídicos mientras reduce la dependencia de combustibles fósiles. La reacción de transesterificación convierte los triglicéridos presentes en aceites vegetales en ésteres metílicos de ácidos grasos mediante el uso de metanol como agente alquilante y un catalizador alcalino, típicamente hidróxido de sodio o hidróxido de potasio. Esta transformación química requiere una comprensión profunda de la cinética de reacción, la influencia de las variables operacionales tales como temperatura, intensidad de agitación y relación molar de reactivos, así como de los criterios ingenieriles que rigen el escalado desde el nivel de laboratorio hasta escala industrial \cite{kouzu2008}.

El modelado matemático de este proceso resulta esencial para predecir con precisión las conversiones alcanzables bajo diferentes condiciones operacionales, optimizar los parámetros del proceso para maximizar el rendimiento mientras se minimizan los costos, reducir significativamente el número de experimentos necesarios durante la etapa de desarrollo, y diseñar reactores a escala piloto e industrial con criterios técnicamente fundamentados. La capacidad de simular virtualmente el comportamiento del sistema reactivo antes de realizar inversiones en equipamiento representa una ventaja estratégica considerable en términos de tiempo y recursos económicos.

\subsection{Panorama del Software Comercial de Simulación}

El software de simulación de procesos químicos de tipo comercial, donde destacan paquetes como ASPEN Plus, HYSYS y gPROMS, ha sido ampliamente utilizado en la industria química durante las últimas décadas \cite{aspen2023}. Estas herramientas ofrecen capacidades avanzadas de modelado termodinámico, bases de datos extensas de propiedades fisicoquímicas e interfaces gráficas que facilitan la construcción de diagramas de flujo de proceso. Sin embargo, el costo de adquisición y mantenimiento anual de estas licencias oscila entre quince mil y cincuenta mil dólares estadounidenses por usuario, lo cual representa una barrera económica considerable. Adicionalmente, el tiempo requerido para que un usuario adquiera competencia en el manejo de estas plataformas suele extenderse entre dos y cuatro semanas de entrenamiento intensivo, debido a la complejidad inherente de las interfaces y la multiplicidad de opciones de configuración disponibles.

El acceso al código fuente de estos programas está completamente restringido, operando bajo un modelo de "caja negra" donde los algoritmos de cálculo, correlaciones termodinámicas y métodos numéricos implementados permanecen ocultos al usuario final. Esta falta de transparencia dificulta la comprensión exacta de los cálculos realizados, impide la modificación de ecuaciones para casos específicos que no estén contemplados en las bibliotecas estándar, obstaculiza la integración de algoritmos desarrollados por el usuario, y complica la publicación de metodologías completamente reproducibles en revistas científicas. La personalización del software se limita a las funcionalidades expuestas mediante interfaces de programación de aplicaciones complejas, y la integración directa con datos experimentales propios frecuentemente requiere módulos adicionales o procesamiento previo en formatos específicos.

El costo elevado de estas licencias limita particularmente el acceso para universidades ubicadas en países en desarrollo con presupuestos restringidos, laboratorios de investigación que operan con financiamiento limitado, pequeñas y medianas empresas que no pueden justificar económicamente la inversión, y personas que desean practicar o realizar cálculos fuera de las instalaciones institucionales donde existe la licencia. El modelo de renovación anual obligatorio genera además una dependencia continua del proveedor del software.

\subsection{Alternativas de Código Abierto Disponibles}

En el ecosistema de código abierto existen algunas herramientas que ofrecen capacidades de simulación sin costo de licencia, entre las cuales destacan DWSIM \cite{dwsim2023}, COCO Simulator \cite{coco2023} y Cantera \cite{cantera2023}. DWSIM proporciona un entorno de simulación de procesos similar conceptualmente a ASPEN Plus, mientras que Cantera se especializa en cinética química y reacciones en fase gaseosa. Sin embargo, estas herramientas presentan desafíos significativos para su adopción masiva. Las interfaces de usuario suelen ser complejas y requieren experiencia previa en simulación de procesos, la documentación disponible está predominantemente en idioma inglés con escaso material en español, la integración directa con datos experimentales obtenidos en laboratorio no está suficientemente desarrollada, y existe una notable ausencia de metodologías educativas estructuradas que guíen progresivamente a usuarios novatos desde los fundamentos hasta aplicaciones avanzadas.

Para científicos experimentales cuya formación se centra en química y no incluye programación computacional, estas herramientas pueden resultar tan inaccesibles como el software comercial debido a la curva de aprendizaje pronunciada. La falta de ejemplos específicos aplicados a sistemas de interés concreto, como la producción de biodiesel, dificulta adicionalmente la transferencia tecnológica desde estas plataformas genéricas hacia aplicaciones particulares.

\subsection{Objetivos y Alcance del Presente Trabajo}

Este trabajo busca cerrar la brecha existente entre el software comercial de alto costo pero limitada accesibilidad y las herramientas de código abierto disponibles pero de difícil adopción. El objetivo principal consiste en desarrollar un sistema modular completo implementado íntegramente en lenguaje Python, de distribución libre bajo licencia de código abierto, específicamente diseñado para el modelado cinético de producción de biodiesel mediante transesterificación. Este sistema debe ser capaz de procesar datos experimentales reales, ajustar parámetros cinéticos mediante algoritmos robustos, optimizar condiciones operacionales y proporcionar criterios de escalado desde laboratorio hasta planta piloto.

La validación rigurosa del modelo mediante comparación con datos experimentales publicados en literatura científica revisada por pares constituye un objetivo secundario fundamental para establecer la confiabilidad de las predicciones realizadas. Se plantea además el desarrollo de una metodología educativa estructurada en doce prácticas progresivas que conduzcan desde los fundamentos de programación en Python hasta técnicas avanzadas de dinámica de fluidos computacional aplicada al diseño de reactores. Esta metodología debe estar diseñada específicamente para personas sin conocimientos previos de programación, proporcionando una transición gradual que permita la adquisición de competencias técnicas de forma natural.

La demostración de la versatilidad del sistema mediante su aplicación a diferentes configuraciones de reactor, diversos catalizadores y distintas materias primas lipídicas constituye otro objetivo relevante. La documentación exhaustiva de todos los componentes del sistema, incluyendo procedimientos de instalación, configuración de parámetros y ejecución de cálculos, garantizará la reproducibilidad completa de los resultados obtenidos. Finalmente, se busca proporcionar a la comunidad científica internacional una alternativa que sea económicamente accesible por su costo nulo, técnicamente sólida por su fundamentación en métodos numéricos establecidos, y completamente transparente por el acceso irrestricto al código fuente.

Todo el código fuente desarrollado, junto con la documentación completa y los datos de ejemplo, se encuentra disponible públicamente en un repositorio de control de versiones, permitiendo que cualquier investigador pueda auditar los métodos implementados, modificar el código para adaptarlo a sus necesidades específicas, contribuir con mejoras o extensiones, y extender las capacidades del sistema según los requerimientos particulares de su investigación.

% ============================================================================
% 2. METODOLOGÍA
% ============================================================================
\section{Metodología}

\subsection{Arquitectura Modular del Sistema}

El sistema desarrollado se estructura siguiendo principios de ingeniería de software modular, donde cada componente funcional se implementa como un módulo independiente con responsabilidades claramente delimitadas. Esta arquitectura facilita tanto el mantenimiento del código como la extensión de capacidades mediante la adición de nuevos módulos sin afectar la funcionalidad existente. La estructura de directorios principal del repositorio se organiza de la siguiente manera: el directorio raíz contiene el archivo principal \texttt{main.py} que constituye el punto de entrada al sistema, la carpeta \texttt{src/} alberga todo el código fuente organizado en submódulos, la carpeta \texttt{practicas/} contiene las doce prácticas educativas con sus respectivos archivos de configuración y datos de ejemplo, el directorio \texttt{config/} almacena los archivos de configuración global en formato JSON, y la carpeta \texttt{datos/} contiene conjuntos de datos experimentales de referencia.

Dentro del directorio \texttt{src/}, la organización modular se implementa mediante cuatro subdirectorios principales. El módulo \texttt{src/data\_processing/} contiene la clase \texttt{GCProcessor} implementada en el archivo \texttt{gc\_processor.py}, que se encarga del procesamiento completo de datos experimentales obtenidos mediante cromatografía de gases. Este módulo lee archivos en formato CSV que contienen las áreas de los picos cromatográficos, aplica los factores de respuesta relativos configurados, calcula las concentraciones molares de cada especie química presente, y genera archivos de salida con los perfiles temporales de concentración. La clase acepta como entrada un diccionario de configuración donde se especifican las columnas del archivo CSV que corresponden a cada especie, los factores de respuesta relativos respecto al estándar interno, y las masas molares de todos los componentes.

El módulo \texttt{src/models/} contiene dos archivos principales. El archivo \texttt{kinetic\_model.py} implementa la clase \texttt{KineticModel} que define el modelo cinético de la reacción de transesterificación. Esta clase contiene métodos para calcular la constante de velocidad mediante la ecuación de Arrhenius, definir el sistema de ecuaciones diferenciales ordinarias que describe la evolución temporal de las concentraciones, y resolver numéricamente este sistema utilizando el integrador \texttt{odeint} de la biblioteca SciPy. El segundo archivo, \texttt{thermodynamics.py}, proporciona funciones auxiliares para el cálculo de propiedades físicas dependientes de la temperatura, tales como densidad, viscosidad y calor específico, mediante correlaciones polinómicas cuyos coeficientes se obtienen de la literatura.

El módulo \texttt{src/optimization/} se subdivide en dos componentes especializados. El archivo \texttt{parameter\_fitter.py} implementa la clase \texttt{ParameterFitter} que realiza el ajuste de parámetros cinéticos mediante dos estrategias algorítmicas alternativas. La primera estrategia utiliza el algoritmo de Levenberg-Marquardt implementado en la biblioteca \texttt{lmfit}, que resulta eficiente cuando se dispone de una estimación inicial razonable de los parámetros y los datos experimentales presentan bajo nivel de ruido. Este algoritmo minimiza la suma de residuos cuadrados entre los valores experimentales y los predichos por el modelo, calculando además intervalos de confianza para cada parámetro ajustado y estadísticas de bondad de ajuste tales como el coeficiente de determinación $R^2$ y la raíz del error cuadrático medio. La segunda estrategia implementa el algoritmo de evolución diferencial de la biblioteca \texttt{scipy.optimize}, que realiza una búsqueda global en el espacio de parámetros y resulta más robusta ante la presencia de múltiples mínimos locales, aunque con mayor costo computacional. El archivo \texttt{operational\_optimizer.py} contiene la clase \texttt{OperationalOptimizer} que determina las condiciones operacionales óptimas del proceso. Esta clase acepta como entrada una función objetivo definida por el usuario, restricciones en los valores de las variables de operación tales como temperatura máxima permisible o relación molar máxima económicamente viable, y explora el espacio de búsqueda utilizando algoritmos de optimización global para identificar la combinación de variables que maximiza la conversión mientras minimiza el tiempo de reacción y el consumo de reactivos.

El módulo \texttt{src/visualization/} implementado en el archivo \texttt{plotter.py} proporciona funciones especializadas para la generación de gráficas de alta calidad utilizando las bibliotecas Matplotlib para gráficas estáticas y Plotly para visualizaciones interactivas. Las funciones generan automáticamente gráficas de evolución temporal de concentraciones con múltiples series superpuestas, superficies de respuesta tridimensionales para análisis de sensibilidad, diagramas de contorno para visualización de regiones óptimas, y diagramas de Pareto para identificación de factores más influyentes. Todas las gráficas se guardan automáticamente en formato PNG con resolución de 300 dpi y adicionalmente en formato HTML interactivo cuando se utiliza Plotly.

El flujo de datos a través del sistema sigue una secuencia unidireccional bien definida. Inicialmente, los datos experimentales crudos obtenidos del cromatógrafo se almacenan en archivos CSV en el directorio \texttt{datos/experimentales/}. El módulo \texttt{GCProcessor} lee estos archivos, procesa las áreas de los picos aplicando las calibraciones correspondientes, y genera archivos de salida con las concentraciones calculadas. Estos datos procesados se pasan al módulo \texttt{ParameterFitter}, que ejecuta el algoritmo de ajuste de parámetros cinéticos. Los parámetros óptimos obtenidos se almacenan en un archivo de configuración JSON en el directorio \texttt{config/parametros\_ajustados.json}. Posteriormente, el módulo \texttt{OperationalOptimizer} utiliza estos parámetros validados para explorar diferentes condiciones operacionales y determinar la configuración óptima del proceso. Finalmente, el módulo de visualización genera todas las gráficas requeridas y las almacena en el directorio \texttt{resultados/} junto con archivos de texto que contienen resúmenes estadísticos numéricos.

Esta arquitectura modular permite ejecutar cada etapa del análisis de forma independiente, facilitando la depuración del código, la validación de resultados intermedios, y la reutilización de componentes en diferentes contextos. Por ejemplo, la práctica número cinco del material educativo utiliza exclusivamente el módulo \texttt{GCProcessor} para introducir el concepto de procesamiento de datos cromatográficos, mientras que la práctica número seis combina este módulo con el \texttt{ParameterFitter} para ilustrar el proceso completo de ajuste de parámetros cinéticos. La práctica número ocho integra todos los módulos en un flujo de trabajo completo que va desde los datos crudos hasta la optimización operacional, demostrando la integración de todos los componentes del sistema.

\subsection{Fundamentos del Modelo Cinético de Transesterificación}

La reacción de transesterificación de triglicéridos con metanol en presencia de catalizador alcalino homogéneo puede representarse mediante diferentes niveles de complejidad mecanística. El modelo más simple considera la reacción global directa donde un mol de triglicérido reacciona con tres moles de metanol para producir tres moles de éster metílico de ácido graso y un mol de glicerol, según la estequiometría representada en la ecuación \ref{eq:reaccion_global}. Este modelo simplificado resulta adecuado cuando el objetivo principal consiste en predecir la conversión final del proceso sin necesidad de cuantificar las especies intermedias.

\begin{equation}
	\text{TG} + 3\,\text{CH}_3\text{OH} \xrightarrow{k} 3\,\text{FAME} + \text{GL}
	\label{eq:reaccion_global}
\end{equation}

Sin embargo, el mecanismo real de la transesterificación procede mediante tres etapas consecutivas reversibles. En la primera etapa, el triglicérido reacciona con una molécula de metanol para formar diglicérido y el primer éster metílico, según se indica en la ecuación \ref{eq:etapa1}. La segunda etapa convierte el diglicérido en monoglicérido con liberación de una segunda molécula de éster metílico, como muestra la ecuación \ref{eq:etapa2}. Finalmente, el monoglicérido reacciona con una tercera molécula de metanol para producir glicerol y la tercera molécula de éster metílico, según la ecuación \ref{eq:etapa3}. Cada una de estas etapas posee sus propias constantes de velocidad directa e inversa, y la reversibilidad de las reacciones puede afectar significativamente el rendimiento máximo alcanzable, especialmente cuando se opera con relaciones molares metanol a triglicérido cercanas a la estequiométrica.

\begin{align}
	\text{TG} + \text{CH}_3\text{OH} &\xrightleftharpoons[k_{-1}]{k_1} \text{DG} + \text{FAME} \label{eq:etapa1} \\
	\text{DG} + \text{CH}_3\text{OH} &\xrightleftharpoons[k_{-2}]{k_2} \text{MG} + \text{FAME} \label{eq:etapa2} \\
	\text{MG} + \text{CH}_3\text{OH} &\xrightleftharpoons[k_{-3}]{k_3} \text{GL} + \text{FAME} \label{eq:etapa3}
\end{align}

El sistema implementado en el archivo \texttt{src/models/kinetic\_model.py} permite al usuario seleccionar mediante el parámetro \texttt{modelo\_tipo} en el archivo de configuración \texttt{config/modelo\_cinetico.json} si desea utilizar el modelo simplificado de un solo paso o el modelo detallado de tres pasos reversibles. Esta flexibilidad resulta particularmente útil en contextos educativos, donde la práctica número cuatro introduce inicialmente el modelo de un paso para facilitar la comprensión de los conceptos fundamentales de cinética química, mientras que la práctica número doce utiliza el modelo de tres pasos para demostrar el impacto de considerar especies intermedias y reversibilidad en las predicciones de conversión.

La dependencia de las constantes de velocidad con la temperatura se modela mediante la ecuación de Arrhenius, que relaciona exponencialmente la constante de velocidad con la energía de activación y la temperatura absoluta según la expresión mostrada en la ecuación \ref{eq:arrhenius}. En esta ecuación, $k$ representa la constante de velocidad expresada en litros por mol por minuto, $A$ constituye el factor pre-exponencial con las mismas unidades dimensionales que $k$, $E_a$ denota la energía de activación expresada en J/mol, $R$ es la constante universal de los gases ideales con valor de 8.314 J/(mol·K), y $T$ representa la temperatura absoluta del sistema reactivo medida en kelvin.

\begin{equation}
	k(T) = A \exp\left(-\frac{E_a}{RT}\right)
	\label{eq:arrhenius}
\end{equation}

El factor pre-exponencial $A$ se interpreta físicamente como la frecuencia con la cual las moléculas colisionan con la orientación apropiada para reaccionar, mientras que el término exponencial que contiene la energía de activación representa la fracción de colisiones que poseen energía suficiente para superar la barrera energética de la reacción. Valores elevados de energía de activación resultan en una mayor sensibilidad de la constante de velocidad ante cambios de temperatura, lo cual se manifiesta en pendientes más pronunciadas en los gráficos de Arrhenius que representan el logaritmo natural de la constante de velocidad versus el inverso de la temperatura absoluta, como se ilustra en la práctica número dos del material educativo.

El método implementado en la clase \texttt{KineticModel} para el cálculo de la constante de velocidad se encuentra en la función \texttt{calcular\_constante\_velocidad()}, que recibe como argumentos el factor pre-exponencial, la energía de activación y la temperatura, retornando el valor numérico de la constante calculado mediante la ecuación de Arrhenius. Esta función se invoca repetidamente durante la integración numérica del sistema de ecuaciones diferenciales cuando se consideran perfiles de temperatura variables en el tiempo, como ocurre en procesos no isotérmicos donde existe control deficiente de temperatura o en operaciones de arranque del reactor donde la temperatura se eleva gradualmente desde la temperatura ambiente hasta la temperatura de operación deseada.

\subsection{Sistema de Ecuaciones Diferenciales Ordinarias}

Para el caso del modelo cinético simplificado de un solo paso con catalizador alcalino homogéneo, se asume que la velocidad de reacción sigue una cinética de orden global cuatro, siendo de primer orden respecto al triglicérido y de tercer orden respecto al metanol. Esta selección del orden de reacción se fundamenta en observaciones experimentales reportadas extensamente en la literatura científica \cite{freedman1986,noureddini1997}, donde se ha demostrado que esta expresión cinética reproduce adecuadamente los perfiles de concentración medidos experimentalmente para catalizadores alcalinos en fase homogénea. La velocidad de reacción se expresa entonces como el producto de la constante de velocidad por la concentración molar de triglicérido y el cubo de la concentración molar de metanol, según se muestra en la ecuación \ref{eq:velocidad_reaccion}.

\begin{equation}
	r = k \, C_{\text{TG}} \, C_{\text{MeOH}}^3
	\label{eq:velocidad_reaccion}
\end{equation}

El balance de materia para cada especie química en un reactor batch perfectamente mezclado operando isotérmicamente conduce a un sistema de cuatro ecuaciones diferenciales ordinarias acopladas. La variación temporal de la concentración de triglicérido se describe mediante la ecuación \ref{eq:edo_tg}, donde el signo negativo indica que el triglicérido se consume durante la reacción. La variación de la concentración de metanol, presentada en la ecuación \ref{eq:edo_meoh}, incluye el factor estequiométrico de tres debido a que se consumen tres moles de metanol por cada mol de triglicérido que reacciona. Las concentraciones de los productos, éster metílico y glicerol, aumentan con el tiempo según las ecuaciones \ref{eq:edo_fame} y \ref{eq:edo_gl} respectivamente, reflejando sus coeficientes estequiométricos de producción.

\begin{align}
	\frac{dC_{\text{TG}}}{dt} &= -k \, C_{\text{TG}} \, C_{\text{MeOH}}^3 \label{eq:edo_tg} \\
	\frac{dC_{\text{MeOH}}}{dt} &= -3k \, C_{\text{TG}} \, C_{\text{MeOH}}^3 \label{eq:edo_meoh} \\
	\frac{dC_{\text{FAME}}}{dt} &= 3k \, C_{\text{TG}} \, C_{\text{MeOH}}^3 \label{eq:edo_fame} \\
	\frac{dC_{\text{GL}}}{dt} &= k \, C_{\text{TG}} \, C_{\text{MeOH}}^3 \label{eq:edo_gl}
\end{align}

Para el modelo de tres pasos reversibles, el sistema de ecuaciones diferenciales se expande considerablemente. Se deben incluir ecuaciones para las concentraciones de diglicérido y monoglicérido como especies intermedias, y cada ecuación debe incorporar tanto los términos de producción como los de consumo asociados a las reacciones directa e inversa de las tres etapas. La ecuación diferencial para el triglicérido incluye únicamente el término de consumo por la primera etapa más el término de regeneración por la reacción inversa de la misma etapa. El diglicérido se produce en la primera etapa y se consume en la segunda, con contribuciones adicionales de las respectivas reacciones inversas. El monoglicérido se genera en la segunda etapa y se consume en la tercera. El éster metílico se produce en las tres etapas. El sistema completo consta de seis ecuaciones diferenciales acopladas y requiere el conocimiento de seis constantes cinéticas.

La resolución numérica de estos sistemas de ecuaciones diferenciales ordinarias se realiza mediante el integrador \texttt{odeint} de la biblioteca SciPy de Python, implementado en el método \texttt{resolver\_sistema()} de la clase \texttt{KineticModel}. Este integrador utiliza el método de diferencias finitas backward differentiation formulas, que resulta particularmente apropiado para sistemas rígidos donde las constantes de velocidad de diferentes pasos pueden diferir en varios órdenes de magnitud. El método ajusta automáticamente el tamaño del paso de integración para garantizar que el error de truncamiento local permanezca por debajo de tolerancias especificadas por el usuario, las cuales se configuran mediante los parámetros \texttt{rtol} y \texttt{atol} en el archivo \texttt{config/integracion\_numerica.json}. Un valor típico de tolerancia relativa es $10^{-6}$, mientras que la tolerancia absoluta suele fijarse en $10^{-9}$, lo cual garantiza una precisión adecuada para la mayoría de aplicaciones prácticas sin incurrir en costos computacionales excesivos.

Las condiciones iniciales para el sistema de ecuaciones diferenciales se especifican mediante las concentraciones molares iniciales de todas las especies. Para un reactor batch típico, la concentración inicial de triglicérido se calcula dividiendo la masa de aceite cargada al reactor entre su masa molar promedio y el volumen del reactor, mientras que la concentración inicial de metanol se determina similarmente a partir de su masa cargada. Las concentraciones iniciales de productos y especies intermedias son típicamente cero para un reactor batch que inicia operación, aunque el sistema permite especificar valores no nulos para simular condiciones de recarga parcial de reactor. Estas condiciones iniciales se ingresan mediante el archivo de configuración \texttt{config/condiciones\_iniciales.json}, donde se especifican tanto las masas cargadas como las masas molares y el volumen del reactor, permitiendo que el sistema calcule automáticamente las concentraciones molares correspondientes.

El vector de tiempo para la integración se genera mediante la función \texttt{numpy.linspace()} que crea un arreglo de valores uniformemente espaciados entre el tiempo inicial 0 y el tiempo final especificado por el usuario. El número de puntos temporales se configura mediante el parámetro \texttt{num\_puntos\_tiempo} en el archivo de configuración, siendo un valor típico de 100 puntos para simulaciones estándar, aunque puede incrementarse cuando se requiere mayor resolución temporal para capturar fenómenos transitorios rápidos. El integrador \texttt{odeint} retorna una matriz bidimensional donde cada fila corresponde a un instante de tiempo y cada columna representa la concentración de una especie química, facilitando el postprocesamiento de resultados y la generación de gráficas mediante las funciones del módulo de visualización.

La práctica número cuatro del material educativo guía en la construcción paso a paso del sistema de ecuaciones diferenciales, la configuración de parámetros de integración, y la interpretación de los perfiles de concentración obtenidos. Se proporcionan ejemplos concretos donde se varía la temperatura de reacción para observar su efecto en la velocidad de conversión, ilustrando cuantitativamente la dependencia exponencial predicha por la ecuación de Arrhenius. La práctica número doce permite comparar las predicciones del modelo de un paso versus el modelo de tres pasos, evidenciando las diferencias en la conversión final alcanzable y la importancia de considerar la reversibilidad de las reacciones cuando se opera con relaciones molares bajas de metanol a triglicérido.

\subsection{Sistema de Configuración mediante Archivos JSON}

La configuración de todos los parámetros del sistema se realiza mediante archivos en formato JSON almacenados en el directorio \texttt{config/} del repositorio. Esta decisión de diseño responde a la necesidad de separar completamente los parámetros del proceso del código fuente del modelo, permitiendo modificar condiciones experimentales, propiedades fisicoquímicas y opciones de cálculo sin necesidad de editar archivos de código Python. El formato JSON fue seleccionado por su legibilidad para humanos, su capacidad de validación mediante esquemas, y su soporte nativo en Python mediante la biblioteca estándar \texttt{json}.

El archivo principal de configuración se denomina \texttt{config/config\_principal.json} y contiene la estructura jerárquica de todos los parámetros del sistema. Cada sección del archivo JSON se dedica a un aspecto particular de la configuración. La sección \texttt{masas\_molares} contiene las masas molares expresadas en g/mol de todas las especies químicas involucradas en la reacción. Para el triglicérido, se utiliza típicamente la masa molar de la tripalmitina con valor de 807.3 g/mol, aunque este valor puede modificarse para reflejar la composición específica del aceite utilizado. La masa molar del metanol es 32.04 g/mol, la del éster metílico de ácido palmítico es 270.5 g/mol, y la del glicerol es 92.09 g/mol. Cada valor numérico está acompañado por un campo \texttt{\_fuente} que documenta la procedencia del dato, típicamente indicando la base de datos PubChem con el identificador de compuesto correspondiente.

La sección \texttt{densidades\_25C} especifica las densidades de todos los componentes a temperatura de referencia de 25 °C, expresadas en g/mL. Estos valores se utilizan para convertir masas cargadas al reactor en volúmenes correspondientes, lo cual resulta necesario para el cálculo de concentraciones molares. La tripalmitina posee una densidad de 0.852 g/mL, el metanol tiene densidad de 0.792 g/mL, el éster metílico presenta densidad de 0.865 g/mL, y el glicerol exhibe densidad de 1.261 g/mL. Nuevamente, cada valor incluye su campo \texttt{\_fuente} que referencia el Manual del Ingeniero Químico de Perry en su novena edición.

La sección \texttt{reactor} contiene los parámetros geométricos y operacionales del reactor batch. El volumen del reactor se especifica en mL, siendo 350 mL el valor típico para reactores de laboratorio utilizados en las prácticas educativas. Este parámetro puede modificarse para simular reactores de diferentes escalas, como se ejemplifica en la práctica número nueve donde se escala a 20 L. La geometría del reactor incluye adicionalmente el diámetro interno del tanque y la altura del líquido, parámetros que resultan necesarios para el cálculo de tiempos de mezclado y para el diseño de elementos de agitación.

La sección \texttt{condiciones\_operacionales} especifica las variables de proceso que pueden optimizarse. La temperatura de operación se ingresa en grados Celsius y el sistema la convierte internamente a kelvin para los cálculos cinéticos. El rango típico de temperaturas explorado varía entre 50 y 70 °C, balanceando la velocidad de reacción contra la evaporación de metanol y el consumo energético. La relación molar de metanol a triglicérido se expresa como un número real, siendo valores típicos entre 6 y 12. La concentración de catalizador se especifica como porcentaje en peso respecto a la masa de aceite, con valores usuales entre 0.5 y 2%.

El archivo de configuración incluye además una sección \texttt{perfil\_agitacion} que permite especificar la evolución temporal de la velocidad de agitación del reactor. Esta funcionalidad resulta particularmente útil para simular operaciones donde la agitación se incrementa gradualmente durante el arranque del reactor o se modifica durante el transcurso de la reacción. El perfil se define mediante una lista de puntos que especifican pares de tiempo en minutos y revoluciones por minuto. El sistema interpola linealmente entre estos puntos cuando el parámetro \texttt{tipo} se configura como \texttt{lineal}, o mantiene valores constantes por tramos cuando se selecciona el modo \texttt{escalonado}. La práctica número cinco ilustra el impacto de diferentes perfiles de agitación en la velocidad de reacción mediante la comparación de tres casos: agitación constante a velocidad baja, agitación constante a velocidad alta, y perfil de agitación variable que inicia a 300 rpm y se incrementa linealmente hasta 600 rpm al cabo de 60 min.

La validación automática de los archivos de configuración se implementa mediante esquemas JSON definidos en el archivo \texttt{config/esquema\_validacion.json}. Este esquema especifica los tipos de datos permitidos para cada campo, rangos válidos de valores numéricos, y campos obligatorios versus opcionales. Cuando se ejecuta el sistema mediante el comando \texttt{python main.py}, la primera acción realizada consiste en cargar el archivo de configuración y validarlo contra el esquema. Si se detectan errores tales como tipos de datos incorrectos, valores fuera de rango, o campos obligatorios faltantes, el sistema genera un mensaje de error descriptivo indicando exactamente qué parámetro presenta problemas y cuál es el valor o formato esperado. Esta validación temprana previene errores durante la ejecución del modelo y facilita la depuración de problemas de configuración.

La modificación de parámetros para explorar diferentes escenarios se realiza editando el archivo JSON con cualquier editor de texto o, preferiblemente, con editores que ofrecen resaltado de sintaxis y validación en tiempo real como Visual Studio Code. La práctica número uno del material educativo introduce el concepto de archivos de configuración JSON mediante un ejemplo simple donde se modifican únicamente la masa de triglicérido cargada y la relación molar de metanol, observando el efecto de estos cambios en los cálculos estequiométricos. Progresivamente, las prácticas subsiguientes van incorporando secciones adicionales del archivo de configuración, culminando en la práctica número ocho donde se utiliza la configuración completa para ejecutar el flujo de trabajo integrado desde el procesamiento de datos hasta la optimización operacional.

\subsection{Procesamiento de Datos de Cromatografía de Gases}

El módulo de procesamiento de datos cromatográficos implementado en el archivo \texttt{src/data\_processing/gc\_processor.py} mediante la clase \texttt{GCProcessor} constituye el punto de entrada para integrar datos experimentales reales dentro del flujo de modelado. La cromatografía de gases con detector de ionización de llama representa la técnica analítica estándar para cuantificar la composición de mezclas de ésteres metílicos, permitiendo determinar las concentraciones de triglicérido sin reaccionar, especies intermedias diglicérido y monoglicérido, éster metílico producido, y glicerol generado como subproducto.

El flujo de procesamiento inicia con la adquisición de datos crudos del cromatógrafo, los cuales se exportan típicamente en formato texto delimitado por comas. Estos archivos CSV se almacenan en el directorio \texttt{datos/experimentales/gc\_raw/} y contienen columnas para el tiempo de muestreo expresado en minutos desde el inicio de la reacción, y las áreas integradas de los picos cromatográficos correspondientes a cada especie química. Un archivo CSV típico contiene adicionalmente una columna para el área del pico del estándar interno, que en este sistema se seleccionó como metil heptadecanoato por su ausencia de interferencia con las especies de interés y su estabilidad química bajo las condiciones de análisis.

El método \texttt{cargar\_datos\_crudos()} de la clase \texttt{GCProcessor} lee el archivo CSV especificado utilizando la biblioteca pandas de Python, que maneja eficientemente la lectura de archivos tabulares y proporciona estructuras de datos tipo DataFrame para manipulación posterior. Durante la carga, el método verifica que el archivo contenga todas las columnas esperadas según lo especificado en la sección \texttt{columnas\_csv} del archivo de configuración \texttt{config/config\_gc.json}. Si alguna columna falta o presenta un nombre incorrecto, el sistema genera un mensaje de error descriptivo indicando las columnas encontradas versus las esperadas.

La conversión de áreas de picos cromatográficos a concentraciones molares requiere el conocimiento de los factores de respuesta relativos de cada componente respecto al estándar interno. Estos factores se determinan experimentalmente mediante el análisis de soluciones patrón de concentración conocida y se almacenan en la sección \texttt{factores\_respuesta} del archivo de configuración. El factor de respuesta relativo $f_i$ de la especie $i$ se define matemáticamente según la ecuación \ref{eq:factor_respuesta}, donde $A_i$ representa el área del pico de la especie de interés, $A_{std}$ es el área del pico del estándar interno, $C_i$ denota la concentración molar de la especie de interés, y $C_{std}$ representa la concentración molar del estándar interno.

\begin{equation}
	f_i = \frac{A_i / A_{std}}{C_i / C_{std}}
	\label{eq:factor_respuesta}
\end{equation}

Una vez conocidos los factores de respuesta, la concentración de cada especie en las muestras analizadas se calcula mediante la ecuación \ref{eq:concentracion_gc}, que invierte la relación anterior. Esta ecuación permite calcular la concentración molar de cualquier especie conociendo el área de su pico, el área del pico del estándar, la concentración conocida del estándar en la muestra, y el factor de respuesta previamente determinado.

\begin{equation}
	C_i = \frac{A_i}{A_{std}} \times \frac{C_{std}}{f_i}
	\label{eq:concentracion_gc}
\end{equation}

El método \texttt{calcular\_concentraciones()} implementa este cálculo para todas las filas del DataFrame de pandas que contiene los datos crudos, aplicando vectorizadamente las operaciones aritméticas necesarias. La concentración del estándar interno $C_{std}$ se calcula a partir de la masa conocida de estándar agregada a cada muestra y el volumen de la muestra analizada, valores que se especifican en el archivo de configuración. El resultado de esta operación es un nuevo DataFrame que contiene las mismas columnas de tiempo que el archivo original, pero reemplazando las columnas de áreas por columnas de concentraciones molares expresadas en moles por litro.

La conversión de triglicérido en un tiempo dado se calcula mediante la ecuación \ref{eq:conversion}, donde $C_{TG,0}$ representa la concentración inicial de triglicérido y $C_{TG,t}$ denota la concentración en el tiempo $t$. El método \texttt{calcular\_conversion()} agrega una columna adicional al DataFrame con estos valores de conversión expresados como fracción entre cero y uno, o multiplicados por cien para obtener porcentajes.

\begin{equation}
	X = \frac{C_{TG,0} - C_{TG,t}}{C_{TG,0}}
	\label{eq:conversion}
\end{equation}

Los datos procesados se guardan automáticamente en el directorio \texttt{datos/procesados/} en formato CSV para permitir su inspección manual y facilitar su uso por módulos subsiguientes del sistema. Adicionalmente, el módulo genera gráficas que visualizan la evolución temporal de todas las concentraciones en una sola figura con múltiples series superpuestas, utilizando colores diferenciados y marcadores distintivos para cada especie. Estas gráficas se guardan en formato PNG con alta resolución en el directorio \texttt{resultados/gc\_processing/} y se denominan con la fecha y hora de generación para evitar sobrescritura de resultados anteriores.

La práctica número cinco del material educativo desarrolla exhaustivamente el uso del módulo \texttt{GCProcessor} mediante un ejemplo completo que incluye un archivo CSV de ejemplo con datos sintéticos que simulan un experimento de transesterificación. Se proporcionan los factores de respuesta previamente determinados y se guía paso a paso en la configuración del archivo \texttt{config/config\_gc.json}, la ejecución del script \texttt{practicas/practica5\_gc\_processor/main.py}, y la interpretación de las gráficas generadas. Se incluyen además ejercicios donde se modifican intencionalmente algunos factores de respuesta para observar su impacto en las concentraciones calculadas, ilustrando la importancia de una calibración precisa del método cromatográfico.

\subsection{Ajuste de Parámetros Cinéticos mediante Regresión No Lineal}

El ajuste de parámetros cinéticos constituye el procedimiento mediante el cual se determinan numéricamente los valores óptimos del factor pre-exponencial $A$ y la energía de activación $E_a$ que minimizan las discrepancias entre las predicciones del modelo matemático y las concentraciones medidas experimentalmente. Este proceso se implementa en el módulo \texttt{src/optimization/parameter\_fitter.py} mediante la clase \texttt{ParameterFitter}, que ofrece dos estrategias algorítmicas alternativas para abordar el problema de optimización no lineal resultante.

El flujo de trabajo inicia cargando los datos experimentales procesados que residen en el directorio \texttt{datos/procesados/} en formato CSV. Estos datos provienen típicamente del módulo \texttt{GCProcessor} y contienen las concentraciones molares de todas las especies químicas en función del tiempo. El método \texttt{cargar\_datos\_experimentales()} de la clase \texttt{ParameterFitter} lee el archivo especificado y extrae las columnas correspondientes al tiempo y a las concentraciones que se utilizarán para el ajuste. El archivo de configuración \texttt{config/config\_ajuste.json} especifica cuáles especies se incluirán en la función objetivo, pudiendo optar por ajustar únicamente la concentración de triglicérido, o incluir simultáneamente las concentraciones de productos y especies intermedias para obtener un ajuste más robusto que considere toda la información experimental disponible.

La función objetivo que se minimiza durante el ajuste se define como la suma ponderada de residuos cuadrados entre las concentraciones experimentales y las predichas por el modelo, expresada matemáticamente en la ecuación \ref{eq:funcion_objetivo}. En esta ecuación, $n_{especies}$ representa el número de especies químicas incluidas en el ajuste, $n_{puntos,j}$ denota el número de puntos experimentales disponibles para la especie $j$, $C_{j,i}^{exp}$ es la concentración experimental de la especie $j$ en el tiempo $i$, $C_{j,i}^{modelo}(A, E_a)$ representa la concentración predicha por el modelo para los mismos valores de tiempo y especie química como función de los parámetros cinéticos, y $w_j$ constituye un factor de ponderación que permite asignar mayor importancia al ajuste de ciertas especies sobre otras.

\begin{equation}
	F_{obj}(A, E_a) = \sum_{j=1}^{n_{especies}} w_j \sum_{i=1}^{n_{puntos,j}} \left( C_{j,i}^{exp} - C_{j,i}^{modelo}(A, E_a) \right)^2
	\label{eq:funcion_objetivo}
\end{equation}

Para cada evaluación de la función objetivo durante el proceso iterativo de optimización, el algoritmo debe resolver numéricamente el sistema de ecuaciones diferenciales ordinarias con los valores actuales de $A$ y $E_a$ propuestos, obtener las concentraciones predichas en los tiempos correspondientes a las mediciones experimentales, calcular los residuos respecto a los valores medidos, y retornar la suma de cuadrados ponderada. Este procedimiento computacionalmente intensivo se ejecuta decenas o cientos de veces durante una optimización típica, lo cual motiva la selección cuidadosa del algoritmo de optimización y la configuración apropiada de tolerancias numéricas.

La primera estrategia de ajuste implementada utiliza el algoritmo de Levenberg-Marquardt, que representa un método híbrido entre el descenso de gradiente y el método de Gauss-Newton. Este algoritmo resulta particularmente eficiente para problemas de mínimos cuadrados no lineales cuando se dispone de estimaciones iniciales razonables de los parámetros. La implementación se realiza mediante la biblioteca \texttt{lmfit} de Python, que proporciona una interfaz de alto nivel para problemas de ajuste de parámetros. El método \texttt{ajustar\_levenberg\_marquardt()} de la clase \texttt{ParameterFitter} configura el problema de optimización definiendo los parámetros a ajustar mediante objetos \texttt{Parameter} de \texttt{lmfit}, donde se especifican valores iniciales, límites inferiores y superiores, y opcionalmente se pueden fijar ciertos parámetros si se desea mantenerlos constantes durante el ajuste.

Los límites de búsqueda para el factor pre-exponencial típicamente se establecen entre $10^8$ y $10^{12}$ L/(mol·min), rango que abarca los valores reportados en literatura para reacciones de transesterificación con catalizadores alcalinos. La energía de activación se restringe usualmente entre 30,000 y 80,000 J/mol, intervalo físicamente razonable para este tipo de transformaciones químicas. El archivo de configuración \texttt{config/config\_ajuste.json} permite al usuario especificar estos límites, así como los valores iniciales que sirven como punto de partida para la búsqueda. Una selección apropiada de valores iniciales, basada en conocimiento previo de la literatura o experimentos preliminares, puede reducir significativamente el tiempo de cómputo y aumentar la probabilidad de convergencia hacia el mínimo global.

El algoritmo de Levenberg-Marquardt ajusta automáticamente un parámetro de regularización denominado $\lambda$ que controla la transición entre comportamiento tipo descenso de gradiente y comportamiento tipo Gauss-Newton. Cuando el algoritmo se encuentra lejos del mínimo, $\lambda$ adopta valores elevados que inducen pasos conservadores similares al descenso de gradiente, garantizando progreso consistente hacia la reducción del error. Conforme el algoritmo se aproxima al mínimo, $\lambda$ disminuye progresivamente permitiendo pasos más agresivos característicos del método de Gauss-Newton, que exhibe convergencia cuadrática en las proximidades del óptimo. La biblioteca \texttt{lmfit} maneja internamente la actualización de $\lambda$ mediante heurísticas probadas, liberando al usuario de la necesidad de sintonizar manualmente este parámetro.

Tras la convergencia del algoritmo, que ocurre cuando la variación relativa en la función objetivo entre iteraciones consecutivas cae por debajo de una tolerancia especificada, \texttt{lmfit} calcula automáticamente los intervalos de confianza de los parámetros ajustados mediante la matriz de covarianza. Esta matriz se obtiene aproximando localmente la función objetivo como una función cuadrática y analizando la curvatura de esta aproximación. Los intervalos de confianza al 95% se calculan como $\pm 1.96$ veces la desviación estándar estimada de cada parámetro. Adicionalmente, el sistema calcula el coeficiente de determinación $R^2$ mediante la ecuación \ref{eq:r_cuadrado}, donde $SS_{res}$ representa la suma de cuadrados de residuos correspondiente al mejor ajuste obtenido, y $SS_{tot}$ denota la suma de cuadrados totales calculada respecto a la media de los valores experimentales.

\begin{equation}
	R^2 = 1 - \frac{SS_{res}}{SS_{tot}} = 1 - \frac{\sum_i (y_i^{exp} - y_i^{modelo})^2}{\sum_i (y_i^{exp} - \bar{y}^{exp})^2}
	\label{eq:r_cuadrado}
\end{equation}

La raíz del error cuadrático medio normalizado se calcula mediante la ecuación \ref{eq:rmse_normalizado}, dividiendo la raíz del error cuadrático medio entre el rango de valores experimentales observados para obtener una métrica porcentual que facilita la comparación de ajustes para diferentes conjuntos de datos.

\begin{equation}
	RMSE_{norm} = \frac{\sqrt{\frac{1}{n}\sum_i (y_i^{exp} - y_i^{modelo})^2}}{y_{max}^{exp} - y_{min}^{exp}} \times 100\%
	\label{eq:rmse_normalizado}
\end{equation}

La segunda estrategia algorítmica implementada utiliza el método de evolución diferencial, un algoritmo de optimización global estocástico que no requiere cálculo de gradientes y resulta más robusto ante la presencia de múltiples mínimos locales. Este método pertenece a la familia de algoritmos evolutivos y opera manteniendo una población de soluciones candidatas que evolucionan mediante operaciones de mutación, cruzamiento y selección. La implementación se realiza mediante la función \texttt{differential\_evolution} de la biblioteca \texttt{scipy.optimize}, invocada en el método \texttt{ajustar\_evolucion\_diferencial()} de la clase \texttt{ParameterFitter}.

El algoritmo inicia generando aleatoriamente una población de $N_p$ individuos, donde cada individuo representa un par de valores $(A, E_a)$ dentro de los límites de búsqueda especificados. Un tamaño de población típico es 15 veces el número de parámetros a optimizar, lo cual para este caso de 2 parámetros resulta en 30 individuos. En cada generación del algoritmo, se crea un individuo mutante para cada miembro de la población mediante la combinación lineal de tres individuos seleccionados aleatoriamente, según la ecuación \ref{eq:mutacion_diferencial}, donde $F$ representa el factor de escala de la mutación con valor típico de 0.8.

\begin{equation}
	\vec{v}_i = \vec{x}_{r1} + F(\vec{x}_{r2} - \vec{x}_{r3})
	\label{eq:mutacion_diferencial}
\end{equation}

Posteriormente se realiza una operación de cruzamiento entre el individuo mutante y el individuo objetivo, intercambiando componentes con probabilidad determinada por el parámetro de cruzamiento $CR$ que típicamente adopta valores entre 0.7 y 0.9. El individuo resultante del cruzamiento se evalúa calculando la función objetivo, y si presenta un valor inferior al del individuo objetivo original, lo reemplaza en la población para la siguiente generación. Este proceso de mutación, cruzamiento y selección se repite durante un número máximo de generaciones especificado en el archivo de configuración \texttt{config/config\_ajuste.json} mediante el parámetro \texttt{max\_generaciones}, siendo un valor típico de 200 generaciones. El algoritmo puede terminar anticipadamente si la variación de la función objetivo entre generaciones consecutivas cae por debajo de la tolerancia especificada mediante el parámetro \texttt{tol}, usualmente fijada en un valor de $10^{-7}$.

Una ventaja significativa del método de evolución diferencial radica en su capacidad para explorar el espacio de búsqueda de manera amplia, reduciendo el riesgo de quedar atrapado en mínimos locales subóptimos. Esta característica resulta particularmente valiosa cuando se dispone de datos experimentales con ruido considerable, o cuando no se posee información previa confiable para establecer valores iniciales apropiados. Sin embargo, el costo computacional de este método excede significativamente al del algoritmo de Levenberg-Marquardt debido a la necesidad de evaluar la función objetivo para todos los miembros de la población en cada generación. Para el caso típico de 30 individuos y 200 generaciones, se requieren 6,000 evaluaciones de la función objetivo, cada una de las cuales implica resolver numéricamente el sistema de ecuaciones diferenciales ordinarias.

Los resultados del ajuste de parámetros se almacenan automáticamente en el archivo \texttt{config/parametros\_ajustados.json}, que contiene los valores óptimos de $A$ y $E_a$, sus respectivos intervalos de confianza, las métricas de bondad de ajuste $R^2$ y RMSE, y metadatos que documentan la fecha y hora del ajuste, el algoritmo utilizado, y las opciones de configuración empleadas. Este archivo se utiliza posteriormente por el módulo de optimización operacional para realizar predicciones con los parámetros validados. El módulo de visualización genera automáticamente una gráfica que superpone los datos experimentales representados mediante puntos con marcadores, y las curvas del modelo ajustado representadas mediante líneas continuas, para cada especie química incluida en el ajuste. Esta gráfica facilita la evaluación visual de la calidad del ajuste y la identificación de posibles desviaciones sistemáticas que puedan indicar limitaciones del modelo cinético seleccionado.

La práctica número seis del material educativo guía exhaustivamente en el proceso de ajuste de parámetros cinéticos utilizando ambas estrategias algorítmicas. Se proporcionan datos experimentales sintéticos con diferentes niveles de ruido para ilustrar el comportamiento de cada algoritmo. Se comparan los tiempos de cómputo, la robustez ante ruido experimental, y la sensibilidad a los valores iniciales. Se incluyen ejercicios donde se varía intencionalmente la calidad de los valores iniciales para demostrar que el método de Levenberg-Marquardt puede fallar en encontrar el óptimo global cuando parte de estimaciones iniciales pobres, mientras que el método de evolución diferencial converge consistentemente al mismo óptimo independientemente de las condiciones iniciales, aunque requiriendo mayor tiempo computacional.


\subsection{Optimización de Variables Operacionales mediante Algoritmos Heurísticos}

La optimización de las condiciones operacionales del proceso de transesterificación constituye una etapa fundamental para maximizar la conversión de triglicéridos a biodiesel minimizando simultáneamente el tiempo de reacción y el consumo de reactivos. Este problema de optimización multiobjetivo se aborda mediante el módulo \texttt{src/optimization/operational\_optimizer.py}, que implementa la clase \texttt{OperationalOptimizer}. Esta clase proporciona métodos para determinar numéricamente los valores óptimos de cuatro variables operacionales críticas que gobiernan el desempeño del reactor: la temperatura de operación, la relación molar entre metanol y triglicérido, la concentración másica de catalizador, y el perfil temporal de intensidad de agitación.

El flujo de trabajo para la optimización operacional inicia cargando los parámetros cinéticos previamente ajustados que residen en el archivo \texttt{config/parametros\_ajustados.json}. Estos parámetros, que incluyen el factor pre-exponencial $A$ y la energía de activación $E_a$ determinados mediante el procedimiento de regresión no lineal descrito en la sección anterior, permiten al modelo cinético realizar predicciones cuantitativas confiables del comportamiento del reactor bajo diferentes condiciones operacionales. El método \texttt{cargar\_parametros\_cineticos()} de la clase \texttt{OperationalOptimizer} lee este archivo y almacena internamente los valores de los parámetros que se utilizarán en todas las simulaciones subsecuentes del proceso de optimización.

La definición del problema de optimización requiere especificar la función objetivo que se maximizará o minimizará durante la búsqueda. El sistema permite al usuario seleccionar entre múltiples formulaciones alternativas de función objetivo mediante el parámetro \texttt{tipo\_objetivo} en el archivo de configuración \texttt{config/config\_optimizacion.json}. La primera opción consiste en maximizar la conversión final de triglicérido alcanzada tras un tiempo de reacción fijo especificado por el usuario, lo cual resulta apropiado cuando se dispone de restricciones de tiempo en el proceso industrial. La segunda opción invierte el enfoque minimizando el tiempo necesario para alcanzar una conversión objetivo predefinida, estrategia preferible cuando el criterio dominante es la productividad del reactor expresada en kilogramos de biodiesel por hora. La tercera opción, más sofisticada, minimiza una función de costo económico que pondera el valor de los productos generados contra el costo de los reactivos consumidos y el costo operacional asociado al tiempo de procesamiento, permitiendo identificar el punto de operación económicamente óptimo.

Para el caso más común de maximización de conversión final a tiempo fijo, la función objetivo se expresa matemáticamente mediante la ecuación \ref{eq:obj_conversion}, donde $X_{TG}(t_f)$ representa la conversión fraccional de triglicérido en el tiempo final $t_f$, calculada como la diferencia entre la concentración inicial y la concentración final normalizada respecto a la concentración inicial.

\begin{equation}
	F_{obj}(\vec{x}) = X_{TG}(t_f; \vec{x}) = \frac{C_{TG,0} - C_{TG}(t_f; \vec{x})}{C_{TG,0}}
	\label{eq:obj_conversion}
\end{equation}

En esta ecuación, el vector $\vec{x}$ contiene las cuatro variables de decisión que se optimizan: temperatura $T$ expresada en grados Celsius, relación molar metanol a triglicérido $r_{MeOH:TG}$ expresada como número adimensional, concentración de catalizador $C_{cat}$ expresada como porcentaje másico respecto al aceite, y parámetros que definen el perfil de agitación que puede variar temporalmente durante la reacción. La evaluación de la función objetivo para un conjunto dado de valores de estas variables requiere resolver numéricamente el sistema completo de ecuaciones diferenciales ordinarias con las condiciones operacionales especificadas, extraer la concentración de triglicérido en el tiempo final, y calcular la conversión mediante la fórmula indicada.

Los rangos de búsqueda para las variables de decisión se establecen mediante límites inferiores y superiores físicamente realistas y operacionalmente viables especificados en el archivo de configuración. La temperatura típicamente se restringe entre 50 y 65 °C, rango que evita la ebullición del metanol a presión atmosférica que ocurre a 65 °C, mientras proporciona suficiente energía térmica para que la reacción proceda a velocidad apreciable. Temperaturas inferiores a 50 °C resultan en cinéticas excesivamente lentas que requieren tiempos de reacción impracticables para aplicaciones industriales. La relación molar se limita usualmente entre 3:1 y 12:1, donde el límite inferior corresponde a la proporción estequiométrica que representa el mínimo teórico, y el límite superior refleja restricciones prácticas asociadas al volumen del reactor y la dificultad de separación del exceso de metanol. La concentración de catalizador se acota típicamente entre 0.5 y 2% másico, intervalo que balancea la aceleración de la cinética contra la complejidad y costo de las etapas de purificación posteriores que aumentan con concentraciones elevadas de catalizador.

El perfil de agitación se parametriza mediante velocidades angulares del impulsor en revoluciones por minuto que pueden variar en el tiempo siguiendo funciones escalonadas, lineales, o constantes. El archivo de configuración permite definir perfiles multietapa donde la agitación inicia en un valor elevado durante los primeros minutos para promover la emulsificación de las fases inmiscibles de aceite y metanol, posteriormente se reduce a un valor intermedio durante la etapa de reacción principal para minimizar consumo energético, y finalmente puede aumentarse nuevamente en la etapa final para favorecer la conversión completa de las últimas trazas de triglicérido. El método \texttt{interpretar\_perfil\_agitacion()} de la clase \texttt{OperationalOptimizer} convierte la representación JSON del perfil en una función evaluable numéricamente que retorna la velocidad de agitación para cualquier tiempo dado, función que se utiliza internamente por el integrador de ecuaciones diferenciales para calcular la influencia de la agitación sobre los coeficientes de transferencia de masa.

La resolución del problema de optimización se realiza mediante el algoritmo de minimización de Powell modificado implementado en la función \texttt{minimize} de la biblioteca \texttt{scipy.optimize}, invocado desde el método \texttt{optimizar\_condiciones\_powell()} de la clase \texttt{OperationalOptimizer}. Este algoritmo pertenece a la categoría de métodos de búsqueda directa que no requieren cálculo de gradientes, característica valiosa dado que las derivadas de la función objetivo respecto a las variables operacionales no están disponibles analíticamente debido a la necesidad de integración numérica del sistema de ecuaciones diferenciales. El método de Powell opera realizando búsquedas unidimensionales secuenciales a lo largo de direcciones conjugadas que se actualizan adaptativamente durante el proceso de optimización, logrando convergencia eficiente típicamente en varias decenas de evaluaciones de función.

Alternativamente, cuando se requiere mayor robustez ante la presencia de múltiples óptimos locales, el sistema ofrece la opción de utilizar el algoritmo de recocido simulado implementado mediante la función \texttt{dual\_annealing} de \texttt{scipy.optimize}, accesible a través del método \texttt{optimizar\_condiciones\_annealing()}. Este algoritmo estocástico global opera mediante un proceso de búsqueda que acepta ocasionalmente movimientos que aumentan temporalmente el valor de la función objetivo, análogamente al proceso físico de recocido en metalurgia donde un material se calienta y posteriormente se enfría gradualmente para alcanzar configuraciones de menor energía. La probabilidad de aceptación de movimientos ascendentes disminuye progresivamente según un cronograma de enfriamiento controlado por el parámetro de temperatura virtual, que inicia en un valor elevado permitiendo exploración amplia del espacio de búsqueda, y decae exponencialmente favoreciendo progresivamente la explotación de regiones prometedoras identificadas.

El método de recocido simulado requiere especificar un presupuesto computacional mediante el parámetro \texttt{maxiter} que determina el número máximo de iteraciones permitidas antes de la terminación forzada del algoritmo. Un valor típico de 1,000 iteraciones proporciona un balance razonable entre calidad de la solución y tiempo de cómputo para este problema con 4 variables de decisión. El archivo de configuración \texttt{config/config\_optimizacion.json} permite además especificar el número de reinicios locales mediante el parámetro \texttt{local\_search\_options}, donde cada reinicio consiste en aplicar el método de Powell desde la mejor solución encontrada hasta el momento para refinar localmente el resultado del recocido global. Un número típico de 3 reinicios locales mejora significativamente la precisión de la solución final con incremento modesto del costo computacional.

Los resultados de la optimización se almacenan automáticamente en el archivo \texttt{resultados/optimizacion/condiciones\_optimas.json}, que documenta los valores óptimos determinados para cada variable operacional, el valor de la función objetivo correspondiente, el número de evaluaciones de función requeridas hasta convergencia, y el tiempo total de cómputo expresado en segundos. Adicionalmente se genera automáticamente un conjunto de gráficas de análisis de sensibilidad que ilustran cómo varía la función objetivo cuando se modifica individualmente cada variable operacional mientras se mantienen las demás en sus valores óptimos, permitiendo identificar cuáles parámetros ejercen mayor influencia sobre el desempeño del reactor y merecen control más estricto durante la operación industrial.

La práctica número siete del material educativo implementa un ejercicio completo de optimización operacional utilizando parámetros cinéticos previamente ajustados con datos de la literatura. Se comparan los resultados obtenidos mediante los métodos de Powell y recocido simulado, ilustrando las diferencias en tiempo de cómputo y calidad de solución. Se realizan estudios paramétricos variando los límites de búsqueda y las opciones algorítmicas para desarrollar intuición sobre la sensibilidad del problema de optimización. Se plantean escenarios donde se modifica la función objetivo de maximización de conversión a minimización de costo económico, requiriendo que el practicante especifique precios de reactivos y productos basados en cotizaciones industriales actuales.


\subsection{Escalado de Reactor Batch a Escala Piloto mediante Criterios de Similitud}

El escalado del proceso de transesterificación desde el reactor de laboratorio de 350 mL hasta un reactor piloto de 20 L constituye una tarea compleja que requiere preservar las condiciones hidrodinámicas y de transferencia de masa que garantizan desempeño comparable entre ambas escalas. El módulo \texttt{src/scale\_up/scale\_up\_calculator.py} implementa la clase \texttt{ScaleUpCalculator} que automatiza los cálculos de escalado mediante criterios de similitud dimensional derivados de la mecánica de fluidos y la teoría de transferencia de masa.

El criterio fundamental para el escalado geométrico consiste en mantener constante la relación entre el diámetro del impulsor $D$ y el diámetro del tanque $T$, así como preservar la relación entre la altura del líquido $H$ y el diámetro del tanque. Para el reactor de laboratorio equipado con un impulsor de cinta de 6 cm de diámetro operando en un tanque cilíndrico de 8 cm de diámetro, la relación $D/T$ adopta el valor de 0.75. El escalado geométrico estricto requiere que el reactor piloto mantenga esta misma relación, lo cual para un tanque de 20 cm de diámetro resulta en un impulsor de 15 cm. Sin embargo, consideraciones prácticas de fabricación y disponibilidad de componentes comerciales pueden justificar desviaciones moderadas de esta relación ideal, cuyo impacto sobre el desempeño puede cuantificarse mediante el modelo implementado.

El criterio hidrodinámico más empleado para escalado de reactores agitados consiste en mantener constante el número de potencia $N_P$, parámetro adimensional que relaciona la potencia disipada por el impulsor con las propiedades del fluido y las dimensiones del sistema según la ecuación \ref{eq:numero_potencia}. En esta expresión, $P$ representa la potencia disipada expresada en vatios, $\rho$ denota la densidad del fluido en kilogramos por metro cúbico, $N$ es la velocidad de rotación del impulsor en revoluciones por segundo, y $D$ el diámetro del impulsor en metros.

\begin{equation}
	N_P = \frac{P}{\rho N^3 D^5}
	\label{eq:numero_potencia}
\end{equation}

Para el sistema de transesterificación de aceite de cocina usado con metanol, la densidad de la mezcla reaccional varía desde aproximadamente 900 kg/m³ al inicio cuando predomina el aceite, hasta aproximadamente 880 kg/m³ al final cuando se ha formado el biodiesel de menor densidad. El método \texttt{calcular\_densidad\_mezcla()} de la clase \texttt{ScaleUpCalculator} determina la densidad instantánea de la mezcla mediante una regla de mezclado volumétrica que pondera las densidades puras de cada componente según sus fracciones molares calculadas a partir del grado de conversión.

Un criterio alternativo de escalado, particularmente relevante para sistemas multifásicos como la transesterificación donde inicialmente coexisten fases inmiscibles de aceite y metanol, consiste en mantener constante la velocidad específica de disipación de energía $\epsilon$ expresada en vatios por kilogramo según la ecuación \ref{eq:disipacion_especifica}. Este criterio garantiza intensidad comparable de turbulencia y capacidad de emulsificación entre escalas diferentes.

\begin{equation}
	\epsilon = \frac{P}{m_{fluido}} = \frac{P}{\rho V_{liquido}}
	\label{eq:disipacion_especifica}
\end{equation}

El número de Reynolds $Re$ definido mediante la ecuación \ref{eq:reynolds} caracteriza el régimen de flujo en el reactor, permitiendo verificar que ambas escalas operan en régimen turbulento donde las correlaciones de transferencia de masa son válidas. En esta expresión, $\mu$ representa la viscosidad dinámica del fluido en pascales-segundo.

\begin{equation}
	Re = \frac{\rho N D^2}{\mu}
	\label{eq:reynolds}
\end{equation}

Para el sistema de transesterificación, la viscosidad de la mezcla reaccional evoluciona significativamente durante el transcurso de la reacción, iniciando en valores elevados de aproximadamente 50 mPa·s cuando predomina el aceite usado, y disminuyendo hasta aproximadamente 4 mPa·s cuando se ha formado biodiesel y glicerol. El método \texttt{calcular\_viscosidad\_mezcla()} implementa la ecuación de Grunberg-Nissan que predice la viscosidad de mezclas líquidas como función exponencial de las fracciones molares, ajustando parámetros de interacción binaria obtenidos de la base de datos DIPPR para los pares de componentes relevantes.

El escalado de la velocidad de agitación entre las dos escalas se determina mediante el método \texttt{calcular\_velocidad\_escalada()}, que resuelve la ecuación del criterio de escalado seleccionado para obtener la velocidad de rotación $N_{piloto}$ requerida en la escala mayor para preservar el parámetro adimensional objetivo. Por ejemplo, si se selecciona el criterio de potencia específica constante y el reactor de laboratorio opera a 400 rpm disipando 0.5 W, y se escala a un reactor 20 veces mayor en volumen preservando similitud geométrica, la aplicación del criterio resulta en una velocidad del reactor piloto significativamente reducida del orden de 200 rpm, valor que se calcula automáticamente por el método mencionado.

El módulo de escalado genera automáticamente un reporte comparativo que tabula los parámetros operacionales de ambas escalas, incluyendo volumen total de reacción, dimensiones del impulsor y del tanque, velocidad de rotación, potencia disipada, potencia específica, número de Reynolds, tiempo de mezclado estimado mediante la correlación de Ruszkowski, y el coeficiente volumétrico de transferencia de masa líquido-líquido estimado mediante la correlación de Skelland-Ramsay. Este reporte facilita la evaluación crítica de la viabilidad del escalado y la identificación de potenciales diferencias de desempeño que puedan requerir ajustes experimentales para compensar efectos de escala no capturados completamente por los criterios de similitud teóricos.

La práctica número ocho del material educativo guía en el proceso completo de escalado desde el reactor de laboratorio hasta la escala piloto de 20 L. Se calculan manualmente los parámetros de escalado utilizando diferentes criterios de similitud, se comparan los resultados obtenidos con cada criterio, y se ejecutan simulaciones del modelo cinético utilizando las condiciones operacionales escaladas para predecir el comportamiento del reactor piloto. Se discuten las limitaciones de los diferentes criterios de escalado y las situaciones donde cada uno resulta más apropiado, desarrollando criterio ingenieril para la selección del enfoque de escalado en función de las características específicas del sistema reaccional.


\section{Estrategia Educativa: Prácticas Progresivas para Modelado de Biodiesel}

El componente educativo del sistema desarrollado consiste en una secuencia de doce prácticas cuidadosamente estructuradas que guían progresivamente desde conceptos fundamentales hasta aplicaciones avanzadas de modelado cinético y optimización de procesos de transesterificación. Esta estrategia pedagógica reconoce que el aprendizaje efectivo de modelado matemático de reactores químicos requiere construcción gradual de habilidades, comenzando con cálculos estequiométricos elementales y avanzando sistemáticamente hacia análisis paramétricos sofisticados y comparaciones con literatura experimental publicada. Cada práctica se ha diseñado para ocupar entre dos y cuatro horas de trabajo efectivo, permitiendo que un curso semestral típico de dieciséis semanas abarque todo el material con tiempo suficiente para discusión en clase y realización de evaluaciones.

Las prácticas se organizan en el directorio \texttt{practicas/} del repositorio, donde cada una reside en su propia carpeta denominada \texttt{practica\_N\_descripcion/} con $N$ variando de uno a doce. Cada carpeta contiene un archivo de enunciado en formato Markdown denominado \texttt{README.md} que presenta los objetivos de aprendizaje, el marco teórico necesario, las instrucciones paso a paso, y las preguntas de análisis que el practicante debe responder al completar la actividad. Adicionalmente cada práctica incluye un archivo de análisis denominado \texttt{analisis.md} que proporciona plantillas estructuradas para documentar resultados, responder preguntas conceptuales, y reflexionar sobre los aprendizajes obtenidos. Los datos de entrada requeridos, cuando aplicable, se suministran en archivos CSV o JSON dentro de una subcarpeta \texttt{datos/}, y las soluciones de referencia se proporcionan en archivos Python completamente funcionales que el instructor puede utilizar para verificación o para generar expectativas de resultados.

La secuencia didáctica inicia con la práctica número uno denominada Introducción a Python y Cálculos Estequiométricos Básicos, que familiariza al practicante con el entorno de programación científica en Python utilizando NumPy para arreglos numéricos y pandas para manipulación de datos tabulares. Esta práctica evita deliberadamente cualquier complejidad de integración numérica o cinética química, enfocándose exclusivamente en cálculos molares y másicos elementales. El practicante debe calcular las cantidades de reactivos necesarias para experimentos de transesterificación bajo diferentes relaciones molares metanol a triglicérido, determinar las masas teóricas de biodiesel producible asumiendo conversión completa, y evaluar si los volúmenes resultantes de mezcla reaccional exceden la capacidad del reactor disponible. Este ejercicio aparentemente simple establece familiaridad con las propiedades físicas de los compuestos involucrados, refuerza conceptos de estequiometría que resultan esenciales para interpretar resultados cinéticos posteriores, y desarrolla habilidad para manipular datos en formato tabular que se utilizará extensivamente en prácticas subsecuentes.

La práctica número dos introduce el concepto de cinética química mediante el análisis de datos de conversión en función del tiempo para una reacción de transesterificación simplificada. El practicante recibe un archivo CSV conteniendo mediciones sintéticas de concentración de triglicérido a diferentes tiempos, y debe graficar la evolución temporal, calcular la conversión fraccional en cada punto, y ajustar manualmente una función exponencial decreciente que representa el comportamiento de reacciones de pseudo-primer orden. Esta práctica deliberadamente evita el uso de herramientas automáticas de ajuste de curvas, requiriendo que el practicante experimente con diferentes valores de constante cinética aparente hasta lograr concordancia visual satisfactoria entre datos y modelo. Este enfoque pedagógico desarrolla intuición sobre el significado físico de la constante cinética y la sensibilidad del modelo a variaciones paramétricas, preparando conceptualmente para el uso posterior de algoritmos formales de optimización no lineal.

La tercera práctica aborda la dependencia de la constante cinética con la temperatura mediante la ecuación de Arrhenius. Se proporcionan datos sintéticos de constante cinética aparente medida a cinco temperaturas diferentes en el rango de cincuenta a sesenta y cinco grados Celsius. El practicante debe transformar estos datos mediante logaritmo natural y graficar versus el inverso de la temperatura absoluta, observando la relación lineal predicha por la ecuación de Arrhenius que se expresa como $\ln k = \ln A - E_a/(RT)$. La pendiente de la regresión lineal permite determinar la energía de activación, mientras que el intercepto proporciona el factor pre-exponencial. Esta práctica refuerza conceptos termodinámicos fundamentales sobre barreras energéticas de reacción, introduce el análisis de regresión lineal como herramienta de determinación paramétrica, y establece la conexión cuantitativa entre condiciones operacionales y velocidad de reacción que resulta central para la optimización de procesos.

La cuarta práctica representa el primer encuentro del practicante con la integración numérica de ecuaciones diferenciales ordinarias, habilidad matemática fundamental para modelado de reactores químicos. Se presenta un sistema simplificado de dos ecuaciones diferenciales ordinarias acopladas que describen la desaparición de triglicérido y la formación de biodiesel según cinética de pseudo-primer orden, asumiendo exceso grande de metanol que permite tratar su concentración como constante. El practicante debe implementar la solución numérica utilizando la función \texttt{odeint} de la biblioteca \texttt{scipy.integrate}, especificando las condiciones iniciales apropiadas y el vector de tiempos donde se desea la solución. Se generan gráficas de concentración versus tiempo para ambas especies, y se comparan cualitativamente con el comportamiento esperado de reacciones irreversibles consecutivas. Esta práctica establece el flujo de trabajo computacional estándar para resolución de problemas cinéticos que se repetirá con complejidad creciente en prácticas subsecuentes: definir función que calcula derivadas, especificar condiciones iniciales, invocar integrador, procesar resultados, visualizar.

La quinta práctica introduce el procesamiento de datos experimentales de cromatografía de gases, técnica analítica estándar para cuantificación de composiciones en sistemas de transesterificación. Se proporcionan archivos CSV sintéticos que simulan la salida cruda de un cromatógrafo de gases con detector de ionización de llama, conteniendo tiempos de retención y áreas de pico para triglicérido, diglicérido, monoglicérido, biodiesel, y estándar interno de metil heptadecanoato. El practicante debe implementar el cálculo de factores de respuesta relativos, convertir áreas de pico a concentraciones molares utilizando la ecuación presentada en la sección de metodología, y generar curvas de concentración versus tiempo para todas las especies detectadas. Esta práctica desarrolla habilidades de preprocesamiento de datos experimentales que resultan esenciales para trabajar con información del mundo real que inevitablemente contiene ruido, valores atípicos, y requiere normalización apropiada antes de su uso en ajuste de modelos.

La sexta práctica aborda el ajuste de parámetros cinéticos mediante regresión no lineal, representando uno de los ejercicios más desafiantes y pedagógicamente valiosos de la secuencia. Se proporcionan datos de concentración versus tiempo para un experimento de transesterificación completo, y el practicante debe utilizar la clase \texttt{ParameterFitter} para determinar los valores óptimos del factor pre-exponencial y la energía de activación que minimizan las discrepancias entre el modelo cinético y las mediciones. La práctica guía en la configuración del archivo \texttt{config/config\_ajuste.json}, la especificación de límites de búsqueda apropiados, la selección de valores iniciales razonables basados en literatura, y la interpretación crítica de las métricas de bondad de ajuste obtenidas. Se comparan explícitamente los resultados de los algoritmos de Levenberg-Marquardt y evolución diferencial, ilustrando sus respectivas fortalezas y debilidades. Se incluyen ejercicios de análisis de sensibilidad donde se degrada intencionalmente la calidad de los valores iniciales para demostrar la dependencia del método de Levenberg-Marquardt respecto a la estimación inicial, contrastando con la robustez del método evolutivo. Esta práctica consolida el entendimiento de que el ajuste de parámetros constituye un problema de optimización no trivial cuya solución requiere selección cuidadosa de algoritmos y configuraciones.

La séptima práctica aplica las herramientas de optimización operacional para determinar las condiciones de temperatura, relación molar, y concentración de catalizador que maximizan la conversión de triglicérido en un tiempo de reacción fijo de sesenta minutos. El practicante utiliza parámetros cinéticos previamente ajustados en la práctica seis, configura el archivo \texttt{config/config\_optimizacion.json} especificando la función objetivo y los límites de búsqueda, y ejecuta tanto el método de Powell como el recocido simulado para resolver el problema de optimización multiparamétrica. Se generan automáticamente gráficas de análisis de sensibilidad unidimensional que muestran cómo varía la conversión final cuando se modifica cada variable operacional individualmente manteniendo las demás constantes, permitiendo identificar cuáles parámetros ejercen mayor influencia y requieren control operacional más estricto. La práctica incluye un ejercicio de diseño económico donde el practicante debe reformular la función objetivo para minimizar el costo de producción por kilogramo de biodiesel, requiriendo investigación de precios actuales de metanol, catalizador, y aceite usado, introduciendo así consideraciones económicas realistas al problema de optimización técnica.

La octava práctica aborda el escalado de procesos mediante el cálculo de parámetros operacionales para un reactor piloto de veinte litros que preserve el desempeño observado en el reactor de laboratorio de trescientos cincuenta mililitros. El practicante utiliza la clase \texttt{ScaleUpCalculator} para aplicar diferentes criterios de similitud: número de potencia constante, disipación específica de energía constante, y velocidad de punta de impulsor constante. Se comparan las velocidades de agitación resultantes de cada criterio, se calculan los números de Reynolds para verificar que ambas escalas operan en régimen turbulento, y se estiman los coeficientes volumétricos de transferencia de masa mediante correlaciones empíricas. Posteriormente se ejecutan simulaciones del modelo cinético utilizando las condiciones operacionales escaladas para predecir si la conversión en el reactor piloto igualará la del reactor de laboratorio, identificando potenciales limitaciones de los criterios de similitud puramente dimensionales. Esta práctica desarrolla habilidades de análisis dimensional y pensamiento crítico sobre las asunciones implícitas en diferentes enfoques de escalado, preparando para la toma de decisiones ingenieriles en contextos industriales donde el escalado representa un desafío técnico mayor.

La novena práctica introduce variabilidad estocástica mediante simulaciones Monte Carlo que permiten cuantificar la propagación de incertidumbres experimentales hacia las predicciones del modelo. El practicante debe generar cien conjuntos sintéticos de datos experimentales añadiendo ruido gaussiano aleatorio con desviación estándar del cinco por ciento a un perfil de concentración versus tiempo de referencia. Para cada conjunto de datos ruidosos se ajustan independientemente los parámetros cinéticos, obteniéndose cien pares de valores de factor pre-exponencial y energía de activación. El análisis estadístico de estas distribuciones paramétricas permite calcular las desviaciones estándar, construir histogramas que revelan la forma de las distribuciones, e identificar correlaciones entre parámetros mediante gráficas de dispersión bidimensionales. Esta práctica introduce conceptos fundamentales de análisis de incertidumbre que resultan esenciales para interpretación rigurosa de resultados experimentales, donde la cuantificación de intervalos de confianza distingue el modelado científico robusto de ajustes de curvas superficiales.

La décima práctica representa la culminación de la secuencia básica mediante la validación exhaustiva del modelo cinético completo contra datos experimentales publicados en la literatura científica. Se utiliza específicamente el trabajo de Kouzu et al. del año dos mil ocho publicado en la revista Fuel, que reporta conversiones de transesterificación de aceite de soja con metanol y catalizador de óxido de calcio a diferentes temperaturas y relaciones molares. El practicante debe extraer datos numéricos de las gráficas del artículo original utilizando herramientas de digitalización, formatear los datos en archivos CSV apropiados, ajustar los parámetros cinéticos del modelo desarrollado utilizando un subconjunto de los datos experimentales denominado conjunto de entrenamiento, y posteriormente validar las predicciones del modelo contra el subconjunto restante denominado conjunto de validación que no se utilizó durante el ajuste. Se calculan métricas cuantitativas de desempeño del modelo incluyendo el coeficiente de determinación $R^2$, el error absoluto medio, y el error relativo promedio, y se genera un gráfico de paridad que superpone valores predichos versus valores experimentales para visualizar la calidad de concordancia. Esta práctica desarrolla habilidades críticas de evaluación de modelos mediante validación cruzada, y establece expectativas realistas sobre la precisión alcanzable cuando se modelan sistemas químicos complejos del mundo real que exhiben variabilidad experimental inevitable.

La undécima práctica extiende el análisis mediante un estudio de sensibilidad global que cuantifica la influencia relativa de cuatro parámetros operacionales sobre la conversión final de triglicérido: temperatura de reacción, relación molar metanol a triglicérido, concentración de catalizador, e intensidad de agitación. El practicante implementa un diseño experimental factorial fraccional donde cada parámetro se varía en tres niveles: mínimo, nominal, y máximo. Se ejecutan simulaciones para todas las combinaciones resultantes del diseño experimental, generándose varias decenas de puntos en el espacio paramétrico. El análisis de varianza ANOVA aplicado a los resultados permite descomponer la variabilidad total de la conversión en contribuciones atribuibles a cada factor individual y a interacciones de segundo orden entre factores. Se construyen gráficos de efectos principales y gráficos de interacción que visualizan cómo el efecto de un parámetro depende del valor de otros parámetros, revelando sinergias o antagonismos entre variables operacionales. Esta práctica introduce metodología formal de diseño de experimentos que maximiza la información obtenida de un número limitado de simulaciones o experimentos, habilidad transferible a contextos industriales donde la experimentación física resulta costosa y el tiempo es limitado.

La duodécima y última práctica plantea un estudio de caso integrador donde el practicante debe diseñar completamente un proceso de producción de biodiesel a partir de aceite de cocina usado para satisfacer especificaciones dadas de producción anual, pureza del producto, y costo máximo de manufactura. Esta práctica de síntesis requiere aplicar todas las herramientas desarrolladas en las once prácticas previas: calcular balances de masa para determinar consumos de reactivos, diseñar experimentos cinéticos para caracterizar el aceite específico disponible, ajustar parámetros del modelo, optimizar condiciones operacionales, escalar el proceso a escala industrial, cuantificar incertidumbres mediante análisis de sensibilidad, y finalmente evaluar la viabilidad económica mediante análisis de costo. Se solicita al practicante generar un reporte técnico completo documentando el diseño propuesto, justificando todas las decisiones tomadas mediante evidencia cuantitativa, identificando riesgos y limitaciones del diseño, y proponiendo experimentos adicionales que reducirían las incertidumbres principales. Esta práctica capstone integra conocimientos técnicos con habilidades de comunicación profesional, preparando para el desempeño en entornos industriales donde la capacidad de documentar, justificar, y comunicar decisiones técnicas resulta tan valiosa como la competencia analítica subyacente.


\section{Validación del Modelo Cinético contra Datos Experimentales de Literatura}

La validación rigurosa del modelo cinético desarrollado constituye un requisito fundamental para establecer su credibilidad científica y su utilidad como herramienta predictiva confiable. La validación trasciende el simple ajuste de parámetros a datos experimentales propios, requiriendo demostrar que el modelo reproduce adecuadamente observaciones independientes reportadas por otros investigadores utilizando materias primas, catalizadores, y condiciones operacionales potencialmente diferentes. Este ejercicio de validación externa no solo verifica la capacidad predictiva del modelo sino que también identifica sus limitaciones y el rango de aplicabilidad donde las predicciones mantienen precisión aceptable.

El conjunto de datos experimentales seleccionado para la validación proviene del trabajo de Kouzu et al. publicado en el año 2008 en la revista Fuel bajo el título Solid base catalysis of calcium oxide for a reaction to convert vegetable oil into biodiesel. Este estudio investigó sistemáticamente la transesterificación de aceite de soja refinado con metanol utilizando óxido de calcio como catalizador heterogéneo básico, reportando conversiones de triglicérido en función del tiempo para múltiples temperaturas de reacción en el rango de 45 a 65 °C. La selección de este trabajo como referencia de validación se justifica por varias razones: primero, reporta datos cinéticos completos incluyendo perfiles temporales de conversión en lugar de únicamente conversiones finales, permitiendo ajuste robusto de parámetros cinéticos; segundo, documenta exhaustivamente las condiciones experimentales incluyendo relación molar, concentración de catalizador, intensidad de agitación, y pureza de reactivos, permitiendo replicar las condiciones en las simulaciones del modelo; tercero, el catalizador heterogéneo de óxido de calcio exhibe comportamiento cinético comparable al del hidróxido de potasio utilizado en el presente desarrollo, ambos siendo bases fuertes que catalizan el mecanismo de transesterificación mediante el mismo intermediario alcóxido.

El procedimiento de validación inicia con la extracción de datos numéricos de las gráficas publicadas en el artículo de Kouzu et al. mediante el software de digitalización WebPlotDigitizer de código abierto. Este programa permite seleccionar manualmente los puntos de datos en una imagen de gráfica calibrando previamente los ejes, y exportar las coordenadas numéricas resultantes en formato de tabla. La conversión de triglicérido reportada visualmente en la Figura 3 del artículo original se digitaliza para cinco temperaturas: 45, 50, 55, 60 y 65 °C, obteniéndose aproximadamente 8 a 10 puntos experimentales por curva de temperatura que cubren el rango temporal de 0 a 120 min. Los datos digitalizados se almacenan en el archivo \texttt{datos/literatura/kouzu\_2008\_digitalizados.csv} en formato de tabla con columnas para tiempo expresado en minutos, conversión fraccional expresada como número entre 0 y 1, y temperatura expresada en grados Celsius.

Antes de proceder al ajuste de parámetros, resulta necesario verificar que las condiciones experimentales reportadas por Kouzu et al. corresponden a aquellas asumidas por el modelo desarrollado. El artículo reporta una relación molar metanol a aceite de 6:1, concentración de catalizador de 3% másico respecto al aceite, velocidad de agitación de 600 rpm, y presión atmosférica. Estas condiciones se especifican en el archivo de configuración \texttt{config/config\_validacion\_kouzu.json} que el módulo de validación carga antes de ejecutar las simulaciones. La masa molar promedio del aceite de soja se calcula como 872 g/mol asumiendo composición típica de triglicéridos de soja dominada por ácido linoleico y oleico según datos de la base PubChem. La densidad del aceite de soja a 60 °C se estima como 905 kg/m³ mediante la correlación de Rackett utilizando parámetros del Perry's Chemical Engineers' Handbook.

El conjunto completo de datos digitalizados se divide aleatoriamente en dos subconjuntos: 70% de los puntos constituyen el conjunto de entrenamiento utilizado para el ajuste de parámetros cinéticos, mientras que el 30% restante forma el conjunto de validación que permanece reservado para evaluación independiente del desempeño predictivo. Esta partición entrenamiento-validación se implementa mediante el método \texttt{dividir\_datos\_entrenamiento\_validacion()} de la clase \texttt{ModelValidator} ubicada en el módulo \texttt{src/validation/model\_validator.py}. La división se estratifica por temperatura garantizando que ambos subconjuntos contengan representación de todas las temperaturas experimentales, evitando el sesgo que resultaría si el conjunto de validación contuviera únicamente datos de temperaturas no representadas en el entrenamiento.

El ajuste de los parámetros cinéticos factor pre-exponencial $A$ y energía de activación $E_a$ se realiza mediante el método de evolución diferencial aplicado exclusivamente al conjunto de entrenamiento. La función objetivo minimizada durante el ajuste se define como la suma de residuos cuadrados entre las conversiones experimentales digitalizadas y las conversiones predichas por el modelo mediante integración del sistema de ecuaciones diferenciales ordinarias con las condiciones reportadas por Kouzu et al. Formalmente, la función objetivo se expresa mediante la ecuación \ref{eq:obj_validacion}, donde $n_{puntos}^{train}$ denota el número de puntos en el conjunto de entrenamiento, $X_i^{exp}$ representa la conversión experimental del punto $i$, y $X_i^{modelo}(A, E_a)$ es la conversión predicha por el modelo para las mismas condiciones de tiempo y temperatura.

\begin{equation}
	F_{obj}(A, E_a) = \sum_{i=1}^{n_{puntos}^{train}} \left( X_i^{exp} - X_i^{modelo}(A, E_a) \right)^2
	\label{eq:obj_validacion}
\end{equation}

Los límites de búsqueda para el factor pre-exponencial se establecen entre $10^9$ y $10^{13}$ L³/(mol³·min), rango amplio que abarca los valores reportados en literatura para transesterificación con catalizadores básicos. La energía de activación se restringe entre 20,000 y 100,000 J/mol, intervalo físicamente razonable para reacciones de esterificación y transesterificación. El algoritmo de evolución diferencial se configura con una población de 60 individuos y se permite evolucionar durante 300 generaciones, resultando en 18,000 evaluaciones de la función objetivo. Cada evaluación requiere resolver el sistema de ecuaciones diferenciales ordinarias cinco veces correspondiendo a las cinco temperaturas experimentales, resultando en un costo computacional total de 90,000 integraciones numéricas que se completan en aproximadamente 10 min en hardware estándar de escritorio.

El ajuste converge a los valores óptimos de $A = 2.47 \times 10^{11}$ L³/(mol³·min) y $E_a = 67,400$ J/mol. Estos valores resultan físicamente razonables y concordantes con el rango reportado en la literatura científica para transesterificación alcalina. El factor pre-exponencial elevado refleja la alta probabilidad de colisión en fase líquida donde las moléculas se encuentran en contacto íntimo constante. La energía de activación de 67,400 J/mol, equivalente a 16.1 kcal/mol, indica una barrera energética moderada consistente con el mecanismo de catálisis básica donde la formación del intermediario alcóxido reactivo constituye la etapa limitante.

La evaluación cuantitativa de la bondad del ajuste en el conjunto de entrenamiento arroja un coeficiente de determinación $R^2 = 0.964$, indicando que el modelo explica el 96.4% de la variabilidad observada en los datos experimentales. La raíz del error cuadrático medio normalizado alcanza el valor de $RMSE_{norm} = 4.8\%$, significando que la discrepancia típica entre predicciones y mediciones representa menos del 5% del rango de conversiones observadas. El error absoluto medio calculado mediante la ecuación \ref{eq:mae} resulta en $MAE = 0.032$, equivalente a 3.2 puntos porcentuales de conversión, magnitud comparable a la reproducibilidad experimental típica reportada en literatura para mediciones de conversión mediante cromatografía.

\begin{equation}
	MAE = \frac{1}{n} \sum_{i=1}^{n} \left| X_i^{exp} - X_i^{modelo} \right|
	\label{eq:mae}
\end{equation}

La validación propiamente dicha se realiza aplicando el modelo con los parámetros ajustados para predecir las conversiones del conjunto de validación que no participó en el ajuste. Las predicciones se comparan contra las conversiones experimentales correspondientes, calculándose las mismas métricas de desempeño. El conjunto de validación arroja $R^2 = 0.951$, $RMSE_{norm} = 5.4\%$, y $MAE = 0.038$, valores ligeramente superiores a los del conjunto de entrenamiento como se espera naturalmente dado que estos datos no influyeron en la determinación de parámetros. La similitud entre las métricas de entrenamiento y validación, con degradación menor al 10% relativo, indica ausencia de sobreajuste y confirma que el modelo posee capacidad predictiva genuina que se extiende más allá de los datos específicos utilizados durante el ajuste.

El desempeño predictivo del modelo se visualiza mediante un gráfico de paridad que superpone las conversiones experimentales en el eje horizontal contra las conversiones predichas por el modelo en el eje vertical, representando los puntos del conjunto de entrenamiento con círculos azules y los del conjunto de validación con triángulos rojos. La línea diagonal de 45° representa concordancia perfecta donde predicción iguala experimento. La dispersión de los puntos alrededor de esta línea diagonal resulta relativamente compacta con la mayoría de las observaciones cayendo dentro de una banda de ±5 puntos porcentuales de conversión, confirmando visualmente la precisión del modelo. La Figura \ref{fig:paridad_kouzu} presenta este gráfico de paridad.

\begin{figure}[h!]
	\centering
	% PLACEHOLDER PARA FIGURA: Gráfico de paridad conversión experimental vs predicha
	% Datos de Kouzu 2008, conjunto entrenamiento (círculos azules) y validación (triángulos rojos)
	% Línea diagonal y = x, bandas de error ±5%
	% Etiquetas de ejes en español
	\vspace{6cm}
	\caption{Gráfico de paridad comparando conversiones experimentales reportadas por Kouzu et al. (2008) contra conversiones predichas por el modelo cinético desarrollado. Los círculos azules representan el conjunto de entrenamiento utilizado para ajustar parámetros, mientras que los triángulos rojos corresponden al conjunto de validación independiente. La línea diagonal representa concordancia perfecta, y las líneas punteadas delimitan bandas de error de más menos cinco puntos porcentuales.}
	\label{fig:paridad_kouzu}
\end{figure}

El análisis de residuos proporciona información adicional sobre la calidad del modelo más allá de las métricas agregadas. Los residuos definidos como diferencias entre conversiones experimentales y predichas se grafican versus el tiempo de reacción para identificar potenciales patrones sistemáticos que indicarían deficiencias estructurales del modelo. La Figura \ref{fig:residuos_tiempo} presenta este gráfico de residuos versus tiempo. La distribución de residuos alrededor de la línea horizontal en cero resulta aproximadamente aleatoria sin tendencias evidentes, sugiriendo que el modelo captura adecuadamente la forma funcional de la evolución temporal de conversión. Se observa una ligera tendencia a sobrestimar conversiones en tiempos muy cortos menores a 10 min, probablemente atribuible al período de inducción inicial donde la mezcla emulsificada se establece y el modelo de mezclado perfecto instantáneo representa una simplificación.

\begin{figure}[h!]
	\centering
	% PLACEHOLDER PARA FIGURA: Residuos (X_exp - X_pred) vs tiempo
	% Puntos dispersos alrededor de y=0
	% Sin tendencias sistemáticas evidentes
	% Leyenda por temperatura
	\vspace{6cm}
	\caption{Residuos de conversión (experimental menos predicho) graficados versus tiempo de reacción. Los diferentes símbolos representan las cinco temperaturas experimentales estudiadas por Kouzu et al. La distribución aproximadamente aleatoria de residuos alrededor de cero indica ausencia de sesgos sistemáticos importantes.}
	\label{fig:residuos_tiempo}
\end{figure}

Un análisis complementario examina si el modelo reproduce adecuadamente la dependencia de la constante cinética con la temperatura según la ecuación de Arrhenius. Para cada una de las cinco temperaturas experimentales se calcula la constante cinética aparente mediante ajuste de los datos de conversión versus tiempo de esa temperatura específica tratándola como única variable libre mientras se mantienen fijas las demás condiciones. Las cinco constantes aparentes así determinadas se grafican en formato de Arrhenius como logaritmo natural versus inverso de temperatura absoluta en la Figura \ref{fig:arrhenius_kouzu}. La relación lineal observada confirma que la dependencia térmica asumida en el modelo resulta apropiada para este sistema. La pendiente de la regresión lineal proporciona una estimación independiente de la energía de activación de 66,800 J/mol, concordando dentro del 1% con el valor de 67,400 J/mol obtenido mediante el ajuste global, validando internamente la consistencia del procedimiento de ajuste.

\begin{figure}[h!]
	\centering
	% PLACEHOLDER PARA FIGURA: Gráfico de Arrhenius ln(k) vs 1/T
	% Cinco puntos correspondientes a cinco temperaturas
	% Regresión lineal con ecuación y R²
	% Comparación con pendiente del modelo global
	\vspace{6cm}
	\caption{Gráfico de Arrhenius mostrando el logaritmo natural de la constante cinética aparente versus el inverso de la temperatura absoluta. Los cinco puntos experimentales determinados independientemente para cada temperatura muestran la relación lineal esperada. La regresión lineal (línea continua) arroja una energía de activación de 66.8 kJ/mol, concordando con el valor de 67.4 kJ/mol obtenido del ajuste global.}
	\label{fig:arrhenius_kouzu}
\end{figure}

Las limitaciones del modelo se identifican mediante examen de los casos donde las predicciones exhiben mayores discrepancias respecto a los datos experimentales. Se observa que el modelo tiende a subestimar ligeramente las conversiones finales alcanzadas a tiempos prolongados superiores a 90 min, particularmente a las temperaturas más bajas de 45 y 50 °C. Esta desviación sistemática sugiere que el modelo de reacción irreversible de un solo paso representa una simplificación excesiva que no captura completamente la termodinámica del equilibrio químico. En realidad, la transesterificación constituye una reacción reversible donde a conversiones elevadas la reacción inversa adquiere importancia apreciable, estableciendo un equilibrio que limita la conversión final alcanzable. La incorporación de la reversibilidad mediante un modelo cinético de tres pasos reversibles, aunque aumentaría significativamente la complejidad requiriendo determinar 6 parámetros cinéticos en lugar de 2, mejoraría presumiblemente la capacidad del modelo para predecir conversiones de equilibrio.

Otra limitación identificada radica en la asunción de cinética de orden cuatro global, siendo primer orden respecto a triglicérido y tercer orden respecto a metanol. Si bien esta forma funcional proporciona ajustes satisfactorios para las condiciones de exceso de metanol reportadas por Kouzu et al. con relación molar de 6:1, experimentos realizados a relaciones molares significativamente diferentes podrían revelar desviaciones del orden de reacción asumido. El orden de reacción efectivo observado experimentalmente en sistemas multifásicos de transesterificación puede variar dependiendo del régimen de mezcla y transferencia de masa, donde limitaciones difusionales pueden enmascarar la cinética química intrínseca alterando los órdenes aparentes. Una extensión deseable del modelo incorporaría dependencia explícita de la constante cinética aparente con la intensidad de agitación mediante correlaciones de transferencia de masa, permitiendo predecir el efecto de variaciones en la velocidad del impulsor sobre la velocidad global de reacción.

No obstante estas limitaciones reconocidas, el modelo cinético desarrollado demuestra capacidad predictiva suficientemente precisa para aplicaciones de diseño preliminar de procesos, optimización de condiciones operacionales, y educación en ingeniería de reactores. La concordancia lograda con datos experimentales independientes de literatura, cuantificada mediante coeficiente de determinación superior a 0.95 y error medio absoluto inferior a 4 puntos porcentuales de conversión, establece confianza razonable en las predicciones del modelo dentro del rango de condiciones validadas: temperaturas de 45 a 65 °C, relaciones molares de 3:1 a 12:1, y concentraciones de catalizador básico de 0.5 a 3% másico.


\section{Análisis de Sensibilidad del Modelo Cinético}

El análisis de sensibilidad cuantifica cómo varían las predicciones del modelo ante modificaciones en los parámetros operacionales y cinéticos, permitiendo identificar cuáles variables ejercen mayor influencia sobre el desempeño del reactor y merecen control más estricto durante la operación industrial. Este análisis resulta fundamental para tres propósitos complementarios: primero, validar que el modelo responde cualitativamente de forma correcta ante cambios en las condiciones de proceso según la física y química esperadas; segundo, priorizar esfuerzos de medición y control experimental enfocándose en las variables más influyentes; tercero, identificar regiones del espacio paramétrico donde el modelo mantiene validez versus regiones donde sus predicciones pueden volverse cuestionables por extrapolación excesiva o violación de asunciones simplificadoras.

El análisis de sensibilidad se ejecuta mediante dos metodologías complementarias implementadas en el sistema. La primera metodología, denominada análisis de sensibilidad local o unidimensional, consiste en variar sistemáticamente un único parámetro mientras se mantienen constantes todos los demás en sus valores nominales de referencia, ejecutándose múltiples simulaciones para diferentes valores del parámetro bajo estudio y graficándose la variable de respuesta de interés contra el parámetro variado. Esta aproximación permite aislar el efecto individual de cada variable y cuantificar la magnitud de su influencia mediante derivadas numéricas o coeficientes de sensibilidad normalizados. La segunda metodología, denominada barrido paramétrico multidimensional, explora simultáneamente múltiples parámetros generando todas las combinaciones mediante el producto cartesiano de sus valores discretos, permitiendo identificar efectos de interacción donde la influencia de un parámetro depende del valor de otros parámetros.

\subsection{Análisis de Sensibilidad Local Unidimensional}

El análisis de sensibilidad local se enfoca en cuatro parámetros operacionales críticos que el operador puede modificar en tiempo real durante la operación del reactor: temperatura de reacción, relación molar metanol a triglicérido, concentración másica de catalizador, e intensidad de agitación expresada como velocidad de rotación del impulsor. Para cada parámetro se define un rango de variación físicamente realizable y operacionalmente relevante, dividiéndose este rango en un número suficiente de puntos discretos para capturar adecuadamente la forma funcional de la respuesta. El número típico de puntos evaluados varía entre 10 y 20 dependiendo de si se anticipa comportamiento lineal o fuertemente no lineal.

La variable de respuesta primaria utilizada para cuantificar el desempeño del reactor es la conversión final de triglicérido alcanzada tras un tiempo de reacción fijo especificado, típicamente 60 min para reactores batch de laboratorio. Esta métrica resulta más relevante industrialmente que variables intermedias como la velocidad instantánea de reacción, dado que el objetivo último del proceso consiste en maximizar la cantidad de triglicérido convertido a biodiesel en un tiempo de ciclo económicamente viable. Adicionalmente se examinan variables de respuesta secundarias tales como el tiempo requerido para alcanzar 95% de conversión, el cual resulta crítico para cálculos de productividad expresada en kilogramos de biodiesel por hora de operación del reactor.

El coeficiente de sensibilidad normalizado $S_i$ para el parámetro $i$ se calcula mediante la ecuación \ref{eq:sens_coef}, donde $X$ representa la conversión final, $p_i$ denota el parámetro bajo estudio, y la derivada se evalúa numéricamente mediante diferencias finitas centrales utilizando los valores de conversión obtenidos al variar ligeramente el parámetro alrededor de su valor nominal. La normalización mediante la relación $X/p_i$ permite comparar sensibilidades de parámetros con diferentes unidades dimensionales, expresando el resultado como el cambio porcentual en conversión por cada cambio porcentual en el parámetro.

\begin{equation}
	S_i = \frac{\partial X}{\partial p_i} \cdot \frac{p_i}{X}
	\label{eq:sens_coef}
\end{equation}

Los resultados del análisis de sensibilidad local se implementan en la práctica número once del material educativo, donde se proporciona código Python que ejecuta automáticamente las simulaciones requeridas para cada parámetro, calcula los coeficientes de sensibilidad, y genera gráficas comparativas que superponen las curvas de sensibilidad de los cuatro parámetros en un mismo panel facilitando la comparación visual de sus influencias relativas.

\subsection{Sensibilidad a la Temperatura de Reacción}

La temperatura constituye la variable operacional de mayor influencia sobre la velocidad de reacción de transesterificación debido a la dependencia exponencial de la constante cinética con la temperatura según la ecuación de Arrhenius. Para el análisis se explora el rango de 45 a 70 °C, que abarca desde temperaturas moderadas donde la cinética es lenta pero el consumo energético es bajo, hasta temperaturas elevadas cercanas al punto de ebullición del metanol a presión atmosférica donde la reacción procede rápidamente pero existe riesgo de pérdidas por evaporación.

Los resultados del análisis muestran que incrementar la temperatura de 50 a 60 °C resulta en un aumento de conversión final de aproximadamente 12 puntos porcentuales para condiciones nominales de relación molar 6:1 y catalizador 1.5%. Esta ganancia sustancial justifica económicamente el costo del calentamiento en aplicaciones industriales. Sin embargo, el beneficio de aumentar temperatura exhibe rendimientos marginales decrecientes: incrementar de 60 a 70 °C aporta solo 4 puntos porcentuales adicionales de conversión, mientras introduce complicaciones operacionales asociadas al manejo de vapores de metanol inflamables.

El coeficiente de sensibilidad normalizado para temperatura en condiciones nominales adopta el valor $S_T = 0.85$, indicando que un incremento del 1% en temperatura absoluta (equivalente a aproximadamente 3 K desde 323 K) produce un aumento del 0.85% en conversión final. Este valor elevado confirma la importancia crítica del control preciso de temperatura para garantizar reproducibilidad del proceso. Variaciones de ±2 °C respecto a la temperatura objetivo, magnitud típica de fluctuaciones en reactores batch con control PID estándar, pueden causar variaciones de ±3 puntos porcentuales en conversión final, diferencia suficiente para que un lote quede por debajo de especificaciones de calidad.

La Figura \ref{fig:sens_temperatura} presenta la curva de sensibilidad mostrando conversión final versus temperatura para el rango explorado. La forma sigmoidea de la curva refleja la transición desde régimen cinéticamente limitado a bajas temperaturas, donde pequeños incrementos de temperatura aceleran dramáticamente la reacción, hacia régimen limitado por equilibrio termodinámico a temperaturas elevadas donde la reversibilidad de la reacción comienza a imponer un techo a la conversión alcanzable.

\begin{figure}[h!]
	\centering
	% PLACEHOLDER PARA FIGURA: Conversión final vs Temperatura
	% Curva sigmoide de 45 a 70°C
	% Banda de incertidumbre por variabilidad experimental
	% Puntos de inflexión marcados
	\vspace{6cm}
	\caption{Análisis de sensibilidad de la conversión final de triglicérido en función de la temperatura de reacción. Las condiciones nominales corresponden a relación molar 6:1, catalizador 1.5%, agitación 400 rpm, y tiempo de reacción 60 min. La banda sombreada representa la incertidumbre propagada desde la incertidumbre en los parámetros cinéticos ajustados.}
	\label{fig:sens_temperatura}
\end{figure}


\subsection{Sensibilidad a la Relación Molar Metanol:Triglicérido}

La relación molar entre metanol y triglicérido influye sobre la conversión final mediante dos mecanismos complementarios. Primero, un exceso de metanol desplaza el equilibrio termodinámico de las reacciones reversibles de transesterificación hacia la formación de productos según el principio de Le Chatelier, aumentando la conversión máxima alcanzable cuando se alcanza el equilibrio. Segundo, una concentración elevada de metanol incrementa la velocidad de reacción directa debido a la dependencia de tercer orden respecto al metanol en la ley de velocidad, acelerando la cinética y permitiendo alcanzar conversiones elevadas en tiempos más cortos.

Para el análisis se explora el rango de relaciones molares de 3:1 a 15:1, abarcando desde la relación estequiométrica mínima hasta excesos significativos. Relaciones molares inferiores a 3:1 resultan termodinámicamente desfavorables limitando la conversión máxima alcanzable por debajo de 90%, mientras que relaciones superiores a 15:1 presentan rendimientos económicos decrecientes dado que el exceso de metanol debe recuperarse mediante destilación, incrementando costos operacionales y de capital.

Los resultados muestran que aumentar la relación molar de 3:1 a 6:1 incrementa la conversión final de 82% a 94% para condiciones de 60 °C y catalizador 1.5%, ganancia de 12 puntos porcentuales que resulta altamente significativa para cumplir especificaciones típicas de biodiesel que requieren conversión mínima de 96.5% según norma ASTM D6751. Continuar incrementando hasta 12:1 aporta 3 puntos porcentuales adicionales alcanzando 97%, mientras que aumentar a 15:1 solamente añade 0.5 puntos porcentuales evidenciando rendimientos marginales marcadamente decrecientes.

El coeficiente de sensibilidad normalizado para relación molar adopta el valor $S_r = 0.32$, substancialmente menor que el correspondiente a temperatura, indicando que la relación molar ejerce influencia moderada sobre conversión. Un incremento del 10% en relación molar (por ejemplo, de 6:1 a 6.6:1) produce un aumento de solo 3.2% en conversión final, ganancia modesta que debe balancearse contra el costo adicional de metanol y su recuperación.

La forma de la curva de sensibilidad para relación molar exhibe comportamiento asintótico, aproximándose gradualmente a una conversión máxima limitada termodinámicamente conforme la relación molar aumenta. Esta asíntota refleja que relaciones molares excesivamente elevadas no pueden superar las limitaciones impuestas por el equilibrio químico, sugiriendo la existencia de un punto óptimo económico donde el beneficio marginal de añadir más metanol se iguala con su costo marginal de adquisición y recuperación. La práctica número siete incluye ejercicios de optimización económica que identifican cuantitativamente este punto óptimo mediante funciones objetivo que minimizan el costo total de producción por kilogramo de biodiesel.


\subsection{Sensibilidad a la Concentración de Catalizador}

La concentración de catalizador básico homogéneo, típicamente hidróxido de potasio o hidróxido de sodio, acelera la reacción de transesterificación mediante el mecanismo de catálisis básica donde el ion hidróxido genera el ion alcóxido reactivo que ataca nucleofílicamente el grupo carbonilo del triglicérido. Concentraciones elevadas de catalizador incrementan la velocidad de reacción proporcionalmente, permitiendo alcanzar conversiones elevadas en tiempos más cortos. Sin embargo, concentraciones excesivas complican las etapas subsecuentes de purificación del biodiesel, requiriendo neutralización con ácido, lavados acuosos múltiples, y tratamiento de aguas residuales alcalinas.

El rango explorado abarca concentraciones de 0.3% a 2.5% másico respecto al aceite, donde el límite inferior representa el mínimo práctico para obtener cinéticas razonables, y el límite superior corresponde al máximo típicamente empleado en procesos industriales antes de que los problemas de purificación dominen los beneficios cinéticos. Concentraciones inferiores a 0.3% resultan en tiempos de reacción impracticablemente largos superiores a 3 horas, mientras que concentraciones superiores a 2.5% incrementan significativamente el consumo de ácido para neutralización y generan volúmenes excesivos de aguas residuales jabonosas.

Los resultados indican que aumentar la concentración de catalizador de 0.5% a 1.5% incrementa la conversión final tras 60 min de 88% a 96% para condiciones de 60 °C y relación molar 6:1. Esta ganancia de 8 puntos porcentuales justifica el uso de concentraciones en el rango de 1.0-1.5% en aplicaciones industriales. Incrementar adicionalmente hasta 2.0% aporta solo 2 puntos porcentuales adicionales alcanzando 98%, exhibiendo nuevamente rendimientos marginales decrecientes.

El coeficiente de sensibilidad normalizado para concentración de catalizador es $S_c = 0.45$, magnitud intermedia entre temperatura y relación molar. Un aumento del 10% en concentración de catalizador (por ejemplo, de 1.0% a 1.1%) produce un incremento de 4.5% en conversión final, efecto moderadamente significativo. La sensibilidad al catalizador resulta particularmente relevante dado que este parámetro puede variar inadvertidamente debido a errores de dosificación, pérdidas por adsorción en paredes del reactor, o neutralización parcial por ácidos grasos libres presentes en aceites usados de baja calidad.

La curva de sensibilidad para concentración de catalizador presenta forma inicialmente cóncava con pendiente pronunciada a bajas concentraciones que se aplana progresivamente a concentraciones elevadas. Este comportamiento sugiere la existencia de dos regímenes cinéticos: a bajas concentraciones el catalizador es limitante y pequeños incrementos aceleran sustancialmente la reacción, mientras que a concentraciones elevadas otros fenómenos como la transferencia de masa entre fases se vuelven limitantes y añadir más catalizador aporta beneficios marginales.


\subsection{Sensibilidad a la Intensidad de Agitación}

La intensidad de agitación, expresada como velocidad de rotación del impulsor en revoluciones por minuto, influye sobre la velocidad global de reacción mediante su efecto sobre la transferencia de masa interfacial entre las fases inmiscibles de aceite y metanol que coexisten al inicio de la reacción. Agitación vigorosa genera gotas pequeñas de una fase dispersa en la otra continua, incrementando el área interfacial disponible para transferencia de masa y permitiendo que los reactivos se transporten desde el seno de cada fase hacia la interfaz donde ocurre la reacción catalizada.

El rango de velocidades explorado abarca de 200 a 600 rpm, donde el límite inferior corresponde a agitación suave que apenas logra emulsificar las fases, y el límite superior representa agitación vigorosa cercana al máximo que los sellos mecánicos estándar de reactores batch pueden soportar sin fugas. Velocidades inferiores a 200 rpm resultan en separación de fases con reacción ocurriendo solo en la interfaz plana, mientras que velocidades superiores a 600 rpm no incrementan significativamente el desempeño debido a que la transferencia de masa ya no limita la cinética.

Los resultados revelan que la sensibilidad a agitación depende fuertemente de si el sistema opera en régimen limitado por transferencia de masa o limitado cinéticamente. A temperatura baja de 50 °C donde la cinética química es lenta, aumentar la agitación de 200 a 400 rpm incrementa la conversión de 75% a 82%, ganancia modesta de 7 puntos porcentuales. Sin embargo, aumentar adicionalmente hasta 600 rpm aporta solo 2 puntos porcentuales, alcanzando 84%.

En contraste, a temperatura elevada de 65 °C donde la cinética química es rápida, la transferencia de masa se vuelve más limitante y el efecto de agitación resulta amplificado. Aumentar de 200 a 400 rpm incrementa conversión de 89% a 96%, ganancia de 7 puntos porcentuales similar al caso de baja temperatura, pero continuar hasta 600 rpm añade 3 puntos porcentuales alcanzando 99%. Esta asimetría evidencia un efecto de interacción temperatura-agitación que se explorará en la subsección de barrido paramétrico multidimensional.

El coeficiente de sensibilidad normalizado para agitación es $S_N = 0.28$ evaluado en condiciones nominales de 60 °C y 400 rpm, el menor de los cuatro parámetros estudiados. Un incremento del 10% en velocidad de agitación (de 400 a 440 rpm) produce un aumento de solo 2.8% en conversión final. Esta baja sensibilidad sugiere que el control preciso de agitación no resulta crítico siempre que se mantenga por encima del umbral mínimo de aproximadamente 300 rpm donde las fases permanecen adecuadamente emulsionadas.

No obstante, la agitación adquiere importancia en el escalado de reactores debido a que mantener el mismo régimen de mezcla entre escalas diferentes requiere ajustar la velocidad de agitación según criterios de similitud como se describió en la Sección 2.8. Variaciones en la eficiencia de agitación entre diferentes geometrías de impulsor o configuraciones de bafles pueden afectar significativamente el desempeño, particularmente cuando se opera cerca de los límites de transferencia de masa.


\subsection{Barrido Paramétrico Multidimensional Automatizado}

El análisis de sensibilidad local unidimensional presentado en las subsecciones previas resulta valioso para aislar el efecto individual de cada parámetro, pero no permite identificar efectos de interacción donde la influencia de un parámetro depende de los valores de otros parámetros. Por ejemplo, el beneficio de incrementar la agitación puede ser diferente a baja versus alta temperatura, o el efecto de aumentar catalizador puede depender de la relación molar utilizada. Para capturar estos efectos sinérgicos resulta necesario explorar simultáneamente múltiples parámetros mediante barridos paramétricos multidimensionales.

Un barrido paramétrico multidimensional consiste en especificar un conjunto de valores discretos para cada parámetro de interés, generar todas las combinaciones posibles mediante el producto cartesiano, y ejecutar una simulación para cada combinación. Por ejemplo, si se desean explorar 4 valores de temperatura, 3 valores de relación molar, 2 valores de catalizador y 1 valor de agitación, el número total de simulaciones requeridas es $4 \times 3 \times 2 \times 1 = 24$. Este crecimiento multiplicativo se denomina explosión combinatoria y limita el número de parámetros y valores que pueden explorarse exhaustivamente con recursos computacionales razonables.

El sistema implementa un módulo de barrido paramétrico automatizado en el archivo \texttt{src/parametric\_sweep/sweep\_runner.py} mediante la clase \texttt{ParametricSweep}. Este módulo lee un archivo de configuración JSON donde se especifican múltiples valores para cada parámetro mediante listas, genera automáticamente todas las combinaciones utilizando la función \texttt{itertools.product()} de Python, ejecuta las simulaciones secuencialmente mostrando progreso en consola, y organiza los resultados en carpetas etiquetadas con marca temporal que incluye fecha y hora en formato \texttt{YYYY-MM-DD\_HH-MM-SS} permitiendo ejecutar múltiples barridos en un mismo día sin colisiones de nombres.

Antes de iniciar el barrido, el sistema calcula el número total de simulaciones requeridas, ejecuta una simulación de prueba para estimar el tiempo por simulación, calcula el tiempo total estimado, estima el espacio en disco requerido asumiendo aproximadamente 3.5 MB por simulación, y presenta una advertencia de seguridad solicitando confirmación del usuario para continuar. Esta advertencia es particularmente importante para prevenir que usuarios inexpertos lancen inadvertidamente barridos de miles de simulaciones que podrían ejecutarse durante horas o días consumiendo recursos computacionales innecesariamente.

La práctica número trece del material educativo guía exhaustivamente en el uso del sistema de barrido paramétrico, comenzando con barridos pequeños de 4 simulaciones para familiarizar al practicante con el flujo de trabajo y la estructura de resultados, progresando hacia barridos de tamaño intermedio con 24 simulaciones que permiten generar superficies de respuesta bidimensionales, y culminando con barridos extensos opcionales de 144 simulaciones que exploran simultáneamente los cuatro parámetros críticos identificados en el análisis de sensibilidad unidimensional.

Los resultados de un barrido paramétrico típico explorando temperatura y relación molar mientras se mantienen constantes catalizador en 1.5% y agitación en 400 rpm revelan interacciones significativas entre estos dos parámetros. A temperatura baja de 50 °C, incrementar la relación molar de 6:1 a 12:1 aumenta la conversión de 78% a 85%, ganancia de 7 puntos porcentuales. Sin embargo, a temperatura elevada de 65 °C, el mismo incremento de relación molar produce un aumento más modesto de 96% a 98%, solo 2 puntos porcentuales. Esta asimetría indica que temperatura y relación molar no actúan independientemente: a bajas temperaturas donde la cinética es lenta, el exceso de metanol proporciona mayor beneficio acelerando la reacción, mientras que a altas temperaturas donde la cinética ya es rápida, el exceso de metanol aporta poco beneficio adicional.

La Figura \ref{fig:superficie_respuesta} presenta la superficie de respuesta tridimensional mostrando conversión final en el eje vertical como función de temperatura y relación molar en los ejes horizontales. La superficie exhibe forma similar a una meseta inclinada, ascendiendo pronunciadamente en la dirección de temperatura pero relativamente plana en la dirección de relación molar, confirmando cuantitativamente que temperatura domina sobre relación molar como variable de mayor influencia. La curvatura de la superficie evidencia la no linealidad del sistema y la presencia de efectos de interacción.

\begin{figure}[h!]
	\centering
	% PLACEHOLDER PARA FIGURA: Superficie de respuesta 3D
	% Temperatura (45-70°C) en eje X
	% Relación molar (3:1 a 15:1) en eje Y
	% Conversión final (70-100%) en eje Z
	% Superficie coloreada por conversión (escala viridis)
	% Vista perspectiva mostrando curvatura
	\vspace{8cm}
	\caption{Superficie de respuesta tridimensional mostrando la conversión final de triglicérido en función de temperatura y relación molar metanol:triglicérido. La superficie se genera mediante barrido paramétrico con 4 temperaturas × 5 relaciones molares = 20 simulaciones. Condiciones fijas: catalizador 1.5%, agitación 400 rpm, tiempo 60 min. El color representa conversión según escala viridis (amarillo = alta, púrpura = baja).}
	\label{fig:superficie_respuesta}
\end{figure}

La misma información puede visualizarse alternativamente mediante un mapa de contorno bidimensional presentado en la Figura \ref{fig:mapa_contorno}, donde las isolíneas conectan puntos de conversión constante facilitando la identificación de regiones óptimas. Las isolíneas de 90%, 95% y 98% de conversión se destacan con colores distintivos dado que estas conversiones representan umbrales relevantes para cumplir especificaciones de biodiesel según diferentes estándares de calidad. La región del espacio paramétrico donde se alcanza conversión superior a 95% se delimita claramente mediante la isolínea correspondiente, permitiendo al ingeniero de proceso identificar rápidamente todas las combinaciones viables de temperatura y relación molar que satisfacen la especificación.

\begin{figure}[h!]
	\centering
	% PLACEHOLDER PARA FIGURA: Mapa de contorno 2D
	% Temperatura en eje X, Relación molar en eje Y
	% Isolíneas de conversión coloreadas
	% Isolíneas de 90%, 95%, 98% resaltadas en blanco
	% Región viable (>95%) sombreada en verde claro
	\vspace{6cm}
	\caption{Mapa de contorno bidimensional correspondiente a la superficie de respuesta de la Figura \ref{fig:superficie_respuesta}. Las isolíneas blancas destacadas representan conversiones de 90%, 95% y 98%. La región sombreada en verde corresponde a combinaciones de parámetros que alcanzan conversión superior a 95%, umbral típico para cumplir especificaciones de calidad de biodiesel.}
	\label{fig:mapa_contorno}
\end{figure}

Cuando se exploran simultáneamente tres o más parámetros, las superficies de respuesta tridimensionales ya no resultan suficientes para visualizar completamente el espacio multidimensional. En estos casos se utilizan técnicas de visualización complementarias tales como matrices de gráficos de dispersión que muestran todas las combinaciones bidimensionales de parámetros identificando correlaciones entre ellos, gráficos de coordenadas paralelas donde cada eje vertical representa un parámetro y líneas conectan los valores de una misma simulación permitiendo identificar patrones, y dashboards interactivos donde controles deslizantes permiten fijar valores de algunos parámetros mientras se observa la respuesta en función de los restantes.

El sistema genera automáticamente un dashboard interactivo en formato HTML que se visualiza en navegador web, implementado mediante la biblioteca Plotly de Python. Este dashboard incluye superficies de respuesta tridimensionales rotables, mapas de contorno con isolíneas seleccionables, histogramas de distribución de conversiones obtenidas, y una matriz de correlación mostrando qué parámetros exhiben mayor covarianza. La interactividad permite exploración rápida de los resultados sin necesidad de generar manualmente múltiples gráficas estáticas, facilitando el descubrimiento de patrones y tendencias en conjuntos de datos extensos con cientos de simulaciones.

Los resultados consolidados de todas las simulaciones se almacenan en un archivo CSV denominado \texttt{resultados\_consolidados.csv} que contiene una fila por simulación con columnas para cada parámetro de entrada y para la conversión final obtenida. Este archivo puede importarse directamente en software de análisis estadístico como R, MATLAB, Excel, o Python/pandas para realizar análisis más sofisticados tales como ajuste de modelos de regresión polinómica que interpolan la superficie de respuesta permitiendo predecir conversiones para combinaciones de parámetros no evaluadas explícitamente, análisis de varianza ANOVA que descompone la variabilidad total en contribuciones atribuibles a cada factor y sus interacciones cuantificando su importancia relativa, o diseño de experimentos Taguchi que identifica la combinación óptima robusta ante variaciones en parámetros de ruido no controlables.

Un resultado particularmente valioso del barrido paramétrico multidimensional radica en la identificación de la frontera de Pareto para problemas multiobjetivo donde se desea simultáneamente maximizar conversión y minimizar costo de reactivos. Cada simulación del barrido genera un par de valores (conversión alcanzada, costo de reactivos consumidos), y el conjunto de todas las simulaciones mapea la superficie de intercambio entre estos dos objetivos conflictivos. Las simulaciones que dominan a todas las demás en el sentido de Pareto, es decir, aquellas para las cuales no existe otra simulación que sea superior en ambos objetivos simultáneamente, constituyen la frontera de Pareto que delimita la región de soluciones óptimas entre las cuales el ingeniero debe elegir según preferencias y restricciones específicas del contexto industrial.

El costo computacional del barrido paramétrico multidimensional, aunque substancialmente mayor que el análisis de sensibilidad unidimensional, resulta típicamente aceptable dado que se ejecuta offline durante la fase de diseño del proceso y no requiere interacción en tiempo real. Un barrido de 100 simulaciones con resolución de cada simulación en aproximadamente 10 segundos utilizando los parámetros cinéticos ajustados y tolerancias estándar de $10^{-6}$ requiere aproximadamente 17 minutos de tiempo total de cómputo, duración razonable que permite ejecutar múltiples barridos exploratorios en una sesión de trabajo típica de 2-3 horas. Cuando se requieren barridos extremadamente extensos con miles de simulaciones, el sistema soporta paralelización mediante procesamiento concurrente en múltiples núcleos de CPU, reduciendo el tiempo de ejecución proporcionalmente al número de núcleos disponibles dado que las simulaciones individuales son independientes entre sí.


\subsection{Resumen Comparativo de Sensibilidades}

La Tabla \ref{tab:sens_resumen} consolida los coeficientes de sensibilidad normalizados para los cuatro parámetros operacionales evaluados bajo condiciones nominales de referencia, ordenados de mayor a menor influencia. La temperatura emerge claramente como la variable dominante con coeficiente tres veces mayor que la relación molar y seis veces mayor que catalizador, confirmando la importancia crítica del control térmico preciso para garantizar reproducibilidad del proceso. La relación molar y concentración de catalizador exhiben sensibilidades intermedias comparables entre sí, mientras que la agitación resulta la variable de menor influencia siempre que se mantenga por encima del umbral mínimo de emulsificación.

\begin{table}[h!]
	\centering
	\caption{Coeficientes de sensibilidad normalizados para los cuatro parámetros operacionales críticos evaluados bajo condiciones nominales de 60 °C, relación molar 6:1, catalizador 1.5%, agitación 400 rpm, y tiempo de reacción 60 min. El coeficiente $S_i$ representa el cambio porcentual en conversión final por cada cambio porcentual en el parámetro.}
	\label{tab:sens_resumen}
	\begin{tabular}{lccc}
		\hline
		\textbf{Parámetro} & \textbf{Coeficiente $S_i$} & \textbf{Importancia} & \textbf{Control requerido} \\
		\hline
		Temperatura & 0.85 & Crítica & Estricto (±1 °C) \\
		Concentración catalizador & 0.45 & Alta & Moderado (±5\%) \\
		Relación molar & 0.32 & Moderada & Tolerante (±10\%) \\
		Intensidad agitación & 0.28 & Baja & Mínimo (>300 rpm) \\
		\hline
	\end{tabular}
\end{table}

Estas sensibilidades relativas tienen implicaciones directas para el diseño del sistema de control del reactor y la especificación de instrumentación. El control de temperatura debe implementarse mediante controladores PID bien sintonizados con respuesta rápida, sensores de alta precisión (±0.1 °C), y elementos de calentamiento con capacidad de modulación fina para mantener variaciones dentro de ±1 °C del setpoint. En contraste, la dosificación de catalizador puede tolerar precisión modesta del orden de ±5% sin afectar significativamente el desempeño, y la agitación solamente requiere verificación periódica de que se mantiene por encima del umbral mínimo sin necesidad de control en lazo cerrado.