\documentclass{syx7}

% ============================================================================
% SOLUCIÓN A CONFLICTOS DE PAQUETES
% ============================================================================
% IMPORTANTE: Estos comandos deben estar ANTES de cargar cualquier paquete
% para prevenir conflictos con amsthm y biblatex que la clase syx7 carga internamente
\let\openbox\relax
\let\c@author\relax
\let\not=\relax
\let\not<\relax
\let\not>\relax

% ============================================================================
% PAQUETES NECESARIOS
% ============================================================================
\usepackage{amsmath}
\usepackage{graphicx}
% NOTA: NO cargar amsthm ni amssymb explícitamente - la clase syx7 los maneja
\usepackage{caption}
\usepackage{color}
\usepackage{float}
\usepackage{tikz}
\usepackage{epsfig}
\usepackage{epstopdf}
\usepackage[ruled,vlined]{algorithm2e}

% Configuración de idioma
\usepackage[spanish]{babel}

% NOTA: NO cargar fontspec ni unicode-math explícitamente si usa XeLaTeX
% La clase syx7 maneja la configuración de fuentes automáticamente

% ============================================================================
% CONFIGURACIÓN DE GEOMETRÍA
% ============================================================================
\geometry{
	a4paper,
	left=25mm,
	right=20mm,
	top=25mm,
	bottom=25mm
}

\setlength{\abstractwidth}{\textwidth}
\setlength{\headheight}{59pt}

% ============================================================================
% CONFIGURACIÓN DE BIBLIOGRAFÍA
% ============================================================================
\usepackage[backend=bibtex,style=ieee,natbib=true]{biblatex}
\defbibheading{bibliography}{\section*{Referencias}}
\addbibresource{biblio.bib}  % Cambie el nombre del archivo .bib si es necesario

\usepackage{etoolbox}
\patchcmd{\bibsetup}{\begingroup}{\begingroup\let\clearpage\relax}{}{}

% ============================================================================
% CONFIGURACIÓN DE HYPERREF (Enlaces y Citas)
% ============================================================================
% Configurar hipervínculos: citas en azul sin recuadro, enlaces internos en negro
\hypersetup{
	colorlinks=true,        % Activar colores en lugar de recuadros
	linkcolor=black,        % Color de enlaces internos (secciones, figuras, etc.)
	citecolor=blue,         % Color de citas bibliográficas en AZUL
	urlcolor=blue,          % Color de URLs
	filecolor=blue,         % Color de enlaces a archivos
	pdfborder={0 0 0},      % Sin borde en los enlaces
	breaklinks=true         % Permitir saltos de línea en enlaces largos
}

% ============================================================================
% CONFIGURACIÓN DE CÓDIGO PYTHON (OPCIONAL)
% ============================================================================
\setpythonpath{./}

% ============================================================================
% METADATOS DEL DOCUMENTO
% ============================================================================
% Fechas de recepción y aceptación
\receiveddate{01-ene-2024}
\accepteddate{01-dic-2024}

% Título completo y título corto para el encabezado
\title{Título de su artículo científico aquí}
\shorttitle{Título corto para encabezado}

% Palabras clave separadas por comas
\keywords{Palabra clave 1, Palabra clave 2, Palabra clave 3, Palabra clave 4}

% ============================================================================
% DOCUMENTO
% ============================================================================
\begin{document}

% ============================================================================
% AUTORES Y CORREOS
% ============================================================================
% Agregue tantos autores como necesite usando el formato:
% \author{Nombre Completo del Autor}
% \email{correo@institucion.edu}

\title{Título de su artículo científico aquí}
\author{Primer Autor*}
\email{autor1@institucion.edu}
\author{Segundo Autor}
\email{autor2@institucion.edu}

\vspace*{-1 \baselineskip}

\maketitle

% ============================================================================
% RESUMEN
% ============================================================================
\begin{abstract}
	Escriba aquí el resumen de su artículo. El resumen debe ser una descripción concisa y precisa del contenido del artículo, incluyendo los objetivos, la metodología empleada, los resultados principales y las conclusiones más relevantes. Se recomienda que el resumen tenga entre 150 y 250 palabras.
\end{abstract}

% ============================================================================
% INTRODUCCIÓN
% ============================================================================
\section{Introducción}
Escriba aquí la introducción de su artículo. La introducción debe presentar el contexto del problema, la motivación del trabajo, los objetivos y la estructura del documento.

Puede citar referencias bibliográficas usando \texttt{\textbackslash cite\{clave\_referencia\}} como en este ejemplo: \cite{ejemplo_referencia}.

% ============================================================================
% DESARROLLO (SECCIONES PRINCIPALES)
% ============================================================================
\section{Marco Teórico}
Desarrolle aquí el marco teórico de su investigación.

\subsection{Subsección de ejemplo}
Puede crear subsecciones según sea necesario.

\section{Materiales y Métodos}
Describa aquí los materiales y métodos utilizados en su investigación.

% Ejemplo de figura
\begin{figure}[htb!]
	\centering
	% Descomente la siguiente línea y ajuste la ruta de la imagen
	% \includegraphics[width=0.5\textwidth]{img/nombre_imagen.png}
	\caption{Descripción de la figura}
	\label{fig:ejemplo}
\end{figure}

% Ejemplo de tabla
\begin{table}[htb!]
	\caption{Título de la tabla}
	\label{tab:ejemplo}
	\centering
	\begin{tabular}{>{\centering\arraybackslash}p{4cm}>{\centering\arraybackslash}p{4cm}}
		\toprule
		Columna 1 & Columna 2 \\
		\midrule
		Dato 1 & Dato 2 \\
		Dato 3 & Dato 4 \\
		\bottomrule
	\end{tabular}
\end{table}

% Ejemplo de ecuación
\begin{equation}
	E = mc^2
	\label{eq:ejemplo}
\end{equation}

\section{Resultados}
Presente aquí los resultados obtenidos en su investigación. Puede hacer referencia a figuras, tablas y ecuaciones usando \texttt{\textbackslash ref\{etiqueta\}}, por ejemplo: ver Figura \ref{fig:ejemplo}, Tabla \ref{tab:ejemplo} o Ecuación \ref{eq:ejemplo}.

\section{Discusión}
Desarrolle aquí la discusión de los resultados obtenidos.

\section{Conclusiones}
Presente aquí las conclusiones de su trabajo de investigación.

% ============================================================================
% CORREOS ELECTRÓNICOS DE LOS AUTORES
% ============================================================================
\printauthoremails

% ============================================================================
% BIBLIOGRAFÍA
% ============================================================================
\pretocmd{\printbibliography}{\vspace{1em}}{}{}
\printbibliography

\end{document}
