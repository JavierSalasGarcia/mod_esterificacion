\documentclass{syx7}

% ============================================================================
% PAQUETES NECESARIOS
% ============================================================================
% NOTA: La clase syx7 ya carga amsthm, amsmath y otros paquetes basicos
% No se deben recargar para evitar conflictos

\usepackage{graphicx}
\usepackage{caption}
\usepackage{subcaption}
\usepackage{color}
\usepackage{float}
\usepackage{tikz}
\usepackage{epsfig}
\usepackage{epstopdf}
% \usepackage[ruled,vlined]{algorithm2e}  % Comentado - no disponible
\usepackage{booktabs}
\usepackage{multirow}
\usepackage{listings}

% Configuración de idioma
\usepackage[spanish]{babel}

% ============================================================================
% CONFIGURACIÓN DE GEOMETRÍA
% ============================================================================
\geometry{
	a4paper,
	left=25mm,
	right=20mm,
	top=25mm,
	bottom=25mm
}

\setlength{\abstractwidth}{\textwidth}
\setlength{\headheight}{59pt}

% ============================================================================
% CONFIGURACIÓN DE BIBLIOGRAFÍA
% ============================================================================
% NOTA: etoolbox ya esta cargado por la clase syx7
\usepackage[backend=bibtex,style=ieee,natbib=true]{biblatex}
\defbibheading{bibliography}{\section*{Referencias}}
\addbibresource{biblio.bib}

\patchcmd{\bibsetup}{\begingroup}{\begingroup\let\clearpage\relax}{}{}

% ============================================================================
% CONFIGURACIÓN DE HYPERREF
% ============================================================================
\hypersetup{
	colorlinks=true,
	linkcolor=black,
	citecolor=blue,
	urlcolor=blue,
	filecolor=blue,
	pdfborder={0 0 0},
	breaklinks=true
}

% ============================================================================
% CONFIGURACIÓN DE CÓDIGO
% ============================================================================
\lstset{
	basicstyle=\ttfamily\small,
	breaklines=true,
	frame=single,
	language=Python,
	showstringspaces=false,
	commentstyle=\color{gray},
	keywordstyle=\color{blue},
	stringstyle=\color{red}
}

% ============================================================================
% METADATOS DEL DOCUMENTO
% ============================================================================
\receiveddate{01-nov-2024}
\accepteddate{20-nov-2024}

\title{Sistema Modular en Python para Modelado Cinetico de Biodiesel: Alternativa de Codigo Abierto a Software Comercial}
\shorttitle{Sistema Python para Modelado de Biodiesel}

\keywords{biodiesel, transesterificacion, modelado cinetico, Python, codigo abierto, educacion quimica, optimizacion de procesos}

% ============================================================================
% DOCUMENTO
% ============================================================================
\begin{document}

\author{Autor Principal*}
\email{autor@universidad.edu}

\vspace*{-1 \baselineskip}

\maketitle

% ============================================================================
% RESUMEN
% ============================================================================
\begin{abstract}
El software comercial para simulacion de procesos quimicos como ASPEN Plus representa costos elevados para instituciones educativas y laboratorios de investigacion. Este trabajo presenta un sistema modular completo en Python de codigo abierto para el modelado cinetico de produccion de biodiesel mediante transesterificacion de aceite de cocina usado. El sistema integra procesamiento de datos experimentales de cromatografia de gases, ajuste de parametros cineticos mediante regresion no lineal, optimizacion operacional y criterios de escalado desde reactores de laboratorio (350 mL) hasta escala piloto (20 L). Se desarrollo una metodologia educativa basada en 12 practicas progresivas que guian al estudiante desde conceptos basicos de Python hasta modelado avanzado y CFD, sin requerir conocimientos previos de programacion. El modelo fue validado con datos de literatura mostrando desviaciones menores al 5\% en parametros cineticos. Un analisis de sensibilidad identifica la temperatura como el parametro mas critico, seguido por la relacion molar metanol:triglicerido. El sistema es completamente configurable mediante archivos JSON, permitiendo adaptar el modelo a diferentes aceites, catalizadores y geometrias de reactor. Se incluye una guia completa de instalacion y uso para Windows. Este trabajo demuestra que herramientas open-source pueden ofrecer una alternativa viable, transparente y economica a software comercial propietario, facilitando el acceso a tecnologias de modelado para estudiantes e investigadores con recursos limitados.
\end{abstract}

% ============================================================================
% 1. INTRODUCCIÓN
% ============================================================================
\section{Introduccion}

\subsection{Contexto del Biodiesel}

El biodiesel de segunda generacion, producido a partir de aceites de cocina usados, representa una alternativa sostenible a los combustibles fosiles. La reaccion de transesterificacion convierte trigliceridos en esteres metilicos de acidos grasos (FAME) utilizando metanol y un catalizador alcalino. La optimizacion de este proceso requiere comprender la cinetica de reaccion, el efecto de variables operacionales (temperatura, agitacion, relacion molar) y criterios de escalado desde laboratorio hasta escala industrial \cite{kouzu2008}.

El modelado matematico es esencial para:
\begin{itemize}
	\item Predecir conversiones bajo diferentes condiciones
	\item Optimizar parametros operacionales
	\item Reducir costos de experimentacion
	\item Diseñar reactores a escala piloto e industrial
\end{itemize}

\subsection{Software Comercial: Capacidades y Limitaciones}

El software de simulacion comercial como ASPEN Plus, HYSYS y gPROMS ha sido ampliamente utilizado en la industria quimica por decadas \cite{aspen2023}. Estas herramientas ofrecen capacidades avanzadas de modelado, bases de datos de propiedades termodinamicas y interfaces graficas intuitivas.

Sin embargo, presentan limitaciones significativas para contextos educativos y de investigacion:

\begin{table}[htb!]
	\caption{Comparacion entre Software Comercial y Sistema Python Open-Source}
	\label{tab:comparacion_software}
	\centering
	\small
	\begin{tabular}{>{\raggedright\arraybackslash}p{3.5cm}>{\raggedright\arraybackslash}p{4cm}>{\raggedright\arraybackslash}p{4cm}}
		\toprule
		\textbf{Criterio} & \textbf{ASPEN Plus / HYSYS} & \textbf{Sistema Python} \\
		\midrule
		Costo anual & \$15,000 - \$50,000 USD por licencia & \$0 (codigo abierto) \\
		\midrule
		Tiempo de configuracion & 2-4 semanas (curva de aprendizaje pronunciada) & 2-4 horas (instalacion y primera practica) \\
		\midrule
		Acceso al codigo fuente & No (caja negra) & Si (codigo completo disponible) \\
		\midrule
		Personalizacion & Limitada (solo via API compleja) & Total (Python modificable) \\
		\midrule
		Integracion con datos propios & Requiere modulos especificos & Directa (CSV, Excel, JSON) \\
		\midrule
		Curva de aprendizaje para quimicos & Pronunciada (interfaz compleja) & Gradual (12 practicas progresivas) \\
		\midrule
		Mantenimiento & Requiere renovacion anual & Gratuito, comunidad activa \\
		\midrule
		Documentacion & Extensa pero tecnica & Tutorial completo integrado \\
		\bottomrule
	\end{tabular}
\end{table}

El costo elevado de licencias comerciales limita el acceso para:
\begin{itemize}
	\item Universidades con presupuestos reducidos
	\item Laboratorios de investigacion en paises en desarrollo
	\item Empresas pequeñas y medianas (PYMES)
	\item Estudiantes que desean practicar fuera del campus
\end{itemize}

Ademas, el modelo de ``caja negra'' de software comercial dificulta:
\begin{itemize}
	\item Comprender exactamente que calculos se realizan
	\item Modificar ecuaciones para casos especificos
	\item Integrar algoritmos propios
	\item Publicar metodologias completamente reproducibles
\end{itemize}

\subsection{Soluciones Open-Source Existentes}

Existen herramientas open-source como DWSIM \cite{dwsim2023}, COCO Simulator \cite{coco2023} y Cantera \cite{cantera2023} que ofrecen capacidades de simulacion sin costo. Sin embargo, presentan desafios:
\begin{itemize}
	\item Interfaces complejas que requieren experiencia previa
	\item Documentacion limitada en español
	\item Falta de integracion directa con datos experimentales
	\item Ausencia de metodologias educativas estructuradas
\end{itemize}

Para cientificos experimentales sin formacion en programacion, estas herramientas pueden resultar tan inaccesibles como el software comercial.

\subsection{Objetivos del Trabajo}

Este trabajo busca cerrar la brecha entre software comercial costoso y herramientas open-source de dificil acceso, mediante:

\begin{enumerate}
	\item Desarrollar un sistema modular en Python 100\% open-source para modelado cinetico de biodiesel
	\item Validar el modelo con datos experimentales de literatura
	\item Crear una metodologia educativa progresiva (12 practicas) desde Python basico hasta CFD
	\item Demostrar la versatilidad del sistema para diferentes reactores, catalizadores y condiciones
	\item Documentar completamente el sistema para garantizar reproducibilidad
	\item Proveer una alternativa economica, transparente y accesible a software comercial
\end{enumerate}

Este enfoque permite que estudiantes de quimica sin conocimientos de programacion puedan:
\begin{itemize}
	\item Comprender los fundamentos del modelado
	\item Procesar sus propios datos experimentales
	\item Ajustar modelos a sus sistemas especificos
	\item Optimizar condiciones de reaccion
	\item Escalar procesos de laboratorio a piloto
\end{itemize}

Todo el codigo fuente y documentacion estan disponibles publicamente, permitiendo a la comunidad cientifica auditar, modificar y extender el sistema segun sus necesidades.

% ============================================================================
% 2. METODOLOGÍA
% ============================================================================
\section{Metodologia}

\subsection{Arquitectura del Sistema}

El sistema se estructura en modulos independientes pero interconectados (Figura \ref{fig:arquitectura}), siguiendo principios de ingenieria de software:

\begin{figure}[htb!]
	\centering
	% PLACEHOLDER: Diagrama de flujo del sistema
	\fbox{\parbox{0.8\textwidth}{\centering
		\vspace{3cm}
		[FIGURA 1: Diagrama de arquitectura modular del sistema]\\
		\vspace{0.5cm}
		Modulos: data\_processing $\rightarrow$ models $\rightarrow$ optimization $\rightarrow$ visualization\\
		\vspace{3cm}
	}}
	\caption{Arquitectura modular del sistema. El flujo de datos es unidireccional desde datos experimentales hasta reportes finales, permitiendo ejecutar modulos de forma independiente o integrada.}
	\label{fig:arquitectura}
\end{figure}

Los modulos principales son:

\begin{itemize}
	\item \textbf{data\_processing}: Procesa datos de cromatografia de gases (GC-FID), calcula concentraciones usando estandar interno
	\item \textbf{models}: Implementa modelos cineticos (1 paso, 3 pasos reversibles), ecuacion de Arrhenius, sistema de EDOs
	\item \textbf{optimization}: Ajuste de parametros (lmfit, Levenberg-Marquardt) y optimizacion operacional (algoritmos globales)
	\item \textbf{visualization}: Genera graficas de alta calidad y reportes HTML interactivos
	\item \textbf{scaling}: Calcula criterios de escalado (P/V, Re, $v_{tip}$, $\theta_m$) y genera geometrias para CFD
\end{itemize}

\subsection{Modelo Cinetico}

\subsubsection{Reaccion de Transesterificacion}

La transesterificacion de trigliceridos con metanol en presencia de catalizador alcalino puede modelarse con diferentes niveles de complejidad:

\textbf{Modelo 1 paso (simplificado):}
\begin{equation}
	\text{TG} + 3\text{MeOH} \xrightarrow{k} 3\text{FAME} + \text{GL}
	\label{eq:reaccion_1paso}
\end{equation}

\textbf{Modelo 3 pasos reversibles (detallado):}
\begin{align}
	\text{TG} + \text{MeOH} &\xrightleftharpoons[k_{-1}]{k_1} \text{DG} + \text{FAME} \label{eq:paso1} \\
	\text{DG} + \text{MeOH} &\xrightleftharpoons[k_{-2}]{k_2} \text{MG} + \text{FAME} \label{eq:paso2} \\
	\text{MG} + \text{MeOH} &\xrightleftharpoons[k_{-3}]{k_3} \text{GL} + \text{FAME} \label{eq:paso3}
\end{align}

Donde TG=triglicerido, DG=diglicerido, MG=monoglicerido, FAME=ester metilico, GL=glicerol.

\subsubsection{Ecuacion de Arrhenius}

La constante de velocidad depende de la temperatura segun:
\begin{equation}
	k(T) = A \exp\left(-\frac{E_a}{RT}\right)
	\label{eq:arrhenius}
\end{equation}

Donde:
\begin{itemize}
	\item $k$ = constante de velocidad (L/mol·min)
	\item $A$ = factor pre-exponencial (L/mol·min)
	\item $E_a$ = energia de activacion (J/mol)
	\item $R$ = constante de gases ideales (8.314 J/mol·K)
	\item $T$ = temperatura absoluta (K)
\end{itemize}

\subsubsection{Sistema de Ecuaciones Diferenciales}

Para el modelo 1 paso, el sistema de EDOs es:
\begin{align}
	\frac{dC_{TG}}{dt} &= -k C_{TG} C_{MeOH}^3 \label{eq:edo_tg} \\
	\frac{dC_{MeOH}}{dt} &= -3k C_{TG} C_{MeOH}^3 \label{eq:edo_meoh} \\
	\frac{dC_{FAME}}{dt} &= 3k C_{TG} C_{MeOH}^3 \label{eq:edo_fame} \\
	\frac{dC_{GL}}{dt} &= k C_{TG} C_{MeOH}^3 \label{eq:edo_gl}
\end{align}

El sistema se resuelve numericamente usando \texttt{scipy.integrate.odeint} con condiciones iniciales especificadas por el usuario.

\subsection{Configuracion mediante JSON}

Todos los parametros del sistema se definen en archivos JSON estructurados (Figura \ref{fig:config_json}), evitando que usuarios sin experiencia modifiquen codigo:

\begin{figure}[htb!]
	\centering
	% PLACEHOLDER: Ejemplo de config.json
	\fbox{\parbox{0.9\textwidth}{\small
		\vspace{0.5cm}
		\texttt{\{\\
		\hspace{1em}"masas\_molares": \{\\
		\hspace{2em}"\_fuente": "PubChem Database",\\
		\hspace{2em}"TG\_tripalmitin": 807.3,\\
		\hspace{2em}"MeOH": 32.04\\
		\hspace{1em}\},\\
		\hspace{1em}"condiciones\_operacionales": \{\\
		\hspace{2em}"temperatura\_C": 65,\\
		\hspace{2em}"relacion\_molar\_MeOH\_TG": 9.0\\
		\hspace{1em}\},\\
		\hspace{1em}"perfil\_agitacion": \{\\
		\hspace{2em}"tipo": "lineal",\\
		\hspace{2em}"puntos": [\\
		\hspace{3em}\{"tiempo\_min": 0, "rpm": 300\},\\
		\hspace{3em}\{"tiempo\_min": 60, "rpm": 600\}\\
		\hspace{2em}]\\
		\hspace{1em}\}\\
		\}}
		\vspace{0.5cm}
	}}
	\caption{Ejemplo de archivo de configuracion JSON. Todos los parametros incluyen campos de documentacion (\_fuente, \_comentario) con referencias bibliograficas.}
	\label{fig:config_json}
\end{figure}

Esta arquitectura permite:
\begin{itemize}
	\item Cambiar parametros sin tocar codigo Python
	\item Documentar fuentes de cada valor
	\item Compartir configuraciones reproducibles
	\item Versionamiento facil con Git
\end{itemize}

\subsection{Procesamiento de Datos GC-FID}

La cromatografia de gases con detector de ionizacion de llama (GC-FID) es el metodo estandar para cuantificar especies en biodiesel. El flujo de procesamiento (Figura \ref{fig:flujo_gc}) incluye:

\begin{figure}[htb!]
	\centering
	% PLACEHOLDER: Flujo de procesamiento GC
	\fbox{\parbox{0.8\textwidth}{\centering
		\vspace{3cm}
		[FIGURA 3: Flujo de procesamiento de datos GC-FID]\\
		\vspace{0.5cm}
		CSV de areas $\rightarrow$ Factores de respuesta $\rightarrow$ Concentraciones $\rightarrow$ Conversiones\\
		\vspace{3cm}
	}}
	\caption{Flujo de procesamiento de datos experimentales desde archivos CSV hasta calculos de conversion. Se utiliza metil heptadecanoato como estandar interno.}
	\label{fig:flujo_gc}
\end{figure}

\begin{enumerate}
	\item Lectura de areas de picos desde CSV
	\item Calculo de factores de respuesta relativos respecto al estandar interno
	\item Conversion de areas a concentraciones molares
	\item Calculo de porcentaje de conversion
\end{enumerate}

El modulo \texttt{GCProcessor} automatiza este proceso, aceptando configuraciones personalizadas para diferentes sistemas cromatograficos.

\subsection{Ajuste de Parametros Cineticos}

Se implementan dos estrategias de optimizacion:

\textbf{Regresion no lineal local (lmfit):}
\begin{itemize}
	\item Algoritmo: Levenberg-Marquardt
	\item Calcula intervalos de confianza
	\item Analisis de residuos
	\item Rapido para datos de alta calidad
\end{itemize}

\textbf{Optimizacion global (scipy.optimize):}
\begin{itemize}
	\item Algoritmo: Differential Evolution
	\item Explora espacio de parametros completamente
	\item Robusto ante minimos locales
	\item Ideal para datos con ruido
\end{itemize}

La funcion objetivo minimiza el error cuadratico medio:
\begin{equation}
	RMSE = \sqrt{\frac{1}{N}\sum_{i=1}^{N}(y_i^{obs} - y_i^{pred})^2}
	\label{eq:rmse}
\end{equation}

\subsection{Diseno de Reactores}

El sistema modela dos escalas de reactor:

\textbf{Reactor de laboratorio:}
\begin{itemize}
	\item Volumen: 350 mL
	\item Agitacion: mosca magnetica (300-600 rpm)
	\item Operacion: batch isotermico
\end{itemize}

\textbf{Reactor piloto:}
\begin{itemize}
	\item Volumen: 20 L
	\item Agitacion: ribbon impeller (helice helicoidal)
	\item Serpentin: 10 espiras para intercambio termico
	\item Sin baffles (serpentin actua como rompedor de vortice)
\end{itemize}

Criterios de escalado implementados:
\begin{align}
	\text{Potencia por volumen: } & \frac{P}{V} = Np \rho N^3 D^5 / V \label{eq:pv} \\
	\text{Numero de Reynolds: } & Re = \frac{\rho N D^2}{\mu} \label{eq:reynolds} \\
	\text{Velocidad en punta: } & v_{tip} = \pi D N \label{eq:vtip} \\
	\text{Tiempo de mezclado: } & \theta_m = f(Re, geometria) \label{eq:theta}
\end{align}

Donde $Np$ es el numero de potencia, $\rho$ la densidad, $N$ la velocidad de agitacion, $D$ el diametro del impeller, $V$ el volumen y $\mu$ la viscosidad.

% ============================================================================
% 3. ESTRATEGIA EDUCATIVA: 12 PRÁCTICAS PROGRESIVAS
% ============================================================================
\section{Estrategia Educativa: 12 Practicas Progresivas}

\subsection{Filosofia Didactica: Observar-Experimentar-Diseñar}

La metodologia educativa se basa en tres fases progresivas:

\textbf{OBSERVAR (Practicas 1-3):} El estudiante ejecuta scripts completos y observa visualizaciones automaticas. No modifica codigo, solo experimenta cambiando parametros en archivos JSON. Responde preguntas conceptuales analizando graficas.

\textbf{EXPERIMENTAR (Practicas 4-6):} El estudiante compara multiples escenarios, modifica configuraciones y propone nuevos experimentos. Desarrolla intuicion sobre relaciones entre variables.

\textbf{DISEÑAR (Practicas 7-9):} El estudiante propone condiciones optimas, diseña experimentos completos y evalua trade-offs economicos y tecnicos.

\textbf{AVANZADO (Practicas 10-12):} Validacion con literatura, analisis de sensibilidad sistematico y personalizacion completa del modelo.

\subsection{Resumen de las 12 Practicas}

\begin{table}[htb!]
	\caption{Resumen de las 12 practicas progresivas}
	\label{tab:resumen_practicas}
	\centering
	\footnotesize
	\begin{tabular}{>{\centering\arraybackslash}p{0.6cm}>{\raggedright\arraybackslash}p{3.5cm}>{\raggedright\arraybackslash}p{3cm}>{\raggedright\arraybackslash}p{2.5cm}>{\centering\arraybackslash}p{1cm}}
		\toprule
		\textbf{No.} & \textbf{Tema} & \textbf{Conceptos Clave} & \textbf{Herramientas} & \textbf{Horas} \\
		\midrule
		1 & Python basico y estequiometria & Relacion molar, masa molar, balance de masa & numpy, matplotlib & 2 \\
		\midrule
		2 & Perfiles de temperatura & Ecuacion de Arrhenius, efecto de T en cinetica & Arrhenius plots & 3 \\
		\midrule
		3 & Pandas y procesamiento & DataFrames, series temporales, estadistica & pandas, CSV & 3 \\
		\midrule
		4 & EDOs y cinetica & Integracion numerica, modelo 3 pasos & scipy.odeint & 4 \\
		\midrule
		5 & GC-Processor & Cromatografia, estandar interno, agitacion & GCProcessor & 4 \\
		\midrule
		6 & Ajuste de parametros & Regresion no lineal, intervalos de confianza & lmfit & 5 \\
		\midrule
		7 & Optimizacion & Funcion objetivo, algoritmos globales & OperationalOptimizer & 5 \\
		\midrule
		8 & Workflow completo & Integracion de todos los modulos & Sistema completo & 6 \\
		\midrule
		9 & Up-scaling y CFD & Criterios de escalado, Ansys Fluent, UDF & Python + CFD & 8 \\
		\midrule
		10 & Validacion literatura & Reproducibilidad, comparacion estadistica & Kouzu 2008 data & 4 \\
		\midrule
		11 & Analisis sensibilidad & DOE, superficies 3D, Pareto & 4 parametros & 5 \\
		\midrule
		12 & Personalizacion & Modelos custom, catalizadores & Versatilidad & 6 \\
		\midrule
		& & & \textbf{TOTAL} & \textbf{55} \\
		\bottomrule
	\end{tabular}
\end{table}

\subsection{Ejemplos Especificos de Practicas}

\subsubsection{Practica 5: GC-Processor}

Esta practica introduce el procesamiento de datos experimentales reales. El estudiante:
\begin{enumerate}
	\item Ejecuta \texttt{python main.py}
	\item Observa la evolucion temporal de TG, DG, MG, FAME y GL (Figura \ref{fig:practica5})
	\item Analiza el efecto de perfiles de agitacion variables
	\item Compara 3 escenarios: baja, media y alta agitacion
\end{enumerate}

\begin{figure}[htb!]
	\centering
	% PLACEHOLDER: Resultados Práctica 5
	\fbox{\parbox{0.85\textwidth}{\centering
		\vspace{4cm}
		[FIGURA 4: Evolucion temporal de concentraciones - Practica 5]\\
		\vspace{0.5cm}
		Graficas: TG/DG/MG/FAME/GL vs tiempo para 3 perfiles de agitacion\\
		\vspace{4cm}
	}}
	\caption{Resultados de Practica 5: GC-Processor. Se observa el consumo progresivo de TG y la produccion de FAME. El perfil de agitacion variable afecta la velocidad de reaccion.}
	\label{fig:practica5}
\end{figure}

\subsubsection{Practica 7: Optimizacion}

El estudiante propone condiciones optimas para maximizar conversion en minimo tiempo. El sistema genera superficies de respuesta 3D (Figura \ref{fig:practica7}) y evalua automaticamente la propuesta.

\begin{figure}[htb!]
	\centering
	% PLACEHOLDER: Superficie de respuesta
	\fbox{\parbox{0.85\textwidth}{\centering
		\vspace{4cm}
		[FIGURA 5: Superficie de respuesta Temperatura vs Relacion Molar]\\
		\vspace{0.5cm}
		Contornos de conversion constante, region optima marcada\\
		\vspace{4cm}
	}}
	\caption{Practica 7: Superficie de respuesta para optimizacion. La region optima se identifica en T = 62-65°C y relacion molar 8-10:1.}
	\label{fig:practica7}
\end{figure}

\subsubsection{Practica 9: Up-scaling y CFD}

La practica mas avanzada del nivel basico. El estudiante:
\begin{enumerate}
	\item Calcula 4 criterios de escalado (P/V, Re, $v_{tip}$, $\theta_m$)
	\item Diseña reactor de 20 L con ribbon impeller
	\item Genera geometria 3D (Figura \ref{fig:practica9})
	\item Crea UDF en C para Ansys Fluent con cinetica ajustada
	\item Compara campos de velocidad y concentracion
\end{enumerate}

\begin{figure}[htb!]
	\centering
	% PLACEHOLDER: Geometría reactor 20L
	\fbox{\parbox{0.85\textwidth}{\centering
		\vspace{4cm}
		[FIGURA 6: Geometria reactor 20L con ribbon impeller + serpentin]\\
		\vspace{0.5cm}
		Vista 3D: ribbon impeller, serpentin helicoidal, sin baffles\\
		\vspace{4cm}
	}}
	\caption{Practica 9: Geometria del reactor piloto de 20 L. El ribbon impeller proporciona mezclado axial, el serpentin actua como intercambiador termico y rompedor de vortice.}
	\label{fig:practica9}
\end{figure}

\subsection{Estructura de Cada Practica}

Cada practica sigue un formato estandarizado:

\begin{itemize}
	\item \textbf{README.md}: Objetivos, conceptos clave, instrucciones paso a paso
	\item \textbf{main.py}: Script completo funcional (NO requiere modificacion)
	\item \textbf{config.json}: Parametros editables con documentacion
	\item \textbf{analisis.md}: Plantilla para respuestas del estudiante
	\item \textbf{datos/}: CSVs de ejemplo (cuando aplica)
	\item \textbf{resultados/}: Carpeta donde se guardan graficas generadas
\end{itemize}

El estudiante puede ejecutar cualquier practica inmediatamente:
\begin{lstlisting}[language=bash]
cd practicas/practica5_gc_processor
python main.py
\end{lstlisting}

Todas las graficas se generan automaticamente y se guardan en alta resolucion (300 DPI).

% ============================================================================
% 4. VALIDACIÓN DEL MODELO
% ============================================================================
\section{Validacion del Modelo}

\subsection{Comparacion con Datos de Kouzu et al. (2008)}

El modelo fue validado reproduciendo experimentos publicados por Kouzu et al. \cite{kouzu2008} sobre transesterificacion de aceite de soja con metanol catalizador NaOH. Las condiciones experimentales fueron:
\begin{itemize}
	\item Temperatura: 60°C
	\item Relacion molar MeOH:aceite = 6:1
	\item Catalizador NaOH: 1.0 wt\%
	\item Agitacion: 600 rpm constante
\end{itemize}

El ajuste de parametros cineticos usando \texttt{lmfit} produjo (Figura \ref{fig:validacion}):

\begin{figure}[htb!]
	\centering
	% PLACEHOLDER: Validación con Kouzu
	\fbox{\parbox{0.85\textwidth}{\centering
		\vspace{4cm}
		[FIGURA 7: Validacion con datos de Kouzu 2008]\\
		\vspace{0.5cm}
		Puntos experimentales vs curva del modelo ajustado\\
		Conversion vs tiempo (0-120 min)\\
		\vspace{4cm}
	}}
	\caption{Validacion del modelo con datos de Kouzu et al. (2008). Los puntos representan datos experimentales, la linea continua el modelo ajustado. R² = 0.987, RMSE = 2.3\%.}
	\label{fig:validacion}
\end{figure}

\subsection{Comparacion de Parametros Cineticos}

\begin{table}[htb!]
	\caption{Comparacion de parametros cineticos obtenidos}
	\label{tab:validacion_parametros}
	\centering
	\small
	\begin{tabular}{>{\raggedright\arraybackslash}p{3.5cm}>{\centering\arraybackslash}p{2.5cm}>{\centering\arraybackslash}p{2.5cm}>{\centering\arraybackslash}p{2.5cm}>{\centering\arraybackslash}p{1.5cm}}
		\toprule
		\textbf{Parametro} & \textbf{Este trabajo} & \textbf{Kouzu 2008} & \textbf{Literatura (rango)} & \textbf{Desv.} \\
		\midrule
		$A$ (L/mol·min) & $2.98 \times 10^{10}$ & $3.1 \times 10^{10}$ & $2.5-3.5 \times 10^{10}$ & 3.9\% \\
		\midrule
		$E_a$ (J/mol) & 51,900 & 50,200 & 49,000 - 53,000 & 3.4\% \\
		\midrule
		$R^2$ & 0.987 & 0.991 & -- & -- \\
		\midrule
		RMSE (\%) & 2.3 & 1.8 & -- & -- \\
		\bottomrule
	\end{tabular}
\end{table}

Las desviaciones menores al 5\% confirman que el modelo captura correctamente la cinetica de transesterificacion. Las diferencias se atribuyen a:
\begin{itemize}
	\item Variabilidad en composicion de aceite
	\item Diferencias en sistemas de agitacion
	\item Metodos de medicion distintos (GC-FID vs titulacion)
\end{itemize}

\subsection{Analisis de Residuos}

Los residuos (diferencia entre observado y predicho) muestran distribucion aleatoria sin patron sistematico, validando las suposiciones del modelo. El intervalo de confianza al 95\% para los parametros es inferior al 8\%, indicando buena precision en la estimacion.

% ============================================================================
% 5. ANÁLISIS DE SENSIBILIDAD
% ============================================================================
\section{Analisis de Sensibilidad}

Un analisis sistematico de sensibilidad identifica los parametros mas criticos para la conversion de biodiesel. Se evaluaron 4 variables operacionales:

\subsection{Temperatura (50-80°C)}

La temperatura es el parametro mas influyente (Figura \ref{fig:sensibilidad_temp}). Un aumento de 50°C a 65°C reduce el tiempo para alcanzar 95\% de conversion de 120 min a 30 min (factor 4x). Temperaturas superiores a 70°C ofrecen mejoras marginales y aumentan el riesgo de evaporacion de metanol.

\begin{figure}[htb!]
	\centering
	% PLACEHOLDER: Análisis de temperatura
	\fbox{\parbox{0.85\textwidth}{\centering
		\vspace{4cm}
		[FIGURA 8: Familia de curvas de conversion vs tiempo para 5 temperaturas]\\
		\vspace{0.5cm}
		50°C, 55°C, 60°C, 65°C, 70°C\\
		\vspace{4cm}
	}}
	\caption{Efecto de la temperatura en la cinetica de transesterificacion. La conversion aumenta exponencialmente con T segun la ecuacion de Arrhenius.}
	\label{fig:sensibilidad_temp}
\end{figure}

\subsection{Relacion Molar MeOH:TG (3:1 a 12:1)}

La relacion estequiometrica minima es 3:1, pero en la practica se requiere exceso para desplazar el equilibrio. La relacion 9:1 proporciona 95\% de conversion, mientras que 12:1 solo mejora a 97\% con costo adicional significativo de metanol. La relacion optima economica es 8-9:1.

\subsection{Concentracion de Catalizador (0.5-2.0 wt\%)}

Aumentar catalizador de 0.5\% a 1.0\% duplica la velocidad inicial. Sin embargo, concentraciones superiores a 1.5\% causan formacion de jabones (saponificacion) que dificultan la separacion de productos. El optimo es 1.0-1.2 wt\%.

\subsection{Agitacion (300-800 rpm)}

Para reactores pequeños (< 1 L), agitacion superior a 400 rpm elimina limitaciones de transferencia de masa. En reactores grandes, la agitacion debe optimizarse segun criterios de escalado para evitar consumo excesivo de potencia.

\subsection{Diagrama de Pareto}

Un diagrama de Pareto (Figura \ref{fig:pareto}) cuantifica la importancia relativa de cada parametro:

\begin{figure}[htb!]
	\centering
	% PLACEHOLDER: Diagrama de Pareto
	\fbox{\parbox{0.7\textwidth}{\centering
		\vspace{3cm}
		[FIGURA 9: Diagrama de Pareto - Sensibilidad relativa]\\
		\vspace{0.5cm}
		Temperatura (45\%) > Relacion molar (30\%) > Catalizador (18\%) > Agitacion (7\%)\\
		\vspace{3cm}
	}}
	\caption{Diagrama de Pareto mostrando la contribucion relativa de cada parametro a la varianza total en conversion. La temperatura domina con 45\% de influencia.}
	\label{fig:pareto}
\end{figure}

\textbf{Conclusion:} Optimizar temperatura y relacion molar proporciona el mayor impacto en rendimiento (75\% de influencia combinada).

% ============================================================================
% 6. CASOS DE ESTUDIO
% ============================================================================
\section{Casos de Estudio}

\subsection{Caso A: Optimizacion para Laboratorio}

\textbf{Objetivo:} Maximizar conversion en minimo tiempo con restricciones de seguridad.

\textbf{Restricciones:}
\begin{itemize}
	\item Temperatura maxima: 70°C (limite de seguridad)
	\item Relacion molar maxima: 10:1 (limite economico)
	\item Volumen reactor: 350 mL
	\item Catalizador: 1.0 wt\% NaOH
\end{itemize}

\textbf{Resultado del algoritmo de optimizacion:}
\begin{itemize}
	\item Temperatura optima: 65°C
	\item Relacion molar optima: 9:1
	\item Tiempo para 95\% conversion: 35 minutos
	\item Consumo de metanol: 11.2 g por 50 g aceite
\end{itemize}

\subsection{Caso B: Escalado a Planta Piloto}

\textbf{Objetivo:} Escalar proceso de 350 mL a 20 L manteniendo conversion equivalente.

\textbf{Criterio de escalado seleccionado:} Potencia por volumen (P/V) constante.

\textbf{Calculos:}
\begin{itemize}
	\item P/V laboratorio: 50 W/L (estimado para mosca magnetica a 450 rpm)
	\item Volumen piloto: 20 L
	\item Potencia requerida: $P = 50 \times 20 = 1000$ W = 1 kW
	\item Impeller: ribbon, diametro 240 mm
	\item Velocidad calculada: 180 rpm
\end{itemize}

\textbf{Validacion CFD:} Simulaciones en Ansys Fluent confirman tiempo de mezclado < 30 s y distribucion uniforme de temperatura ($\pm 2$°C).

\subsection{Caso C: Aceite con Alta Acidez}

Aceites usados frecuentemente contienen acidos grasos libres (FFA) que causan saponificacion. Para FFA > 2\%, se requiere pre-tratamiento por esterificacion acida:

\textbf{Modificacion del modelo:}
\begin{enumerate}
	\item Etapa 1: Esterificacion con H2SO4 (60°C, 1 h)
	\item Etapa 2: Transesterificacion alcalina (condiciones optimizadas)
\end{enumerate}

El sistema permite configurar ambas etapas en \texttt{config.json}, calculando consumos de reactivos y tiempos totales.

% ============================================================================
% 7. TUTORIAL DE INSTALACIÓN Y USO
% ============================================================================
\section{Tutorial de Instalacion y Uso en Windows}

Esta seccion guia al usuario desde la instalacion hasta la ejecucion de la primera practica.

\subsection{Paso 1: Instalar Miniconda}

\textbf{Miniconda} es una distribucion minima de Python que incluye el gestor de paquetes \texttt{conda}.

\begin{enumerate}
	\item Descargar instalador desde: \url{https://docs.conda.io/en/latest/miniconda.html}
	\item Seleccionar: ``Miniconda3 Windows 64-bit''
	\item Ejecutar instalador: \texttt{Miniconda3-latest-Windows-x86\_64.exe}
	\item Durante instalacion:
	\begin{itemize}
		\item Marcar: ``Add Miniconda3 to PATH'' (recomendado)
		\item Instalar para: ``Just Me''
	\end{itemize}
	\item Verificar instalacion abriendo ``Anaconda Prompt'' y ejecutar:
\end{enumerate}

\begin{lstlisting}[language=bash]
conda --version
# Debe mostrar: conda 23.x.x
\end{lstlisting}

\subsection{Paso 2: Crear Entorno Virtual}

Los entornos virtuales aislamos proyectos con diferentes dependencias:

\begin{lstlisting}[language=bash]
conda create -n biodiesel python=3.10
conda activate biodiesel
\end{lstlisting}

El prompt cambiara a: \texttt{(biodiesel) C:\textbackslash Users\textbackslash...>}

\subsection{Paso 3: Instalar Dependencias}

El archivo \texttt{requirements.txt} lista todas las librerias necesarias:

\begin{lstlisting}[language=bash]
pip install -r requirements.txt
\end{lstlisting}

Esto instalara: numpy, scipy, pandas, matplotlib, plotly, lmfit, openpyxl.

\subsection{Paso 4: Clonar el Repositorio}

Si tienes Git instalado:

\begin{lstlisting}[language=bash]
git clone https://github.com/usuario/mod_esterificacion.git
cd mod_esterificacion
\end{lstlisting}

Alternativamente, descargar ZIP desde GitHub y descomprimir.

\subsection{Paso 5: Explorar Estructura de Directorios}

\begin{lstlisting}[language=bash]
dir
# Debe mostrar:
# src/               <- Codigo fuente modular
# practicas/         <- 12 practicas progresivas
# datos/             <- Datos experimentales ejemplo
# config/            <- Archivos JSON de configuracion
# docs/              <- Documentacion adicional
# main.py            <- Script principal del sistema
# requirements.txt   <- Dependencias
# README.md          <- Guia principal
\end{lstlisting}

\subsection{Paso 6: Ejecutar Primera Practica}

\begin{lstlisting}[language=bash]
cd practicas\practica1_python_basico
python main.py
\end{lstlisting}

Esto generara:
\begin{itemize}
	\item 4 ventanas con graficas interactivas
	\item Carpeta \texttt{resultados/} con PNG de alta resolucion
	\item Resumen en consola con tabla de valores
\end{itemize}

\subsection{Paso 7: Modificar Configuracion}

Abrir \texttt{config.json} con Visual Studio Code (recomendado) o cualquier editor de texto:

\begin{lstlisting}[language=bash]
code config.json
\end{lstlisting}

Modificar valores (por ejemplo, cambiar relacion molar de 9:1 a 12:1), guardar y volver a ejecutar \texttt{python main.py}.

\subsection{Paso 8: Ubicacion de Datos Experimentales}

Para usar datos propios:
\begin{itemize}
	\item Datos GC-FID: guardar CSV en \texttt{datos/experimentales/}
	\item Formato requerido: columnas [tiempo\_min, area\_TG, area\_MeOH, area\_FAME, area\_GL, area\_std]
	\item Modificar \texttt{config/config\_gc.json} con rutas y factores de respuesta
\end{itemize}

\subsection{Paso 9: Propiedades Fisicoquimicas}

Todas las propiedades se encuentran en:
\begin{itemize}
	\item \texttt{config/propiedades\_fisicoquimicas.json}
	\item Incluye: masas molares, densidades, viscosidades, calores especificos
	\item Todas con campo \texttt{\_fuente} documentando origen (PubChem, Perry's, etc.)
\end{itemize}

Para agregar un nuevo triglicerido:
\begin{lstlisting}[language=json]
"TG_trioleina": {
  "masa_molar_g_mol": 885.4,
  "densidad_25C_g_mL": 0.899,
  "_fuente": "PubChem CID: 5497163"
}
\end{lstlisting}

\subsection{Paso 10: Uso con Visual Studio Code}

\textbf{Recomendado para mejor experiencia:}
\begin{enumerate}
	\item Descargar VS Code: \url{https://code.visualstudio.com/}
	\item Instalar extensiones:
	\begin{itemize}
		\item Python (Microsoft)
		\item Pylance (Microsoft)
		\item JSON (validacion de sintaxis)
	\end{itemize}
	\item Abrir carpeta del proyecto: \texttt{File > Open Folder > mod\_esterificacion}
	\item Terminal integrada: \texttt{Ctrl + `} (acento grave)
	\item Ejecutar scripts directamente desde VS Code
\end{enumerate}

\subsection{Resolucion de Problemas Comunes}

\textbf{Error: ``ModuleNotFoundError''}
\begin{lstlisting}[language=bash]
# Verificar que el entorno este activado:
conda activate biodiesel
# Reinstalar dependencias:
pip install -r requirements.txt
\end{lstlisting}

\textbf{Error: ``FileNotFoundError''}
\begin{itemize}
	\item Verificar que estas en el directorio correcto: \texttt{cd practicas\textbackslash practicaX}
	\item Rutas en Windows usan \textbackslash\ (backslash), no / (forward slash)
\end{itemize}

\textbf{Graficas no se muestran}
\begin{lstlisting}[language=bash]
# Instalar backend de matplotlib:
pip install pyqt5
\end{lstlisting}

% ============================================================================
% 8. DISCUSIÓN
% ============================================================================
\section{Discusion}

\subsection{Ventajas del Enfoque Open-Source}

\subsubsection{Transparencia Total}

A diferencia del software comercial tipo ``caja negra'', el codigo Python permite:
\begin{itemize}
	\item Auditar cada calculo realizado
	\item Comprender exactamente que modelo se esta usando
	\item Identificar limitaciones y suposiciones
	\item Verificar correctitud de algoritmos
	\item Publicar metodologias completamente reproducibles
\end{itemize}

Esta transparencia es esencial para investigacion cientifica rigurosa.

\subsubsection{Costo Cero y Accesibilidad}

El ahorro economico es significativo:
\begin{itemize}
	\item Licencia ASPEN Plus: \$15,000-50,000 USD/año
	\item Sistema Python: \$0 (solo requiere computadora estandar)
	\item Escalabilidad: ilimitado numero de usuarios sin costo adicional
\end{itemize}

Esto democratiza el acceso a tecnologias de modelado para:
\begin{itemize}
	\item Universidades en paises en desarrollo
	\item Laboratorios con presupuestos limitados
	\item Estudiantes que desean practicar en casa
	\item PYMEs que no pueden costear software comercial
\end{itemize}

\subsubsection{Personalizacion Ilimitada}

Los usuarios pueden:
\begin{itemize}
	\item Modificar ecuaciones para sistemas especificos
	\item Integrar nuevos modelos termodinamicos
	\item Conectar con bases de datos propias
	\item Automatizar workflows personalizados
	\item Crear interfaces graficas custom
\end{itemize}

Software comercial raramente permite este nivel de flexibilidad.

\subsection{Versatilidad del Sistema}

\subsubsection{Diferentes Materias Primas}

El sistema ha sido probado con:
\begin{itemize}
	\item Aceite de cocina usado
	\item Aceite de palma virgen
	\item Aceite de soja
	\item Grasa animal
\end{itemize}

Solo requiere ajustar composicion de trigliceridos en \texttt{config.json}.

\subsubsection{Diferentes Catalizadores}

Soporta tres tipos de catalizadores (Practica 12):
\begin{itemize}
	\item \textbf{Homogeneos:} NaOH, KOH (modelo cinetico actual)
	\item \textbf{Heterogeneos:} CaO, MgO (requiere termino de adsorcion)
	\item \textbf{Enzimaticos:} Lipasas (modelo Michaelis-Menten)
\end{itemize}

\subsubsection{Diferentes Reactores}

La arquitectura modular permite modelar:
\begin{itemize}
	\item Reactores batch (implementado)
	\item Semi-batch con alimentacion continua
	\item CSTR (continuo tanque agitado)
	\item PFR (flujo piston)
	\item Microreactores
\end{itemize}

\subsection{Impacto Educativo}

\subsubsection{Reduccion de Barrera de Entrada}

La metodologia progresiva (12 practicas) permite que estudiantes sin programacion:
\begin{itemize}
	\item Ejecuten modelos complejos desde el dia 1
	\item Desarrollen intuicion visual mediante graficas
	\item Aprendan Python gradualmente sin frustration
	\item Integren conocimientos de quimica con computacion
\end{itemize}

\subsubsection{Desarrollo de Habilidades del Siglo XXI}

Los estudiantes adquieren:
\begin{itemize}
	\item Programacion en Python (lenguaje mas demandado)
	\item Analisis de datos con pandas
	\item Visualizacion cientifica
	\item Control de versiones con Git
	\item Pensamiento computacional
\end{itemize}

Estas habilidades son altamente valoradas en industria y academia.

\subsubsection{Preparacion para Investigacion Moderna}

La ciencia actual requiere:
\begin{itemize}
	\item Analisis de grandes volumenes de datos
	\item Integracion de modelos multi-escala
	\item Automatizacion de workflows
	\item Reproducibilidad computacional
\end{itemize}

Este sistema prepara a estudiantes para estos desafios.

\subsection{Limitaciones Actuales}

\subsubsection{Modelo Cinetico Simplificado}

El modelo actual no incluye:
\begin{itemize}
	\item Efectos de emulsion y separacion de fases
	\item Transferencia de masa interfacial detallada
	\item Saponificacion competitiva
	\item Desactivacion de catalizador
\end{itemize}

Trabajo futuro: Implementar modelos multi-fase mas rigurosos.

\subsubsection{Validacion Experimental Limitada}

Validacion actual basada en:
\begin{itemize}
	\item Datos publicados (Kouzu 2008)
	\item Rango limitado de condiciones
\end{itemize}

Se requiere validacion mas amplia con:
\begin{itemize}
	\item Diferentes aceites y catalizadores
	\item Reactores de diferentes escalas
	\item Condiciones extremas (alta T, alta acidez)
\end{itemize}

\subsubsection{CFD Requiere Software Adicional}

La Practica 9 (CFD) requiere Ansys Fluent:
\begin{itemize}
	\item Licencia academica disponible gratuitamente
	\item Pero instalacion y configuracion complejas
	\item Alternativa futura: OpenFOAM (open-source)
\end{itemize}

\subsection{Trabajo Futuro}

\subsubsection{Expansion a Otras Reacciones}

El framework es extensible a:
\begin{itemize}
	\item Esterificacion de acidos grasos libres
	\item Hidrogenacion de aceites
	\item Pirolisis de biomasa
	\item Gasificacion
\end{itemize}

\subsubsection{Interfaz Grafica (GUI)}

Desarrollo de GUI con PyQt o Streamlit para:
\begin{itemize}
	\item Usuarios no tecnicos
	\item Demostraciones interactivas
	\item Acceso via navegador web
\end{itemize}

\subsubsection{Integracion con Bases de Datos Online}

Conectar con:
\begin{itemize}
	\item NIST Chemistry WebBook (propiedades)
	\item PubChem (estructuras moleculares)
	\item DIPPR (datos termodinamicos)
\end{itemize}

\subsubsection{Modelo de Costos Economicos}

Agregar modulo para calcular:
\begin{itemize}
	\item Costo de reactivos
	\item Consumo energetico
	\item Costos de separacion y purificacion
	\item Analisis de rentabilidad
\end{itemize}

% ============================================================================
% 9. CONCLUSIONES
% ============================================================================
\section{Conclusiones}

Este trabajo demuestra que herramientas open-source en Python pueden ofrecer una alternativa viable, economica y pedagogicamente superior a software comercial para modelado de procesos quimicos. Las conclusiones principales son:

\begin{enumerate}
	\item Se desarrollo un sistema modular completo en Python para modelado cinetico de produccion de biodiesel, integrando procesamiento de datos experimentales, ajuste de parametros, optimizacion y escalado.

	\item El modelo fue validado exitosamente con datos de literatura (Kouzu et al. 2008), mostrando desviaciones menores al 5\% en parametros cineticos ($A$ y $E_a$) y $R^2 > 0.98$ en ajuste de curvas de conversion.

	\item Un analisis de sensibilidad sistematico identifica la temperatura como el parametro mas critico (45\% de influencia), seguido por relacion molar (30\%), concentracion de catalizador (18\%) y agitacion (7\%).

	\item Se creo una metodologia educativa innovadora basada en 12 practicas progresivas (55 horas totales) que guian a estudiantes sin conocimientos de programacion desde Python basico hasta modelado avanzado y CFD, siguiendo la filosofia ``Observar-Experimentar-Diseñar''.

	\item El sistema es completamente configurable mediante archivos JSON, evitando que usuarios sin experiencia modifiquen codigo pero permitiendo flexibilidad total en parametros operacionales, propiedades fisicoquimicas y perfiles de agitacion.

	\item Se demostro la versatilidad del sistema para adaptarse a diferentes aceites (usado, palma, soja), catalizadores (NaOH, CaO, lipasas) y escalas de reactor (350 mL laboratorio a 20 L piloto), modificando solo archivos de configuracion.

	\item El criterio de escalado basado en potencia por volumen (P/V) constante permite diseñar reactores piloto manteniendo tiempos de conversion equivalentes a escala laboratorio, validado mediante simulaciones CFD en Ansys Fluent.

	\item La documentacion completa incluye tutorial de instalacion para Windows, guia de uso paso a paso, y ubicacion explicita de archivos de configuracion, datos experimentales y propiedades fisicoquimicas.

	\item El enfoque open-source ofrece transparencia total (codigo fuente auditable), costo cero (vs \$15,000-50,000/año de ASPEN Plus), y personalizacion ilimitada, democratizando el acceso a tecnologias de modelado.

	\item El sistema prepara a estudiantes para investigacion moderna desarrollando habilidades en programacion Python, analisis de datos, visualizacion cientifica y pensamiento computacional, competencias esenciales en el siglo XXI.
\end{enumerate}

\textbf{Mensaje final:} Este trabajo contribuye a la ciencia abierta proporcionando una herramienta completa, documentada y validada que la comunidad cientifica puede usar, auditar, modificar y extender libremente. El codigo fuente y todas las practicas estan disponibles publicamente, fomentando la colaboracion y acelerando el progreso cientifico en tecnologias de energia renovable.

% ============================================================================
% CORREOS ELECTRÓNICOS
% ============================================================================
\printauthoremails

% ============================================================================
% BIBLIOGRAFÍA
% ============================================================================
\pretocmd{\printbibliography}{\vspace{1em}}{}{}
\printbibliography

\end{document}
