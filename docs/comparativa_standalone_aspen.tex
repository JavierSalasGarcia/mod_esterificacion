\documentclass[12pt,a4paper]{article}

% Paquetes
\usepackage[utf8]{inputenc}
\usepackage[spanish]{babel}
\usepackage{geometry}
\geometry{left=2.5cm,right=2.5cm,top=3cm,bottom=3cm}
\usepackage{graphicx}
\usepackage{amsmath}
\usepackage{amssymb}
\usepackage{float}
\usepackage{xcolor}
\usepackage{hyperref}
\usepackage{fancyhdr}
\usepackage{tcolorbox}
\usepackage{booktabs}
\usepackage{longtable}
\usepackage{multirow}
\usepackage{enumitem}

% Configuración de tcolorbox
\tcbuselibrary{skins,breakable}

\newtcolorbox{probox}{
    colback=green!5!white,
    colframe=green!75!black,
    fonttitle=\bfseries,
    title=✓ Ventajas,
    breakable
}

\newtcolorbox{conbox}{
    colback=red!5!white,
    colframe=red!75!black,
    fonttitle=\bfseries,
    title=✗ Desventajas,
    breakable
}

\newtcolorbox{infobox}[1]{
    colback=blue!5!white,
    colframe=blue!75!black,
    fonttitle=\bfseries,
    title=#1,
    breakable
}

% Headers
\pagestyle{fancy}
\fancyhf{}
\fancyhead[L]{Comparativa: Modelo Standalone vs ASPEN HYSYS}
\fancyhead[R]{\thepage}
\renewcommand{\headrulewidth}{0.4pt}

% Título
\title{
    \vspace{-2cm}
    \textbf{Análisis Comparativo}\\
    \Large{Modelo Standalone en Python vs ASPEN HYSYS}\\
    \large{para Modelado de Transesterificación de Biodiésel}\\
    \vspace{1cm}
}
\author{Sistema de Modelado Cinético\\Versión 1.0}
\date{\today}

\begin{document}

\maketitle
\thispagestyle{empty}

\begin{abstract}
Este documento presenta un análisis comparativo exhaustivo entre el modelo standalone desarrollado en Python y el simulador comercial ASPEN HYSYS para el modelado de la reacción de transesterificación catalizada por CaO. Se evalúan ventajas, desventajas, alcances, limitaciones y casos de uso apropiados para cada enfoque. El objetivo es proporcionar criterios claros para la selección de la herramienta más adecuada según las necesidades específicas del usuario.
\end{abstract}

\newpage
\tableofcontents
\newpage

%==============================================================================
\section{Introducción}
%==============================================================================

\subsection{Contexto}

El modelado de procesos químicos puede abordarse mediante dos enfoques principales:

\begin{enumerate}
    \item \textbf{Simuladores comerciales}: Software especializado (ASPEN HYSYS, Aspen Plus, ProII, ChemCAD)
    \item \textbf{Modelos standalone}: Código personalizado en lenguajes de programación (Python, MATLAB, Julia)
\end{enumerate}

Este documento compara estos enfoques específicamente para el caso de la transesterificación de biodiésel.

\subsection{Alcance del Análisis}

La comparación se centra en:

\begin{itemize}
    \item Capacidades de modelado cinético
    \item Flexibilidad y personalización
    \item Precisión y validación
    \item Facilidad de uso
    \item Costos (licencias, tiempo de desarrollo)
    \item Reproducibilidad científica
    \item Integración con otras herramientas
    \item Aplicabilidad en investigación vs industria
\end{itemize}

%==============================================================================
\section{Descripción de los Enfoques}
%==============================================================================

\subsection{Modelo Standalone en Python}

\subsubsection{Características Principales}

\begin{itemize}
    \item \textbf{Lenguaje}: Python 3.8+
    \item \textbf{Bibliotecas principales}:
    \begin{itemize}
        \item NumPy, SciPy (cálculos numéricos)
        \item lmfit (ajuste de parámetros)
        \item Matplotlib, Plotly (visualización)
        \item Pandas (manipulación de datos)
    \end{itemize}
    \item \textbf{Arquitectura}: Modular (13 módulos independientes)
    \item \textbf{Líneas de código}: $>$ 6500
    \item \textbf{Licencia}: Open-source (MIT)
\end{itemize}

\subsubsection{Funcionalidades Implementadas}

\begin{enumerate}
    \item Procesamiento de datos GC-FID
    \item Modelos cinéticos (1 paso y 3 pasos)
    \item Ajuste de parámetros (A, Ea)
    \item Optimización de condiciones (T, RPM, catalizador)
    \item Integración con ASPEN HYSYS vía COM API
    \item Comparación estadística de modelos
    \item Generación de reportes y gráficas
\end{enumerate}

\subsection{ASPEN HYSYS}

\subsubsection{Características Principales}

\begin{itemize}
    \item \textbf{Tipo}: Simulador de procesos comercial
    \item \textbf{Desarrollador}: Aspen Technology, Inc.
    \item \textbf{Enfoque}: Simulación de estado estacionario y dinámico
    \item \textbf{Base de datos}: $>$ 10,000 componentes
    \item \textbf{Modelos termodinámicos}: $>$ 50 paquetes (UNIFAC, NRTL, PR, SRK, etc.)
    \item \textbf{Licencia}: Comercial (costo elevado)
\end{itemize}

\subsubsection{Funcionalidades para Biodiésel}

\begin{enumerate}
    \item Reactores: CSTR, PFR, Batch
    \item Cinética: Arrhenius, Langmuir-Hinshelwood
    \item Separación: Destilación, extracción
    \item Balance de energía riguroso
    \item Optimización integrada
    \item Análisis económico
\end{enumerate}

%==============================================================================
\section{Comparación Detallada}
%==============================================================================

\subsection{Tabla Comparativa General}

\begin{longtable}{p{4cm}p{5cm}p{5cm}}
\caption{Comparación general: Standalone vs ASPEN HYSYS} \\
\toprule
\textbf{Criterio} & \textbf{Standalone (Python)} & \textbf{ASPEN HYSYS} \\
\midrule
\endfirsthead
\multicolumn{3}{c}{\tablename\ \thetable\ -- continuación} \\
\toprule
\textbf{Criterio} & \textbf{Standalone (Python)} & \textbf{ASPEN HYSYS} \\
\midrule
\endhead
\midrule
\multicolumn{3}{r}{Continúa en la siguiente página} \\
\endfoot
\bottomrule
\endlastfoot

\textbf{Costo} & Gratuito (open-source) & \$20,000-\$50,000/año \\
\textbf{Plataforma} & Windows/Linux/Mac & Solo Windows \\
\textbf{Curva de aprendizaje} & Moderada (Python) & Alta (interfaz compleja) \\
\textbf{Flexibilidad} & Total (código abierto) & Limitada (GUI fija) \\
\textbf{Personalización} & Completa & Mediante VBA/UDF \\
\textbf{Reproducibilidad} & Excelente (código versionable) & Limitada (archivos binarios) \\
\textbf{Velocidad desarrollo} & Lenta inicialmente & Rápida (GUI) \\
\textbf{Velocidad ejecución} & Rápida (Python optimizado) & Variable \\
\textbf{Documentación} & Código + comentarios & Manuales extensos \\
\textbf{Soporte} & Comunidad open-source & Soporte comercial \\
\textbf{Integración datos} & Directa (CSV, JSON, Excel) & Import/Export \\
\textbf{Validación} & Manual (vs experimental) & Automática (flash, equilibrio) \\
\textbf{Termodinámica} & Limitada (manual) & Rigurosa (50+ modelos) \\
\textbf{Cinética química} & Completamente flexible & Predefinida + UDF \\
\textbf{Visualización} & Alta calidad (Matplotlib) & Básica (gráficos HYSYS) \\
\textbf{Escalabilidad} & Excelente (paralelización) & Buena \\
\textbf{Uso académico} & Ideal (aprendizaje) & Limitado (costo) \\
\textbf{Uso industrial} & Limitado (validación) & Estándar (certificado) \\
\textbf{Control de versiones} & Sí (Git) & No (archivos .hsc) \\
\textbf{Automatización} & Total (scripts Python) & Parcial (COM API) \\
\end{longtable}

\subsection{Ventajas y Desventajas}

\subsubsection{Modelo Standalone en Python}

\begin{probox}
\textbf{Ventajas Principales}:

\begin{enumerate}[leftmargin=*]
    \item \textbf{Costo cero}: No requiere licencias comerciales

    \item \textbf{Transparencia total}: Código fuente accesible, ecuaciones visibles

    \item \textbf{Flexibilidad máxima}:
    \begin{itemize}
        \item Cambiar ecuaciones cinéticas fácilmente
        \item Implementar modelos no estándar
        \item Agregar nuevas funcionalidades
    \end{itemize}

    \item \textbf{Reproducibilidad científica}:
    \begin{itemize}
        \item Código versionable con Git
        \item Resultados exactamente reproducibles
        \item Compartible con otros investigadores
    \end{itemize}

    \item \textbf{Integración con herramientas científicas}:
    \begin{itemize}
        \item Jupyter notebooks para análisis interactivo
        \item Machine learning (scikit-learn, TensorFlow)
        \item Optimización avanzada (scipy.optimize)
    \end{itemize}

    \item \textbf{Portabilidad}: Funciona en Windows, Linux, Mac

    \item \textbf{Automatización completa}: Scripts para batch processing

    \item \textbf{Gráficas de alta calidad}: Matplotlib, Plotly, Seaborn

    \item \textbf{Aprendizaje profundo}: Entender ecuaciones subyacentes

    \item \textbf{Personalización total}: Desde procesamiento de datos hasta reportes
\end{enumerate}
\end{probox}

\begin{conbox}
\textbf{Desventajas Principales}:

\begin{enumerate}[leftmargin=*]
    \item \textbf{Tiempo de desarrollo inicial}: Requiere programar desde cero

    \item \textbf{Termodinámica limitada}:
    \begin{itemize}
        \item No incluye modelos avanzados (UNIFAC completo, NRTL)
        \item Propiedades deben ingresarse manualmente o calcularse
        \item Flash calculations no implementadas
    \end{itemize}

    \item \textbf{Validación manual}: No hay verificación automática de balances

    \item \textbf{Equipos de proceso}: No incluye modelos de equipos (bombas, intercambiadores)

    \item \textbf{Curva de aprendizaje}: Requiere conocimiento de programación

    \item \textbf{Soporte limitado}: Depende de comunidad open-source

    \item \textbf{No certificado}: Para uso industrial requiere validación exhaustiva

    \item \textbf{Economía de procesos}: No incluye análisis económico integrado

    \item \textbf{Diseño de flowsheets}: No hay GUI para diseño de diagramas de flujo

    \item \textbf{Mantenimiento}: Responsabilidad del usuario
\end{enumerate}
\end{conbox}

\subsubsection{ASPEN HYSYS}

\begin{probox}
\textbf{Ventajas Principales}:

\begin{enumerate}[leftmargin=*]
    \item \textbf{Termodinámica rigurosa}:
    \begin{itemize}
        \item Base de datos de 10,000+ componentes
        \item 50+ modelos termodinámicos validados
        \item Flash calculations automáticas
    \end{itemize}

    \item \textbf{Biblioteca de equipos extensa}:
    \begin{itemize}
        \item Reactores (CSTR, PFR, Batch)
        \item Separadores (destilación, extracción)
        \item Intercambiadores de calor
        \item Bombas, compresores, válvulas
    \end{itemize}

    \item \textbf{GUI intuitiva}: Diseño visual de flowsheets

    \item \textbf{Validación automática}: Balances de masa y energía verificados

    \item \textbf{Optimización integrada}: Herramientas de optimización builtin

    \item \textbf{Análisis económico}: Evaluación de costos de capital y operación

    \item \textbf{Estándar industrial}: Ampliamente usado en la industria

    \item \textbf{Soporte comercial}: Documentación extensa y soporte técnico

    \item \textbf{Certificación}: Resultados aceptados para diseño industrial

    \item \textbf{Simulación dinámica}: Modelado transitorio de procesos
\end{enumerate}
\end{probox}

\begin{conbox}
\textbf{Desventajas Principales}:

\begin{enumerate}[leftmargin=*]
    \item \textbf{Costo prohibitivo}: \$20,000-\$50,000/año por licencia

    \item \textbf{Solo Windows}: No funciona en Linux/Mac

    \item \textbf{Caja negra}: Ecuaciones internas no transparentes

    \item \textbf{Flexibilidad limitada}:
    \begin{itemize}
        \item Difícil implementar modelos no estándar
        \item UDFs requieren compilación
        \item No acceso a código fuente
    \end{itemize}

    \item \textbf{Reproducibilidad problemática}:
    \begin{itemize}
        \item Archivos binarios (.hsc) no versionables
        \item Dependencia de versión específica de HYSYS
        \item Difícil compartir con otros investigadores
    \end{itemize}

    \item \textbf{Curva de aprendizaje alta}: Interfaz compleja

    \item \textbf{Integración limitada}:
    \begin{itemize}
        \item COM API compleja y propensa a errores
        \item Difícil automatizar
        \item No integra con herramientas modernas (Git, Jupyter)
    \end{itemize}

    \item \textbf{Actualización forzada}: Nuevas versiones pueden romper compatibilidad

    \item \textbf{Visualización básica}: Gráficos limitados y poco personalizables

    \item \textbf{No open-source}: Dependencia del vendor (Aspen Tech)
\end{enumerate}
\end{conbox}

%==============================================================================
\section{Alcances y Limitaciones}
%==============================================================================

\subsection{Alcances del Modelo Standalone}

\begin{infobox}{¿Para qué es adecuado el modelo standalone?}

\textbf{Casos de uso ideales}:

\begin{enumerate}
    \item \textbf{Investigación académica}:
    \begin{itemize}
        \item Desarrollo de nuevos modelos cinéticos
        \item Prueba de hipótesis
        \item Publicaciones científicas (código reproducible)
        \item Tesis de maestría/doctorado
    \end{itemize}

    \item \textbf{Optimización paramétrica}:
    \begin{itemize}
        \item Ajuste de parámetros cinéticos
        \item Búsqueda de condiciones óptimas
        \item Análisis de sensibilidad
        \item Diseño de experimentos
    \end{itemize}

    \item \textbf{Procesamiento de datos experimentales}:
    \begin{itemize}
        \item Análisis de cromatografías GC-FID
        \item Cálculo de conversiones y rendimientos
        \item Generación de gráficas publication-ready
    \end{itemize}

    \item \textbf{Prototipado rápido}:
    \begin{itemize}
        \item Probar diferentes modelos cinéticos
        \item Validar conceptos antes de escalar
        \item Comparar alternativas de reacción
    \end{itemize}

    \item \textbf{Educación}:
    \begin{itemize}
        \item Enseñar fundamentos de cinética química
        \item Entender ecuaciones diferenciales de reactores
        \item Aprender programación científica
    \end{itemize}

    \item \textbf{Integración con machine learning}:
    \begin{itemize}
        \item Predicción de propiedades
        \item Regresión simbólica
        \item Optimización bayesiana
    \end{itemize}
\end{enumerate}
\end{infobox}

\subsection{Limitaciones del Modelo Standalone}

\begin{infobox}{¿Para qué NO es adecuado el modelo standalone?}

\textbf{Casos donde se requiere HYSYS u otro simulador}:

\begin{enumerate}
    \item \textbf{Diseño de plantas industriales}:
    \begin{itemize}
        \item Flowsheets completos con múltiples equipos
        \item Dimensionamiento de equipos
        \item Cálculos de costos de capital
        \item Certificación para construcción
    \end{itemize}

    \item \textbf{Termodinámica compleja}:
    \begin{itemize}
        \item Sistemas con fases múltiples (líquido-líquido-vapor)
        \item Cálculos rigurosos de equilibrio
        \item Azeótropos
        \item Propiedades a alta presión
    \end{itemize}

    \item \textbf{Operaciones unitarias avanzadas}:
    \begin{itemize}
        \item Torres de destilación con múltiples etapas
        \item Extractores líquido-líquido
        \item Sistemas de refrigeración
        \item Redes de intercambiadores de calor
    \end{itemize}

    \item \textbf{Simulación dinámica rigurosa}:
    \begin{itemize}
        \item Control de procesos
        \item Arranque/parada de plantas
        \item Análisis de seguridad (HAZOP)
    \end{itemize}

    \item \textbf{Análisis económico detallado}:
    \begin{itemize}
        \item CAPEX, OPEX
        \item VAN, TIR
        \item Optimización económica
    \end{itemize}
\end{enumerate}
\end{infobox}

\subsection{Alcances de ASPEN HYSYS}

ASPEN HYSYS es la herramienta de elección para:

\begin{itemize}
    \item Diseño completo de plantas de biodiésel industriales
    \item Simulación de procesos de purificación (lavado, destilación de metanol)
    \item Integración energética (pinch analysis)
    \item Evaluación económica completa
    \item Validación de diseño para construcción
\end{itemize}

\subsection{Limitaciones de ASPEN HYSYS}

ASPEN HYSYS tiene dificultades en:

\begin{itemize}
    \item Implementar modelos cinéticos no estándar (requiere UDF en C/Fortran)
    \item Procesamiento de datos experimentales (debe hacerse externamente)
    \item Optimización multi-objetivo compleja
    \item Integración con herramientas modernas de ciencia de datos
    \item Reproducibilidad científica (archivos binarios)
\end{itemize}

%==============================================================================
\section{Casos de Uso Recomendados}
%==============================================================================

\subsection{Flujo de Trabajo Híbrido (Recomendado)}

\begin{infobox}{Mejor práctica: Combinar ambos enfoques}

\textbf{Etapa 1: Desarrollo de modelo (Standalone)}
\begin{enumerate}
    \item Procesar datos experimentales GC-FID
    \item Ajustar parámetros cinéticos (A, Ea)
    \item Validar modelo con datos experimentales
    \item Optimizar condiciones de reacción
\end{enumerate}

\textbf{Etapa 2: Validación cruzada (Standalone + HYSYS)}
\begin{enumerate}
    \item Implementar modelo en HYSYS con parámetros ajustados
    \item Comparar resultados (RMSE, R²)
    \item Verificar consistencia termodinámica
    \item Ajustar si es necesario
\end{enumerate}

\textbf{Etapa 3: Scale-up y diseño (HYSYS)}
\begin{enumerate}
    \item Diseñar flowsheet completo en HYSYS
    \item Agregar operaciones de separación y purificación
    \item Optimizar integración energética
    \item Evaluar economía del proceso
\end{enumerate}

\textbf{Etapa 4: Análisis avanzado (Standalone)}
\begin{enumerate}
    \item Análisis de sensibilidad detallado
    \item Superficie de respuesta
    \item Machine learning para predicciones
    \item Generación de gráficas para publicación
\end{enumerate}
\end{infobox}

\subsection{Árbol de Decisión}

\begin{figure}[H]
\centering
\begin{minipage}{0.9\textwidth}
\textbf{¿Qué herramienta usar?}

\begin{enumerate}[label=\textbf{\arabic*.}]
    \item ¿Tienes licencia de HYSYS?
    \begin{itemize}
        \item \textbf{NO} → Usar Standalone
        \item \textbf{SÍ} → Continuar
    \end{itemize}

    \item ¿Es investigación académica?
    \begin{itemize}
        \item \textbf{SÍ} → Standalone (+ HYSYS para validación)
        \item \textbf{NO} → Continuar
    \end{itemize}

    \item ¿Necesitas diseñar planta completa?
    \begin{itemize}
        \item \textbf{SÍ} → HYSYS (+ Standalone para cinética)
        \item \textbf{NO} → Continuar
    \end{itemize}

    \item ¿Modelo cinético no estándar?
    \begin{itemize}
        \item \textbf{SÍ} → Standalone
        \item \textbf{NO} → HYSYS
    \end{itemize}

    \item ¿Requieres termodinámica rigurosa?
    \begin{itemize}
        \item \textbf{SÍ} → HYSYS
        \item \textbf{NO} → Standalone
    \end{itemize}

    \item ¿Necesitas compartir código/resultados?
    \begin{itemize}
        \item \textbf{SÍ} → Standalone
        \item \textbf{NO} → HYSYS
    \end{itemize}
\end{enumerate}
\end{minipage}
\end{figure}

%==============================================================================
\section{Comparación de Precisión}
%==============================================================================

\subsection{Validación Experimental}

Ambos enfoques pueden alcanzar alta precisión si se configuran correctamente:

\begin{table}[H]
\centering
\caption{Métricas de ajuste típicas (datos de transesterificación con CaO)}
\begin{tabular}{lcc}
\toprule
\textbf{Métrica} & \textbf{Standalone} & \textbf{HYSYS} \\
\midrule
R² & 0.985 - 0.995 & 0.980 - 0.995 \\
RMSE (\%) & 1.5 - 3.0 & 1.8 - 3.5 \\
MAE (\%) & 1.0 - 2.5 & 1.2 - 2.8 \\
\bottomrule
\end{tabular}
\end{table}

\textbf{Observaciones}:
\begin{itemize}
    \item Standalone: Precisión controlable (depende del modelo implementado)
    \item HYSYS: Precisión limitada por modelos predefinidos
    \item Diferencias < 5\% en la mayoría de casos
\end{itemize}

\subsection{Fuentes de Error}

\begin{table}[H]
\centering
\caption{Fuentes de error en cada enfoque}
\begin{tabular}{p{7cm}p{7cm}}
\toprule
\textbf{Standalone} & \textbf{HYSYS} \\
\midrule
Implementación de ODEs (errores numéricos) & Conversión batch→continuo \\
Propiedades termodinámicas aproximadas & Selección de paquete termodinámico \\
Factores de respuesta GC-FID & Definición de componentes \\
Condiciones iniciales & Convergencia de flash \\
Método de integración (tolerancias) & Parámetros de reactor (tau, V) \\
\bottomrule
\end{tabular}
\end{table}

%==============================================================================
\section{Análisis de Costos}
%==============================================================================

\subsection{Costos Directos}

\begin{table}[H]
\centering
\caption{Análisis de costos (valores estimados)}
\begin{tabular}{lrr}
\toprule
\textbf{Concepto} & \textbf{Standalone} & \textbf{HYSYS} \\
\midrule
\textbf{Costos iniciales} & & \\
Licencia software & \$0 & \$25,000 \\
Hardware (PC) & \$800 & \$1,200 \\
Capacitación & \$500 & \$3,000 \\
\textbf{Subtotal inicial} & \$1,300 & \$29,200 \\
\midrule
\textbf{Costos anuales} & & \\
Mantenimiento licencia & \$0 & \$5,000 \\
Actualizaciones & \$0 & \$2,000 \\
Soporte técnico & \$0 & \$3,000 \\
\textbf{Subtotal anual} & \$0 & \$10,000 \\
\midrule
\textbf{Total 5 años} & \$1,300 & \$79,200 \\
\bottomrule
\end{tabular}
\end{table}

\subsection{Costos Indirectos}

\begin{itemize}
    \item \textbf{Tiempo de desarrollo}:
    \begin{itemize}
        \item Standalone: 2-4 semanas (desarrollo inicial)
        \item HYSYS: 1-2 días (setup inicial)
    \end{itemize}

    \item \textbf{Curva de aprendizaje}:
    \begin{itemize}
        \item Standalone: 1-2 meses (Python + cinética)
        \item HYSYS: 2-3 meses (interfaz + termodinámica)
    \end{itemize}

    \item \textbf{Mantenimiento}:
    \begin{itemize}
        \item Standalone: Bajo (actualizaciones bibliotecas Python)
        \item HYSYS: Alto (actualizaciones de licencia obligatorias)
    \end{itemize}
\end{itemize}

%==============================================================================
\section{Reproducibilidad Científica}
%==============================================================================

\subsection{Importancia en Investigación}

La reproducibilidad es crítica para:
\begin{itemize}
    \item Validación de resultados por pares
    \item Publicación en revistas científicas
    \item Tesis de posgrado
    \item Transferencia de conocimiento
\end{itemize}

\subsection{Comparación de Reproducibilidad}

\begin{table}[H]
\centering
\caption{Aspectos de reproducibilidad}
\begin{tabular}{p{5cm}cc}
\toprule
\textbf{Aspecto} & \textbf{Standalone} & \textbf{HYSYS} \\
\midrule
Código fuente disponible & ✓ & ✗ \\
Versionable (Git) & ✓ & ✗ \\
Archivos texto plano & ✓ & ✗ (binario) \\
Independiente de software comercial & ✓ & ✗ \\
Ejecutable en cualquier PC & ✓ & ✗ (solo Windows) \\
Resultados bit-a-bit idénticos & ✓ & ± (depende de versión) \\
Compartible públicamente & ✓ & ✗ (requiere licencia) \\
Documentación integrada & ✓ (docstrings) & ± (manuales) \\
\bottomrule
\end{tabular}
\end{table}

\textbf{Conclusión}: El modelo standalone es significativamente superior en reproducibilidad científica.

%==============================================================================
\section{Recomendaciones por Perfil de Usuario}
%==============================================================================

\subsection{Estudiante de Posgrado (Maestría)}

\textbf{Recomendación}: \textbf{Modelo Standalone}

\textbf{Justificación}:
\begin{itemize}
    \item Presupuesto limitado (sin acceso a licencia HYSYS)
    \item Enfoque en aprendizaje de fundamentos
    \item Necesidad de flexibilidad para experimentar
    \item Requisito de reproducibilidad para tesis
    \item Publicación en revistas open-access
\end{itemize}

\textbf{Uso de HYSYS}: Solo si la universidad tiene licencia, para validación cruzada.

\subsection{Estudiante de Doctorado}

\textbf{Recomendación}: \textbf{Standalone + HYSYS} (híbrido)

\textbf{Justificación}:
\begin{itemize}
    \item Desarrollo de modelos cinéticos propios (Standalone)
    \item Validación con herramienta industrial (HYSYS)
    \item Publicaciones requieren validación rigurosa
    \item Posible colaboración con industria
\end{itemize}

\subsection{Investigador Académico}

\textbf{Recomendación}: \textbf{Standalone} (con posible validación en HYSYS)

\textbf{Justificación}:
\begin{itemize}
    \item Desarrollo de modelos innovadores
    \item Publicaciones científicas (open-source preferido)
    \item Colaboración con otros grupos (código compartible)
    \item Integración con machine learning y otras herramientas
\end{itemize}

\subsection{Ingeniero en Industria}

\textbf{Recomendación}: \textbf{ASPEN HYSYS}

\textbf{Justificación}:
\begin{itemize}
    \item Diseño de plantas industriales
    \item Certificación requerida
    \item Estándar de la industria
    \item Soporte comercial garantizado
    \item Integración con otras herramientas Aspen
\end{itemize}

\textbf{Uso de Standalone}: Para desarrollo de modelos cinéticos específicos antes de implementar en HYSYS.

\subsection{Consultor / Ingeniero de Proyectos}

\textbf{Recomendación}: \textbf{HYSYS} (+ Standalone para casos especiales)

\textbf{Justificación}:
\begin{itemize}
    \item Clientes requieren herramientas certificadas
    \item Rapidez en generación de propuestas
    \item Análisis económico integrado
    \item Reputación del software en la industria
\end{itemize}

%==============================================================================
\section{Conclusiones}
%==============================================================================

\subsection{Síntesis del Análisis}

\begin{enumerate}
    \item \textbf{No hay un enfoque universalmente superior}: La elección depende del contexto, objetivos y recursos.

    \item \textbf{Modelo Standalone}:
    \begin{itemize}
        \item Ideal para investigación, educación y desarrollo de modelos
        \item Superior en flexibilidad, costo y reproducibilidad
        \item Limitado en termodinámica rigurosa y diseño industrial
    \end{itemize}

    \item \textbf{ASPEN HYSYS}:
    \begin{itemize}
        \item Estándar para diseño industrial
        \item Superior en termodinámica y biblioteca de equipos
        \item Limitado por costo y falta de transparencia
    \end{itemize}

    \item \textbf{Enfoque híbrido}: Combinar ambos aprovecha fortalezas de cada uno

    \item \textbf{Tendencia futura}: Mayor uso de herramientas open-source en investigación, HYSYS sigue dominando en industria
\end{enumerate}

\subsection{Criterios de Selección Clave}

\begin{table}[H]
\centering
\caption{Criterios decisivos para selección de herramienta}
\begin{tabular}{ll}
\toprule
\textbf{Si tu prioridad es...} & \textbf{Usar...} \\
\midrule
Costo cero & Standalone \\
Reproducibilidad científica & Standalone \\
Flexibilidad en modelos & Standalone \\
Aprendizaje de fundamentos & Standalone \\
Termodinámica rigurosa & HYSYS \\
Diseño de planta industrial & HYSYS \\
Certificación & HYSYS \\
Análisis económico & HYSYS \\
Educación de posgrado & Standalone + HYSYS \\
Consultoría industrial & HYSYS + Standalone \\
\bottomrule
\end{tabular}
\end{table}

\subsection{Mensaje Final}

\begin{infobox}{Recomendación Principal}
\textbf{Para máximo impacto científico e industrial}:

\begin{enumerate}
    \item \textbf{Desarrollar} el modelo cinético en Python (Standalone)
    \item \textbf{Validar} con datos experimentales exhaustivos
    \item \textbf{Verificar} con ASPEN HYSYS (si disponible)
    \item \textbf{Publicar} código Python en repositorio público (GitHub, Zenodo)
    \item \textbf{Escalar} a diseño industrial usando HYSYS con parámetros validados
\end{enumerate}

Este enfoque combina:
\begin{itemize}
    \item Rigor científico (código reproducible)
    \item Validación industrial (HYSYS)
    \item Máximo impacto (open-source + aplicación práctica)
\end{itemize}
\end{infobox}

%==============================================================================
% FIN DEL DOCUMENTO
%==============================================================================

\vspace{2cm}
\begin{center}
\rule{\textwidth}{0.4pt}

\textbf{El mejor modelo es el que responde a tus preguntas}\\
\textit{con la precisión necesaria al menor costo posible.}

\rule{\textwidth}{0.4pt}
\end{center}

\end{document}
